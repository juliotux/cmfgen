\documentclass[12pt]{report}
%\usepackage{epsfig}
%\usepackage{times}
\usepackage[usenames,dvipsnames]{color}
\usepackage{fix-cm}
\usepackage{tabularx}
\usepackage{needspace}

\usepackage{hyperref}
\hypersetup{linkbordercolor={1 1 1}} 
%colorlinks=false,linkcolor=white}
%\hypersetup{colorlinks,citecolor=black, filecolor=black, linkcolor=black,urlcolor=black}
%\hypersetup{linktocpage}

%\input ladef.tex
\def\blankline{\par\vskip 12pt\noindent}
\def\blankhalf{\par\vskip 6pt\noindent}
\textheight 9.0 true in
\textwidth 6.5 true in
\topmargin -0.8 true in
\oddsidemargin 0.0 true in
\evensidemargin 0.0 true in

\newcommand{\Mdot}{\hbox{$\dot M$}}
\newcommand{\Rsun}{\hbox{$R_\odot$}}
\newcommand{\Rstar}{\hbox{$R_*$}}
\newcommand{\Lsun}{\hbox{$L_\odot$}}
\newcommand{\Msun}{\hbox{$M_\odot$}}
\newcommand{\Msunyr}{\hbox{$M_\odot\,$yr$^{-1}$}}
\newcommand{\Teff}{\hbox{$T_{\rm \scriptsize eff}$}}
\newcommand{\Vinf}{\hbox{$V_\infty$}}
\newcommand{\kms}{\hbox{km$\,$s$^{-1}$}}
\newcommand{\tauross}{\hbox{$\tau_{\hbox{\rm \scriptsize Ross}}$}}
\newcommand{\chiross}{\hbox{$\chi_{\hbox{\rm \scriptsize Ross}}$}}
\newcommand{\Vo}{\hbox{$V_{\hbox{\rm \scriptsize o}}$}}
\newcommand{\Vcore}{\hbox{$V_{\hbox{\rm \scriptsize core}}$}}
\newcommand{\Psub}[2]{\hbox{$#1_{\hbox{\rm \scriptsize #2}}$}}
 
\newcommand{\mhd}[2]{ {\centerline{\large \color{red} \bf #1}}  \label{sec_#2}  \vskip 3pt}
\newcommand{\shd}[2]{\needspace{4 \baselineskip} {\centerline{\color{red} #1}}  \label{sec_#2}}
\newcommand{\ind}[2]{#1 .\dotfill{} \pageref{sec_#2}}
\newcommand{\sind}[2]{\indent {\it \small  \lowercase{#1}} .\dotfill{} \pageref{sec_#2}}
\newcommand{\myref}{\par\vskip 5 pt \indent\hangindent=40pt\hangafter=1}
\newcommand{\script}[1]{\par\vskip 5 pt \indent\hangindent=40pt\hangafter=1 {\color{blue} #1}\\ }
\newcommand{\optdesc}[1]{\par\vskip 8 pt \indent\hangindent=50pt\hangafter=1 {\color{blue} \textbf{#1}}}

\newcommand{\bbf}[1]{{\color{Brown} \bf #1}}

\newcommand{\case}[1]{\vskip 15pt  \addtocounter{casecount}{1}
\centerline{\color{blue} \textbf{Case \arabic{casecount}: #1}} }

%\newenvironment{file_list}[1]
% {\begin{list}{}{\settowidth{\labelwidth}{\textbf{#1}}
%   \setlength{\leftmargin}{\labelwidth}
%   \addtolength{\leftmargin}{\labelsep}
%   \renewcommand{\makelabel}[1]{\color{blue} \textbf{\hfill##1}}}}%
%  {\end{list}}

% file_list

\newenvironment{file_list}[1]
{\begin{list}{}{\settowidth{\labelwidth}{\textbf{MOD}} %\textbf{#1}}
 \setlength{\leftmargin}{\labelwidth}
  \addtolength{\leftmargin}{\labelsep}
   %\setlength{\labelsep}{3 in}
 \renewcommand{\makelabel}[1]{\color{blue} \textbf{##1} \hspace{8 in} \hbox{}}}\allowbreak}%
  {\end{list}}
  
% green_list 
  
\newenvironment{green_list}[1]
 {\begin{list}{}{\settowidth{\labelwidth}{\textbf{#1}}
   \setlength{\leftmargin}{\labelwidth}
   \addtolength{\leftmargin}{\labelsep}
   \renewcommand{\makelabel}[1]{\color{OliveGreen} \textbf{\hfill##1}}}}%
  {\end{list}}

% indented_list

\newenvironment{indented_list}[1]
 {\begin{list}{}{\settowidth{\labelwidth}{\textbf{#1}}
   \setlength{\leftmargin}{\labelwidth}
   \setlength{\itemsep}{-4pt}
   \addtolength{\leftmargin}{\labelsep}
   \renewcommand{\makelabel}[1]{\color{OliveGreen} \textbf{\hfill [##1]\ }}}}%
  {\end{list}}
  
% plt_list

\newenvironment{plt_list}[1]
 {\begin{list}{}{\settowidth{\labelwidth}{\textbf{#1}}
   \setlength{\leftmargin}{\labelwidth}
   \setlength{\itemsep}{-4pt}
   \addtolength{\leftmargin}{\labelsep}
   \renewcommand{\makelabel}[1]{\color{OliveGreen} \textbf{\hfill##1}}}}%
  {\end{list}}
  
%plt_list_2nd

  \newenvironment{plt_list_2nd}[1]
 {\begin{list}{}{\settowidth{\labelwidth}{\textbf{#1}}
   \setlength{\leftmargin}{\labelwidth}
   \setlength{\itemsep}{-4pt}
   \addtolength{\leftmargin}{\labelsep}
   \renewcommand{\makelabel}[1]{\color{Brown} \textbf{\hfill##1}}}}%
  {\end{list}}
  
\newenvironment{dir_list}[1]%
 {\begin{list}{}{\settowidth{\labelwidth}{\textbf{#1}}
   \setlength{\leftmargin}{\labelwidth}
   \addtolength{\leftmargin}{\labelsep}
   \renewcommand{\makelabel}[1]{\color{blue} \textbf{\hfill##1} \hspace{8 in} \hbox{} }}}%
  {\end{list}}

% scriptlist

\newenvironment{scriptlist}[1]%
 {\begin{list}{}{\settowidth{\labelwidth}{\textbf{#1}}
   \setlength{\leftmargin}{\labelwidth}
   \addtolength{\leftmargin}{\labelsep}
   \renewcommand{\makelabel}[1]{\color{blue} \textbf{\hfill##1}}}}%
  {\end{list}}

%mylist
  
  \newenvironment{mylist}[1]%
  {\begin{list}{}{
   \renewcommand{\makelabel}[1]{\color{blue} \textbf{\hfill[##1]}}}}%
  {\end{list}}
  
  \newenvironment{myinlist}[1]%
  {\begin{list}{}{
   \renewcommand{\makelabel}[1]{\color{OliveGreen} \textbf{\hfill[##1]}}}}%
  {\end{list}}

\newenvironment{mynarrow}[1]{\addtolength{\leftskip}{#1}\parindent=0pt
\addtolength{\leftskip}{#1}}

\begin{document}
%\vskip  -0.7 in

%{\fontsize{50}
{
{ \it  \fontsize{80}{100} \selectfont  \color{blue}{  \centerline{CMFGEN}
{\hfill MANUAL \hfil} \\ 
\vskip 20 pt 
}
{\it \fontsize{60}{75} \selectfont  \color{black}{
\centerline{\today}


\vskip 15pt
\centerline{D. John Hillier}}
 }}
 }
 
%}

\vfill
\eject

\mhd{Index}{}
\label{sec_Index} 

\noindent
\ind{INDEX}{Index}      \\
\ind{INTRODUCTION}{introduction}  \\
\ind{PROGRAM NAMES}{prog_names} \\
\ind{SPECIES NAMES}{spec_names} \\
\ind{PROGRAM UNITS}{prog_units} \\
\ind{MAIN VARIABLES}{main_var} \\
\ind{VERSIONS}{version} \\
\ind{MEMORY}{memory} \\
\ind{EXECUTION TIME}{timing} \\
\ind{NG ACCELERATION}{ng_accel} \\
\ind{FIXING THE LINEARIZATION MATRIX (BA)}{fix_ba} \\
\ind{CONVERGENCE}{convergence} \\
\ind{FEATURES AND BUGS}{bugs} 
\blankhalf
\ind{INSTALLATION}{install} \\
\sind{MAKEFILE}{make} \\
\sind{MAKEFILE SYSTEM DEPENDENCIES}{make_depend} \\
\sind{POSSIBLE COMPILATION PROBLEMS}{compile_prob} \\
\sind{PARALLELIZATION}{parallel} \\
\sind{DIRECTORY STRUCTURE}{dir_struc} \\
\ind{ATOMIC DATA}{atomic_data} \\
\sind{ATOMIC DATA PROGRAMS}{atomic_data_progs} 
\blankhalf
\ind{USEFUL SCRIPTS}{scripts} \\
\ind{MODEL COMPUTATION}{mod_comp} \\
\sind{COMPUTATIONAL CHECKLIST FOR STARTING A MODEL}{start_mod} \\
\sind{COMPUTATIONAL CHECKLIST FOR RUNNING/CONVERGED MODEL}{comp_check} \\
\sind{TROUBLESHOOTING}{troub_shoot} \\
\ind{CMFGEN FILES}{cmfgen_files} \\
\sind{MAIN MODEL CONTROL FILES}{input} \\
\sind{OTHER CONTROL FILES}{other_control} \\
\sind{MODEL INPUT FILES}{input} \\
\sind{ATOMIC DATA FILES}{atomic_files} \\
\sind{DIAGNOSTIC FILES}{diag_files} \\
\sind{OUTPUT FILES}{output_files} \\
\sind{SCRATCH FILES}{scratch_files} \\
\blankhalf
\ind{EXPLANATION OF FIELDS IN MODEL\_SPEC}{model_spec} 

\blankhalf
\ind{EXPLANATION OF OPTIONS IN VADAT}{vadat_opts} \\
   \sind{OPTIONS FOR ATMOSPHERIC STRUCTURE}{atm_struct} \\
   \sind{OPTIONS FOR CONTINUUM FREQUENCY GRID}{cont_freq_grid} \\
   \sind{COMPUTATION OF RADIATION FIELD}{comp_rad_field} \\
   \sind{ACCURACY OPTIONS}{act_opt} \\
   \sind{LINE OPTIONS}{line_opts} \\    
   \sind{OPTIONS FOR CALCULATION OF SPECTRUM IN OBSERVER'S FRAME}{obs_spec} \\
   \sind{Options controlling the treatment of bound-bound transitions}{control_bb_trans} \\
   \sind{Options controlling the heating and cooling due to line terms}{line_cool} \\
   \sind{PHYSICAL PROCESSES TO BE INCLUDED}{phys_proc} \\
   \sind{OPTIONS FOR BEGINNING A NEW MODEL}{new_mod} \\
   \sind{Option specifying method of handling lines}{line_handling} \\
   \sind{OPTIONS FOR ASSISTING CONVERGENCE}{assist_conv} \\
   \sind{OPTIONS FOR CONTROLLING THE LINEARIZATION}{linear} \\
   \sind{OPTIONS FOR AN ENHANCED SPATIAL GRID}{spat_grid} \\
   \sind{EDDINGTON AND BA COMPUTATION}{edd_comp} \\
   \sind{PARAMETERS TO CONTROL Ng ACCELERATIONS}{ng_accel_params} \\
   \sind{SN MODEL OPTIONS}{sn_options} \\
   \sind{Additional options for beginning a NEW SN model}{new_sn} \\
   \sind{Non-thermal model options}{non-thermal} \\
   
\blankhalf
\centerline{\color{red} AUXILIARY PROGRAMS}

\blankhalf
\ind{CMF\_FLUX}{cmf_flux} \\
  \sind{Running CMF\_FLUX}{run_cmf_flux} \\
  \sind{Input needed for CMF\_FLUX:}{out_cmf_flux} \\
  \sind{Output from CMF\_FLUX:}{out_cmf_flux} \\
  \sind{Explanation of options in CMF\_FLUX\_PARAM}{cmf_flux_param} \\

\noindent
\ind{DISPGEN}{dispgen} \\
\ind{PLT\_SPEC}{plt_spec} \\
\ind{TLUSTY\_VEL}{tlusty} \\
\ind{GRAMON\_PGPLOT}{gramon_pgplot} \\
\ind{MAIN\_LTE}{main_lte} \\
\ind{DO\_NG\_V2}{do_ng_v2} \\
\ind{GUESS\_DC}{guess_dc} \\
\ind{LAND\_COL\_MERGE}{land_col_merge} \\
\ind{MOD\_COOL}{mod_cool}\\
\ind{MOD\_PRRR}{mod_prrr} \\
\ind{N\_COL\_MERGE}{n_col_merge} \\
\ind{PLT\_CMF\_LUM}{plt_cmf} \\
\ind{PLT\_IP}{plt_ip} \\
\ind{PLT\_JH}{plot_jh} \\
\ind{PLT\_JH\_CUR}{plot_jh_cur}\\
\ind{PLT\_RJ}{plot_rj} \\
\ind{PLT\_SCR}{plt_scr} \\
\ind{REWRITE\_DC}{rewrite_dc} \\
\ind{WIND\_HYD}{wind_hyd} \\
\ind{WR\_F\_TO\_S}{wr_f_to_s} \\
 
\vfill
\eject
\mhd{Introduction}{introduction}
              
CMFGEN is a radiative transfer code designed to solve the radiative transfer and statistical equilibrium equations in plane-parallel or spherical geometry. It has been designed for applications to Wolf-Rayet (W-R) stars, O stars, Luminous Blue Variables (LBVs) and supernovae (SNe). For stars with stellar winds the mass-loss rate and velocity law (above the sonic point) must be specified --- at present we cannot solve for them self-consistently. The hydrostatic structure, below the photosphere, can be determined self-consistently. \\

\noindent
Three main modes are available:

\begin{mynarrow}{0.1in}
\blankhalf
{\color{blue} Blanketing:} 
The effect of line overlap, and the effect of lines on the continuous energy distribution, is explicitly taken into account. It is the preferred mode that will give the most accurate results. It is this mode that has been continuously used, and is the least likely to suffer from
a problem due to an update in CMFGEN.

\blankhalf
{\color{blue} Sobolev:} 
Bound-bound transitions are treated using the Sobolev approximation. Okay for fast-dirty models. Model is unblanketed, although it allows for collisional cooling. 

\blankhalf
{\color{blue} CMF:} 
Bound-bound transitions are computed as individual lines in the comoving-frame. Obsolete, and may no longer work correctly. Use Blanketing mode for accurate calculations, Sobolev mode for fast and crude calculations.
\end{mynarrow}

\blankhalf
It is possible to treat species in different modes simultaneously. This is primarily useful for flux calculations (done with CMF\_FLUX).
For example, a spectrum can be calculated showing only H\,{\sc i} line transitions. Alternatively, a spectrum can be computed
with all  H\,{\sc i} line transitions omitted. These two options provide a mechanism to determine the direct influence of a species
on the spectrum.

\blankhalf
Several different options/assumptions are available to compute the radiation field. 
\begin{enumerate}
\item
Solve the transfer equation for spherical geometry in the comoving-frame (default); this is the mode used when studying massive stars and their stellar winds.
\item
Solve the static transfer equation in the plane parallel approximation.
\item
Solve the transfer equation in the plane parallel approximation with  a vertical velocity field (zero order in v/c).
\item
Solve the static transfer code for a spherical atmosphere allowing for all relativistic terms.
\item
Solve the time-dependent spherical transfer equation to first order in $v/c$ for a Hubble flow.
\item
Solve the time-dependent spherical transfer equation for all orders in $v/c$ (still under development).
\end{enumerate}

\noindent
The last 3 options were primarily developed for work on supernovae.

\vfill
\eject
\blankline
\noindent
{\color{red} References}

\myref
Busche, J., Hillier, D. J., 2005, AJ, 129, 454 \\
    Spectroscopic Effects of Rotation in Extended Stellar Atmospheres.

\myref
Hillier, D. J., 2003, {\it in Stellar Atmosphere Modeling}, ASP. Conf. Ser., 288, 199,
eds. {Hubeny}, I. and {Mihalas}, D. and {Werner}, K., p.~199 \\
On the Solution of the Statistical Equilibrium Equations.

\myref
Hillier, D. J., Miller, D., 1999, ApJ, 519, 354 \\
Constraints on the Evolution of Massive Stars through Spectral Analysis. I. The WC5 Star HD165763.

\myref
Hillier, D. J., Miller, D., 1998, ApJ, 496, 407 \\
The treatment of non-LTE line blanketing in spherically expanding outflows.

\myref
Hillier, D. J., 1990, A\&A, 231, 116 \\
An iterative method for the solution of the statistical and radiative equilibrium equations in expanding atmospheres.

\myref
Hillier, D.J. 1987, ApJS, 63, 947. \\ 
Modeling the extended atmospheres of WN stars

\blankline
\noindent
{\color{red} Acknowledgements}

\blankhalf
The author gratefully acknowledges support for CMFGEN from acknowledges support from STScI theory grant HST-AR-11756.01.A and NASA theory grant NNX10AC80G. I would also like to thank Paco Najarro and Doug Miller for their direct contributions to CMFGEN, and Jean-Claude Bouret, Paul Crowther, Luc Dessart, Leo Gerogiev, Ivan Hubeny, Thierry Lanz, Kathryn Neugent, Fabrice  Martins, Phil Massy, Stan Owocki and many others for discussions on  radiative transfer, massive stars and their winds, and supernovae.

The author also would like to acknowledge the numerous workers who have made atomic data calculations, and
who have made that data publicly available. I would especially like to thank Keith Butler, Robert~Kurucz, Sultana Nahar, Anil Pradhan, 
and Peter Storey.

\vfill
\eject
\mhd{Program Names}{prog_names}

\begin{table}[h]
\begin{tabular}{ll}              
cmfgen\_dev.exe \hspace{0.37in}       & Main program to compute atmospheric structure. \\
cmf\_flux.exe        & Main program to compute observed spectrum. \\
dispgen.exe        & General display and diagnostic package. \\
plt\_spec.exe        & Plot package to treat theoretical and observed spectra. \\
                              &                                                                                         \\
                              &                                                                                         \\
gramon\_pgplot.f    &     General plotting package (called by other routines). \\
\end{tabular}
\end{table}

\begin{table}  %[h]
\begin{tabular}{ll}
chk\_phot.exe &  Simple routine to check photoionization data file.\\               
count\_phot\_data.exe & Count number of levels and data values in a photoionization file.\\
do\_ng\_v2.exe         & Manual acceleration routine. \\
guess\_dc.exe           & Guess departure coefficients for a new model. \\
land\_col\_merge.exe    & Merge postscript files into a single file (landscape format). \\
land\_multi\_merge.exe    & Merge two postscript files into a single file (row/columns). \\
main\_lte.exe        & Compute Rosseland mean opacities. \\
mod\_cool.exe      & Creates summary of GENCOOL (heating./cooling rates). \\
mod\_prrr.exe        & Creates summary of PRRRXzV file (recombination/ionization) for \\
                               & \hspace{0.5 in} a given species. \\
n\_col\_merge.exe     &     Merge N postscript files into a single file. \\
n\_multi\_merge.exe     &     Merge N postscript files into a single file (row/columns). \\
rev\_rvsing.exe             & Modify RVS\_COL file. Can modify file for change in \\
                                      & mass-loss rate, velocity law, number of depth points,\\
                                      &  grid spacing etc. \\
rev\_rdinr.exe             & Modify file (departure coefficient format) that can be used \\
                                   &to define the R grid (RVS\_COL cannot be in use). \\
                                      & velocity law, number of depth points, grid spacing etc. \\
                                      &Many options.\\                                      
rewrite\_dc.exe          & Rewrite departure coefficient file.  \\
rewrite\_scr.exe        & Rewrite SCRTEMP to reduce its size (default is final 2 iterations) \\
                                           & \hspace{0.5 in} or change its format. \\
set\_new\_sn\_mod.exe        & Changes SN associated options in VADAT file making it easier \\
                                           & to go from time step 1 to 2. Can also be used for later time steps. \\
plt\_dfr.exe            & Plot the origin of the observed flux (as computed by cmf\_flux.exe) \\
                            & \hspace{0.5 in}  as a function of depth. \\
plt\_jh.exe            & Plot {\bf J, H, $\chi$, $\eta$} etc from files with EDDFACTOR like formats. H, $\chi$,  \\ 
                                  & \hspace{0.5 in} and $\eta$ files can be created by cmf\_fluxe.exe \\
plt\_jh\_cur.exe            & Similar to plt\_jh.exe, but access the JH file produced for SN models. \\ 
plt\_ip.exe &         Plot/examine {\bf I} as a function of impact parameter and frequency. \\
plt\_phot\_raw.exe        & Plot cross-sections/check recombination data from photoionization \\
                                    & file(s). Photoionization cross-sections can also be plotted in DISPGEN. \\
plt\_scr.exe        &     Plot SCRTEMP file. \\
tlusty\_vel.exe        & Create a hydrostatic structure from a TLUSTY model. \\
wind\_hyd.exe        & Create a wind with a theoretical hydrostatic structure. \\
wr\_f\_to\_s.exe    &     Create and modify links between levels and super levels
\end{tabular}
\end{table}


\blankline
\vfill
\eject
\mhd{Species Names}{spec_names}

\blankhalf
Each atomic species has two abbreviations associated with it. These abbreviations are set in cmfdist/new\_main/mod/mod\_cmfgen.f. The first abbreviation (e.g., CARB) refers to all ionization stages, and is used to set abundances etc. The second is used as the prefix for each ionization stage. The following abbreviations are in use. For consistency, and for the ease of moving models between users, it is best to stick with these definitions.  {\bf In this documentation, DUM is used to refer to the species (e.g., CARB) while XzV refers to a particular ionization stage (e.g., C\,{\sc iv})}. 

\blankhalf
\begin{mynarrow}{0.5 in}
\begin{tabular}{lll}
       Hydrogen \hspace{30pt}  &  HYD   \hspace{25pt} &   H   \\
       Helium   &   HE   &    He   \\
       Carbon   &   CARB   &  C    \\
       Oxygen   &   OXY   &   O    \\
       Nitrogen   &   NIT   &   N    \\
     Neon   &     NEON   &  Ne   \\
       Sodium   &   SOD   &   Na   \\
       Magnesium   &  MAG   &   Mg   \\
       Aluminum  &    ALUM   &  Al   \\
     Silicon   &    SIL   &   Sk  \\
       Phosphorous &  PHOS   &  P  \\
       Sulfur   &   SUL   &   S    \\
       Argon   &    ARG   &   Ar   \\
       Calcium   &  CAL   &   Ca  \\
       Titanium   &   TIT   &   Tk  \\
       Vanadium   &   VAN   &   V  \\
       Chromium    &  CHRO   &  Cr  \\
       Manganese   &  MAN   &   Mn   \\
       Iron   &     IRON   &  Fe  \\
       Cobalt   &   COB   &   Co  \\
       Nickel      & NICK   &  Nk   \\
\end{tabular}
\end{mynarrow}


\blankhalf
Successive ionization stages are referred to as 
\blankhalf
\hbox{}  \hspace{1in}  {\bf I, 2, III, IV, V, SIX, SEV, VIII, IX, X,} \\ 
 \hbox{} \hspace{1in}  {\bf XI, XII, XIII, XIV, XV, XSIX, XSEV, X8, X9, and XX}.
\blankhalf
 This mixed nomenclature was developed to facilitate code development, since it was easy to inadvertently write II instead of III etc. This is no longer of major concern, but have been left for consistency with earlier models.

Sk is used for Si (and Nk for Ni) since SiV could be interpreted as SIV. In a much older version we used Sx for Si but there was still the possibility of confusion since SxI will conflict with SXI. The changeover is transparent to new users, but users with very old model files may need to modify VADAT, MODEL\_SPEC, and batch.sh files for consistency. Also note that SxIV\_IN will become SkIV\_IN. DISPGEN will still read old Silicon files in which case you should still use the Sx nomenclature for all options. A line in the new RVTJ files indicates the adopted convention.


Instead of using H, He etc. the abbreviations H\_, He\_ etc. could be used. This should cause no problems in CMFGEN, and would avoid any naming ambiguities and confusion. Unfortunately �\_� is used in DISPGEN as a species separator in DISPGEN commands. The separator could be changed in DISPGEN to something else (For example, instead of  \\
 \hbox{} \hspace{0.5in}                      IF\_ARG you would type IF-ARG \\
  \hbox{} \hspace{0.5in}                     DC\_Ar2 you would type DC-Ar\_2 \\
This does increase the amount of typing in DISPGEN. This approach could also be handled automatically, with a different convention string in RVTJ. Some minor editing of CMFGEN etc. would be required. NB: DISPGEN is an integral part of CMFGEN and should be used as an examination tool.

The capitalization is important. While FORTRAN is generally case insensitive, string comparison commands are case sensitive. Thus FIX\_HeI must be specified in the VADAT file, not FIX\_HEI. Under VMS filenames are case insensitive while in UNIX file names are case sensitive. This can cause difficulties depending on how the files are opened.

\blankline
Options in the display packages, PLT\_SPEC and DISPGEN, are NOT case sensitive. However, the ``sve'' filenames (which save the last used parameters for an option) are case sensitive. 

\vfill
\eject
\mhd{Program Units}{prog_units}

\noindent
For historical reasons the following units have been adopted:

\blankline
\begin{mynarrow}{0.2in}
\begin{tabular}{lll}
Length Scales (e.g., R$_*$)    & $10^{10}$\,cm \\
Opacity [CHI]    & Chosen so that R. CHI is correct \\
Emissivity [ETA]  &    Chosen so that ETA/CHI is correct \\
Photoionization cross-sections:     & Megabarns (in atomic data files only) \\
Densities    &cm$^{-3}$ \\
Temperature    & 10$^4$\,K \\
Velocity  &     km\,s$^{-1}$ \\
Mass    & Solar mass: $1.989 \times 10^{33}$ gm. \\
Luminosity &    Solar luminosity: $3.826 \times 10^{33}$ erg s$^{-1}$ \\
Distances &    kpc: $3.0856 \times 10^{21}$ cm \\
                 &      \hspace{10pt} For PLT\_SPEC, default distance is 1\,kpc. \\
Fluxes    & Janskies:  1 Jy = $10^{-23}$ ergs/cm$^2$/sec/Hz \\
                 &  \hspace{10pt} In OBSFLUX,  fluxes assume d=1\,kpc \\
Frequency & $10^{15}$\,Hz
\end{tabular}
\end{mynarrow}

\blankline
\noindent
Adopted solar values 

\blankline
\begin{mynarrow}{0.2in}
\begin{tabular}{lll}
    Rsun \hspace{128pt } & $6.96 \times 10^{10}$\,cm     &  [$6.95508 \times 10^{10}$\,cm] \\
    Lsun & $3.826 \times10^{33}$\,erg\,s$^{-1}$  & [$3.845 \times 10^{33}$\,cm] \\
    Teff(sun) & $5784$ \,K     &  [5777\,K (5780\,K)] \\
    Msun &  $1.989 \times 10^{33}$\,gm. 
\end{tabular}
\end{mynarrow}
        
\blankline
The values in brackets are from AAQ(2000). An effective temperature of 5780\,K is calculated from Rsun and Lsun.
%Teff is only used for creating the MOD\_SUM file. 

\blankhalf
{\color{blue} Wavelengths:}   All plotting is done assuming vacuum wavelengths. Wavelength input can (generally) be for air (if $\lambda > 2000$\,\AA) or for vacuum. In the atomic data files, wavelengths are generally air for  $\lambda > 2000$\,\AA; otherwise vacuum. Be careful --- there is the potential for confusion. In CMFGEN itself, only frequencies are used, so there is no confusion.


\blankhalf
{\color{blue} Physical constants:}   Function calls (located in \$cmfdist/subs/phys\_con.f) return cgs values.

\blankhalf
{\color{blue} Abundances:}    Solar abundances are taken from Allen�s Astrophysical Quantities by Cox (2000) p.29-31. For future reference, the adopted solar values are output to MOD\_SUM. The  abundances have not been updated to more recent estimates, since their values (particularly CNO) are still uncertain.



\vfill
\eject
\mhd{Main Variables}{main_var}

{\narrower
\begin{table}[h]
\begin{tabular}{lll}
\hline
\hline
       Variable      &    Description             &     Typical Value \\
   \hline
       ND      &    Number of depth points         &         50 to 100 \\
       NC      &    Number of core rays            &     10 to 15 \\
       NP      &    Number of rays (angles)        &         ND+NC \\
       NUM\_BNDS&    Number of bands used for linearization    & 1 or 3 \\
       
       NCF\_MAX  &    Maximum number of (continuum) frequencies     & 5,000 to 300,000 \\
       NLINE\_MAX     & Maximum number of lines in model    &     100,000 \\
       MAX\_SIM    & Number of bound-bound transitions that \\
                          & \hspace{10pt} can be treated simultaneously  & 70 to 200 \\
       
       NXzV\_F    & Total number of levels in FULL ion XzV    & Ion dependent \\
       NXzV        & Number of super levels for ion XzV    &     $\le$ NXzV\_F \\
       NXzV\_IV    &  Number of important levels for ion XzV    & $\le$ NXzV \\
                   & Only used in cmfgen\_dev.exe \\     
       NT    &    Total number of unknown populations at each depth         &         100-2,000 \\
       NCF        &Number of (continuum) frequencies     &     5,000 to 100,000 \\
\hline
\end{tabular}
\end{table}
}

\noindent
Notes: 

\begin{enumerate}
\item
The first 10 entries (9 for cmf\_flux\_v5.exe) are specified, in the MODEL\_SPEC file, by the user.

\item
NT is the total number of unknown (super level) populations at each depth. It is the sum of the NXzV + number of atomic species + 2 (for T and Ne), and is computed by CMGEN.

\item
The preferred value of NUM\_BNDS is 3, which is a tradeoff between computational speed and memory. It (always) gives a good convergence with sufficient iterations, even without Ng acceleration. NUM\_BNDS=5 would reduce the total number of iterations (and possibly reduce execution time) but requires 50\% more memory. It has not been fully implemented in CMFGEN\_DEV.EXE, as the BA matrix solver is not written.

NUM\_BNDS=1 is useful when running very large atomic models, since it requires a factor of 2 less memory than NUM\_BNDS=3, and is faster per iteration. Unfortunately a given model will require (many) more iterations to converge, and overall the time needed to obtain a converged model will generally be longer. Further, the convergence may not be as clearcut as with NUM\_BNDS=3, and Ng acceleration is an important tool for obtaining convergence.
However, for SN models we routinely use NUM\_BNDS=1, with great success.
\item
ND is the number of depth points. In general 60 depth points is adequate for most models. For dirty W-R (and in many cases more than adequate models) 40 depth points is satisfactory. The advantage of using a smaller ND is a smaller execution time, and less memory. In some cases convergence may be faster. For O stars, at least 60 depth points is HIGHLY recommended. Smaller values can lead to numerical instabilities, especially when using low turbulent velocities. Additional depth points (especially around the sonic point) may lead to improved convergence when numerical instabilities are occurring (e.g., persistent fluctuations in the corrections at a single depth). The reason for the need for a higher number of points in O stars is that you need to have sufficient points in both the photosphere and the wind. Due to instabilities in the transfer equation it is not practical to have only a few points in the wind, even though its optical depth may be very small. For SN modeling, we routinely use ND=100.

\item
MAX\_LINES should indicate the maximum number of lines that will be treated in the model. It is specified in MODEL\_SPEC so that the necessary amount of memory can be allocated. If it is too small, an error message will be issued (to OUTGEN), and program execution will terminate.


\item
NCF\_MAX indicates the maximum number of continuum frequencies (which will also contain lines in BLANKETING mode) that will be treated in the model. It is specified in MODEL\_SPEC so that the necessary amount of memory can be allocated. If it is too small, an error message will be issued (to OUTGEN), and program execution will terminate.

\item
NCF is computed by CMFGEN. It is determined by the calculation mode (BLANK, SOB or CMF), and is dependent on VTURB, and the total number of bound-bound transitions in the model.

\item
MAX\_SIM should be large enough to handle the maximum number of lines whose intrinsic (i.e., Doppler) profiles overlap. For a large turbulent velocity, it may be necessary to increase its value. A very large value may slow down the linearization calculation as it partially determines how many lines influence the radiation field at the current frequency. Because of the velocity field, the radiation field at a given frequency can be influenced by lines at higher frequencies, even those outside the Doppler core. For H/He models MAX\_SIM can be less than 20, while for models with many iron-group elements it may need to be $> 100$. An error is output to OUTGEN if the value set in MODEL\_SPEC is too small. If an error occurs, keep increasing MAX\_SIM until the error stops occurring.

\item
The choice of NXzV\_F (the number of levels in the model ion) is entirely up to the user (practical upper limit is set by the available atomic data). At present, only the first NXzV\_F levels are used --- that is,
the levels must be contiguous in energy. This may mean the need for a very large atomic model if you want to include levels above the ionization limit. Reasonable choices are given in the supplied models. Further,
it is relatively easy to test the influence of the number of levels on the model spectrum

\item
NXzV (number of super levels) is also set by the user -- its value must be consistent with the links
specified in the XzV\_F\_TO\_S file. Reasonable choices are given in the supplied models.
It is important to realize that setting NXzV to NXzV\_F is not necessarily the best choice (except for H and He), since it is likely that collisional processes are being underestimated, and these will tend to drive
certain levels (e.g., those belonging to the same LS state) into LTE with respect to each other. Further,
increasing NXzV increases the memory requirements. The choice of NXzV can influence the observed spectrum.

\item
For H, CNO, and Fe, the  number of important levels (NXzV\_IV) should be the same as the number of super levels (NXzV), especially when solving for the temperature structure of the atmosphere. For other elements, NXzV\_IV can often be set to zero. The final solution is NOT affected by the choice of IMPORTANT LEVELS --- only the rate of convergence. The optimal choice is not clear --- it is a tradeoff between memory and convergence. If the temperature is not converging, of if some species give convergence difficulties, adjust the levels considered to be important. If in doubt, set NXzV\_IV=NXzV. The inclusion of additional ``IMPORTANT LEVELS" increases the memory requirement but gives the faster convergence. If you are holding the temperature fixed, the constraints on NXzV\_IV are less stringent.
       
       \end{enumerate}

\vfill
\eject
\mhd{Versions}{version}

\noindent
Only one version of CMFGEN is supplied. The executable is called

\blankline
\indent\indent
         cmfgen\_dev.exe

\blankline\noindent
although this could easily be changed to cmfgen.exe by editing the Makefile in dist/new\_main.
%An very early version, cmfgen.exe is no longer supported, and is not supplied. The newer version should give very similar answers to the older version. 

If you are a new user, check your installation by running some of the supplied models. If you are a regular user of CMFGEN, please run some of your existing models. With each new version, you may need to add additional keywords to the VADAT files. CMFGEN will report any missing keywords in OUTGEN --- when it finds a missing keyword execution stops. While your calculations should yield models in ``agreement'' with older models, the agreement will not necessarily be exact because some code changes cause accuracy changes in the calculations. However, spectra computed using different versions of CMFGEN should agree closely --- if there is a significant difference please report the difference to me. The more of these checks that are done, the less likely the chance that a new version will introduce an undetected bug.


%The primary difference with cmfgen\_dev.exe is that the structure of the large variation matrix (BA) was altered in order to allow very large (\& time consuming) models to be run.

%CMFGEN\_DEV.EXE uses an easily modified form of MODEL\_SPEC. In this file, the number of �IMPORTANT LEVELS� is also specified. 

\vfill
\eject
\mhd{Memory}{memory}

The main memory hog in CMFGEN is the BA matrix used to store the linearized statistical equations. In the worst case scenario, the amount of memory used by BA (and an associated matrix BA\_PAR) is \\
\indent\indent
NT$ \times$NT$ \times$(NUM\_BNDS+1)$ \times$ND \\
floating point words (typically 8 bytes on a workstation). For example, with NT=500, ND=60, and NUM\_BNDS=3 at least 480 Megabytes of memory are required. Large models may require several Gigabytes of memory. When important levels are used, the memory is approximately \\
\indent\indent
NT$ \times$NILV$ \times$(NUM\_BNDS+1)$ \times$ND\\
 floating point words.
        
The BA array is accessed for every continuum frequency and for every line. To avoid excessive paging, it is essential that the BA matrix remain in memory. Thus the BA array limits the size of the atomic models. This memory limit is ultimately more important than execution time in determining the largest model that can be run. A second substantial array is BAION (and BAION\_PAR), which has dimensions NSPEC$ \times$NT$ \times$(NUM\_BNDS+1)$ \times$ND.

In a very early version of CMFGEN, BA was a 4 dimensional matrix with BA(I,J,K,L) giving the variation of the Ith statistical equilibrium equation at depth L with respect to species J at depth K. K only extends to NUM\_BNDS, and is accessed such that 1 corresponds to L-1, 2 to L, and 3 to L+1.

In the current version the format of BA is different: SE(ID)\%BA(I,J,K,L) where ID refers to the particular ionization stage, I refers to the statistical equilibrium equation for the ionization stage, J refers to the variation species but the numbering is dependent on ID. K and L have the same meaning as for the BA matrix. For convenience I will still refer to the variation matrix as the BA matrix in this document. 

\vskip 20pt
\mhd{Execution Time}{timing}

There is no simple formula that gives the execution time as a function of the fundamental variables. For smaller models, the execution time is proportional to
NF$\times$ND$^2$. This scaling rises in the computation of the Eddington factors, and in the linearization. NF is the number of frequencies, and is roughly proportional to the number of lines (at least when not too large).

For models with many bound-bound transitions (e.g., $> 30,000$) other scalings may become important. For example, updating the BA matrix is proportional to NF$\times$NT$\times$ND. Due to excessive paging, models may grind to a halt if the BA array is too large to fit into physical memory.

The system dependent TUNE routine is supplied to determine in which routines or sections of the code most time is spent. On VMS systems the amount of paging is also output. TUNE statements already enclose most of the important sections of the code. It is worth checking the output of TUNE occasionally (as listed in the file TIMING) to verify that the timing of different code sections is not unusual, especially when moving to a new model regime, or a new computer.  NB. Due to excessive overheads, a call to TUNE should not be placed inside an innermost loop. 

\vfill
\eject
\mhd{NG acceleration}{ng_accel}

An option to perform an Ng acceleration is supplied with the code.  Ng acceleration significantly improves convergence, particular when the corrections are small (e.g., $< 10$\%). An  Ng acceleration can also be forced by hand using the stand-alone program DO\_NG\_V2, which reads, and updates, SCRTEMP.

It is recommended that  Ng accelerations only be performed when at least 15 (or more) iterations have been completed. This allows time for changes in the inner regions of the atmosphere to influence the outer regions. If applied too early, very large corrections, often of the wrong sense, may be predicted. I typically perform an  Ng acceleration when the corrections are less than 10\%. In general the number of iterations is more important than the size of the corrections.  Ng accelerations can be applied at each depth independently, over a range of depths, or to all depths at the same time. The later option is probably preferred when using a Diagonal operator, although in many cases I have found that applying the corrections to each depth independently was more efficient (particularly with Tri-diagonal error). 

Even more sophisticated  Ng options could be envisaged. One difficulty is that in the early stages of a model, the nature of the iterations changes with iteration. For example, a FULL LINEARIZATION is usually followed by 1 or 2 $\Lambda$ iterations. Further, the temperature in the outer regions of the stellar wind may be held fixed. This complicated iteration procedure has been adopted to ensure stable convergence.

As noted above,  Ng accelerations can be done outside the code execution by running DO\_NG\_V2. While having similar options to VADAT, it also has greater flexibility. In particular, the  Ng acceleration can be done using every nth iteration, rather than consecutive iterations. Further it is possible to simply average the last two iterations. After a manual  Ng acceleration, CMFGEN should be restarted. For stability reasons, I strongly recommend that you perform a LAMBDA iteration when restarting the model. This can be done by setting both [DO\_LAM\_IT] and [DO\_LAM\_AUTO]   to TRUE in IN\_ITS before restarting the model.


\vskip 20pt
\mhd{Fixing the Linearization matrix (BA)}{fix_ba}

When the corrections to the populations are small the BA and BAION matrices do not need to be recomputed. They are thus saved, and read in when needed.
This trick significantly reduces the computation time per iteration. The only drawback is that for large NT, a significant amount of disk space is needed. 
 
I used to typically adopt 10\% but have made changes to the code to allow more flexibility. The BA can be held fixed for several iterations, even when the corrections are very large. This can reduce the computation effort significantly. The optimal strategy is probably a function of the stellar parameters, and may also depend on the model initialization. A lack of time, and continual changes to the models, has not allowed an optimal strategy, if it exists, to be determined. A similar statement also applies to  Ng acceleration.

Remember to delete the BAMAT and BAIONMAT files when a model is finalized (use the shell command clean (clean.sh) defined in aliases\_for\_cmfgen.sh. The *SCRATCH* files should also be deleted. These *SCRATCH* files allow more efficient solution of the simultaneous (rate) equations contained in the BA matrix, particularly when NT is large.  On some systems and for some model regimes, it may be more efficient not to write the *SCRATCH* files --- this is set by a parameter in VADAT.


\vfill
\eject
\mhd{Convergence}{convergence}

CMFGEN defines an iteration as any cycle loop that changes the level populations. Thus the sequence:

%\blankhalf
\begin{mynarrow}{0.5in}
%\noindent
$\Lambda$-iteration, \\
full linearization, \\
full linearization, \\
hydrostatic density correction, \\
$\Lambda$-iteration, \\
full linearization, \\
full linearization, \\
 Ng Acceleration, \\
$\Lambda$-iteration, \\
    full linearization, 
\end{mynarrow}
\blankhalf
\noindent
constitutes  10 iterations. Other atmospheric codes might count this as only 5 iterations. The total number of iterations required to obtain a converged model depends on NUM\_BNDS and, not surprisingly, on the parameters of the model. For the following discussion we assume NUM\_BNDS=3.

The minimum number of iterations required for convergence is approximately 20. However, this only occurs when changing abundances of impurity species etc. A similar, or smaller number, may also be required if the number of super-levels associated with a given species is changed.

Generally 30 or more iterations are required, when using a tridiagonal operator, to obtain convergence of a NEW model (i.e., a model with new stellar parameters) to 0.1\%. The convergence refers to the MAXIMUM correction obtained on the last iteration. The actual convergence achieved depends on depth, and the level under consideration. Usually the corrections to a given population form a geometric series: Thus if

\blankline
\centerline{$r = $(previous correction)/(current correction)$ = b/a$}

\blankhalf
an estimate of the true error is $a/r + a/r^2 + a/r^3 \dots$ which is approximately $a/(r-1)$. In general, r is only slightly larger than unity. In practice, this error is too pessimistic if calculated using the maximum error returned in OUTGEN. Usually the maximum error is for an unimportant level at an unimportant depth. When successive corrections are less than 0.1\%, the observed spectrum usually shows negligible changes between successive iterations (and with a truly converged model). When using the tridiagonal operator, it is relatively easy (but not generally warranted) to improve the model convergence to less than 0.0001\%.

For some models 60, or more, iterations may be required. The need for this many iterations can occur in several ways:

\begin{enumerate}

\item
The atmosphere is intrinsically unstable. Small changes in the stellar parameters result in large changes in the populations at some depths. In this case little can be done to improve convergence. These instabilities occur, for example, in some LBV-like models. 

\item
An ionization front (discontinuity) has formed in the wind. CMFGEN is trying to eat its way through. Fudging the populations by hand can often assist in obtaining convergence, as can adding additional grid points in the neighborhood of the discontinuity.

\item
A few levels at one depth (or a few depths) have very bad population estimates. Successive iterations do not appear to be changing the estimates. When the populations are examined, they are found to be anomalous with respect to those at surrounding depths. Fudging the populations by hand, or with DISPGEN, will assist in obtaining convergence. Including additional depth points at the region of instability may also improve convergence.

\item
The temperature structure is not converging. If you are using cmfgen\_dev.exe with levels omitted from the full linearization  (i.e., NXzV\_IV $<$ NXzV)  the inclusion of additional levels (usually for the most abundant species) will help the convergence. Alternatively, you may need to increase the number of super levels,

\item
You are using a diagonal operator, which has an inherently slower asymptotic convergence than the tridiagonal operator.

\item
A few populations at a few depths are oscillating and show larger corrections than other populations
(as seen in CORRECTION\_SUM). You can use ADJUST\_CORRECTIONS and/or DO\_NG\_V2.EXE
(see page \pageref{sec_troub_shoot}).
\end{enumerate}

\vfil
\eject
\mhd{Features and Bugs}{bugs}

CMFGEN is a large complicated program. As such it will contain bugs. The VADAT file contains over 100 options --- it is IMPOSSIBLE to test each option every time the code is updated. Obsolete options, or those not used very much, are more likely to cause problems. PLEASE CHECK YOUR CALCULATIONS FOR UNEXPECTED RESULTS.  Many (too many!) diagnostic files are created to ensure that your results are reasonable.

Please report all bugs, no matter how trivial, immediately. We will inform users of important bugs --- eventually there will be a WEB page where these bugs can be reported and checked.

Extensive comparisons have been made, and are being made, with TLUSTY (the plane parallel code created by Hubeny and Lanz). In general there is excellent agreement. 

%We are working on eliminating the SMALL discrepancies.


\blankhalf
{\bf Known features:}

   $\bullet$
    When using the CMF option, NP must be ND+NC-2, otherwise it should be ND+NC.

    $\bullet$ Clumping has only been fully implemented and tested in BLANKETING mode. 

    $\bullet$ When T is adjusted in an input file, the departure coefficients should also be adjusted. This is not trivial. It would probably be better to read in the populations in the outer layers, but we are unable to do this at present. An option in DISPGEN can be used to adjust
 T at a range of depths, and recompute new departure coefficients assuming the populations are held fixed.

$\bullet$ The CMF\_FLUX calculation may crash (although this is now very rare using the
 [FRM\_OPT]=INT option) with a floating point overflow exception. If the error occurs try changing LAMBDA\_ITERATION in CMF\_FLUX\_SUB\_V5 to .FALSE., and recompile/link CMF\_FLUX. As a consequence of the change, CMF\_FOR\_SOL\_V2 will not be called.


\blankline
\centerline{\bf Atomic data bugs}
\blankhalf

$\bullet$
One of the lower levels in the N\,{\sc i}  data has the wrong statistical weight. I will fix this as soon as I get a chance. It is not a simple change, since the f values to the state may change (depending on where they come from etc).

$\bullet$
In some atomic modes, a few levels have the wrong energy order (the levels have virtually identical
energies). This will make no difference, provided you do not end the model atom with one of these levels.
 
$\bullet$
Some line wavelengths are inaccurate. Due to the coupling with the atomic levels, and adopted
atomic data, these are not necessarily easily fixed. When computing spectra using CMF\_FLUX, revised wavelengths
for important lines can be specified in the file \$cmfdist/misc/REVISED\_LAMBDAS. 

$\bullet$
The Ar\,{\sc v} data set  \$ATOMIC/ARG/V/1dec99/fin\_osc should not be used. Use instead \$ATOMIC/ARG/V/1dec99/arvosc\_rev.dat  (as stated in the README file). If the Ar\,{\sc v} line at 1345\AA\ is too strong, make sure you are using the correct data set. As a precaution, you might like to rename the fin\_osc file
in the atomic data set so you don't accidently use it -- I ship it to maintain compatiblility for testing etc.

\vfill
\eject
\mhd{Installation}{install}

For installation purposes it is recommended that you adopt the same directory structure used by the author.
This will automatically be obtained when you untar the distribution package. After untaring, 
it is suggested that you move CMFGEN into a distribution directory of the form \\
\indent
\indent
    {\color{blue} cmfgen/date/} \\
\noindent
(or some similar variant) where ``date" refers to the current distribution date. In the following this distribution directory will be referred to as \$cmfdist.
All atomic data should be stored in directories of the form

\indent
\indent
    {\color{blue} something/ATOMIC/DUM/XzV/date}

\noindent
where date is of the form ddmmyy. This storage method will facilitate updates, bookeeping, and transfer of files to collaborators. Several different data sets already exist for some species.  It is a nightmare. NB. For the atomic data directories, we tend to use I, II, III, IV etc. instead of the usual CMFGEN conventions. 

You should also place the following (or equivalent) command in the shell source startup file (e.g., .tcshrc for the tc-shell).

\indent
\indent
    {\color{blue} source \$cmfdist/com/aliases\_for\_cmfgen.sh.}

\noindent
where \$cmfdist must be expanded to its full path. This defines the source directory (\$cmfdist or \$CMFDIST) and atomic data directory (\$atomic or \$ATOMIC) for the shell. The csh script also defines command names for common programs (e.g., cmfgen for \$cmfdist/exe/{\-}cmfgen\_dev.exe, plt\_spec, dispgen etc), and well as other useful commands (page \pageref{sec_scripts}). Users should check that the cmfgen
scripts do not conflict with their own scripts.

\blankline
\blankline
\shd{External routines/packages required by CMFGEN}{ext_rout}
\vspace{-10pt}
\begin{file_list}{PG}
\item
[PGPLOT]    Free package available off the WEB (see http://www.astro.caltech.edu/$\sim$tjp/pgplot/). Used by many of my routines for line drawings. I have a MAC version on my website. 
 \item
[BLAS]    Basic procedures for performing matrix manipulations. Should be available with the F90 compiler. If not, generic FORTRAN routines (obtained off the web)  are supplied, but these will be less efficient than the system dependent BLAS routines. The compiled generic routines are placed  in lib/libmy\_blas.a The my was inserted to avoid using these libraries inadvertently. 
\item
[LAPACK    FORTRAN numerical library] Often comes with F90 compiler. If not, routines available on the WEB. Several are supplied.
The compiled routines are placed in lib/libmy\_lpack.a. 
\end{file_list}


\vfill
\eject
\shd{Makefile}{make}

The distribution comes with Makefiles which will allow compilation of CMFGEN on your UNIX system.  I have been successfully using the standard make on Mac and Linux systems.
To use these Makefiles do the following:

Edit Makefile\_definitions in the dist/ directory. Change it to ensure the following:

\begin{enumerate}
\item
 The distribution directory is defined.
\item
 It uses your f90 compiler.
\item
 The correct f90 flags have been set.
\item
The library locations for pgplot, blas, lapack, etc  are correct.
\end{enumerate}

\noindent
Then \\

\indent\indent
make clean    

\blankline
\noindent which removes all object files, libraries, executables from the distribution directories

\noindent
Then \\
\blankline
\indent\indent make �i  $>$\& HOPE    

\blankline
\noindent
does the compilation. All error messages will be output to the file HOPE. -i indicates to ignore any warning messages which may be necessary on some systems. Try to avoid the �i option, if possible.  If the compilation/linking works, the directory \$cmfdist/exe will contain the following files (plus many others): 

{\narrower
\begin{table}[h]
\begin{tabular}{lll}
   cmf\_flux.exe     \hspace{10pt}   &  do\_ng.exe \\
   cmfgen.exe       & append\_dc.exe \\
   dispgen.exe       & rewrite\_dc.exe \\
   plt\_spec.exe    & plt\_jh.exe \\
   plt\_jh.exe         &    wr\_f\_to\_s.exe
\end{tabular}
\end{table}
}

\noindent
If there is a failure, individual Makefiles in the sub-directories can be executed simply by typing

make

\noindent
in the relevant sub-directory. 

\blankline
NB: All file dependencies should be specified in the Makefile. However, I have been known to
miss some of them. If strange errors occur, after updating a subroutine (particularly modules),
the user should recompile and link all routines.

\vskip 20pt
\shd{Makefile System Dependencies}{make_depend}

Redhat 6.1    Supplied Makefiles work moderately well, with only a few minor (and insignificant) problems.
                     Over the last few years I have only used them in LINUX systems, and on an Intel MAC.
                     
%DEC OSF Alpha    Makefiles supplied with package may cause problems unless GNU make is used. Those in linux\_make.tar (which contains only Makefiles) will work with the OSF make, although not perfectly. These have minor modifications from those supplied with the general distribution, and will allow the full installation to proceed. Some of the Makes seem to have features which don�t work properly (e.g., library dependencies). These Makefiles can be installed using the command
%    tar �xvf linux\_make.tar
%in the installation directory.

If you improve these files, please make a new linux\_make.tar so I can distribute it. To make this file do the following (in \$cmfdist):

     set MAKELIST=`find . �name Makefile\*`
     tar cvf linux\_make.tar \$MAKELIST


\vskip 20pt
\shd{Possible Compilation Problems}{compile_prob}

\begin{enumerate}

\item
New compiler releases (especially major updates) often contain bugs which are fixed in subsequent revisions. Thus I highly recommend not deleting your current FORTRAN compiler (especially if it works)  when a new version comes in.

\item
Most of the programs and subroutines are written in f90 fixed format style. Most compilers can figure this out for themselves, but some need help. It may depend on the extension.

\item
Some compilers may insist on certain file extensions. A HP compiler will not compile programs of the form *.for. The usual convention on most compilers is that compilers assume *.f to contain formatted fortran files, while *.f90 files are unformatted.

\item
Several programs (e.g., CMFGEN, CMF\_FLUX, DISPGEN) use modules for data storage. In these modules, data types are defined. For example, mod\_cmfgen.f, defines the data type MODEL\_ATOM\_DATA. In f90, all arrays in these data types must be declared as a POINTER. This is VERY inefficient. The new standard now allows arrays to be declared ALLOCATABLE. Some of the new modules have been converted to use ALLOCATABLE arrays, but these may not compile with older compilers. If this is the case convert to the  POINTER approach. Note that the behavior and structure of CMFGEN and its routines is independent of which declaration is used. 

\item
For historical reasons most routines use the notation {\bf REAL*8}. Most compilers have switches to change/override this. On alphas, with the PGF compiler and with the Intel compiler, these definitions are fine. If not, it is relatively easy to use the SED editor to change all occurrences of REAL*8 to REAL. Hopefully, all routines just use INTEGER for integer declarations. NB: It is important that INTEGER and INTEGER*4 refer to the same sized variable.

The supplied BLAS routines use DOUBLE\_PRECISION notation.

All programs use REAL*8 with the exception of routines in the pgplt directory (discussed below).
All programs can be compiled to use the default INTEGER size (but at least 4 bytes).

The only routines that uses REAL*4 are routines related to PGPLOT since pgplot uses REAL*4.  -- The data storage module for the plotting routines, MOD\_CURVE\_DATA, also uses REAL*4 to save space. This is done simply to conserve memory. Data to GRAMON\_PGPLOT is generally passed as REAL*8 via DP\_CURVE. A few routines don't use the REAL*8 notation, but this should be taken care of in the Makefile.

\item
Some routines will give compilation warnings. They should have no effect. Typical warning messages are: \\
\hbox{}\hspace{15pt}     Passed variables not used in subroutine. \\
\hbox{}\hspace{15pt}     Variable defined but not used.\\ 
\hbox{}\hspace{15pt}     Variable set but not used. \\
While some of these could be fixed it might necessitate a change in the calling routine etc. Some are given for FORTRAN parameters which I nearly always define, but which may not be used in the particular routine under consideration.

\item
On some compilers \verb+\+ is interpreted as a control character. For reasons known to no one I use this character in the atomic data files as a delimiter. Thus you may need a switch on the compiler to ensure that \ it is not interpreted as a control character.

\item
There may be a few compiler dependent features. For example, some compilers insist on writing 1X in format statements (instead of just X). I fixed as many as I know about but \dots


\item
On some compilers, \$cmfdist/disp/maingen.f can take a long time to compile. If this occurs, change \$cmfdist/disp/Makefile so that maingen.f compiles with a lower level of optimization. It will not noticeably effect 
performance of this interactive program.

 \item
Under DEC OSF on an ALPHA, record alignment problems may be encountered. I have no idea what causes these, nor a general prescription for their removal. The last time this problem occurred it was solved (temporarily?) by moving the declaration \\
\hbox{}\hspace{15pt}                          TYPE(MODEL\_ATOM)   ATM() \\
to the end of the module declaration file. None of the compiler options appeared to make a difference.

\item
Routines with known system dependencies are placed in \$cmfdist/unix, and may need to be edited. The main routines that may need editing  are {\bf tune.f}, {\bf date\_time.f}, and {\bf set\_line\_buffering.f}. The default routines work under pgf fortran with MAC and LINUX systems. With Intel compilers {\bf set\_line\_buffering.f} should be changed to a dummy routine --- it is used to stop buffering
of ASCII files with pgf. No buffering is the default with Intel systems.

\item
The Intel Compiler can have issues, and these change from release to release. I used \\

\indent
\verb+ FG = -cm -extend_source -w95 -w90 -nbs -O3 -axTW -I$(MOD_DIR)+

when compiling. In the latest release (at least on a Mac Pro) you may need to change the Makefiles to force some modules to be linked. The latest INTEL compiler would not link pure modules that were in Libraries. It did not run successfully with OMP.

\end{enumerate}

\vskip 20pt
\shd{Run time problems}{compile_prob}

\begin{enumerate}

\item
With some Intel compilers, some character statements are incorrectly handled. When this error occurs CMFGEN stops almost immediately, complaining about missing data files. The two most likely occurrences arise in {\color{blue} \$cmfdist/newsubs/genosc\_v8.f} and {\color{blue} \$cmfdist/newsubs/rdphot\_gen\_v2.f}. To fix the error do the following:  In the first routine change FILNAME to have an explicit length ``LEN=20'', while in the second
change DESC to have an explicit length ``LEN=12'', and recompile.

\item
You have compiled the code using INTEL with parallelization,  and it crashes in the \$cmfdist/plane/fg\_j\_cmf\_v12.f routine. This seems to a bug with the
INTEL compiler. Simplest fix is to comment out the OMP parallelization statements in fg\_j\_cmf\_v12.f. 

\item
You have compiled the code using GFORTRAN.


\end{enumerate}



\vskip 20pt
\shd{Parallelization}{parallel}

Some of the routines associated with cmfgen\_dev.exe, cmf\_flux\_v5.exe, and main\_lte.exe have been parallelized using OMP commands. To implement the commands in the executable, compile/link options need to be specified. For the pgf95 compiler, for example, I define F90 as  pgf95 -mp. These commands  significantly improve the execution of the code on systems with multiple processors (e.g., factor of 2 to 3 faster with 4 processors). Using more than  8 processors does not lead to significant gains.

These parallelized routines work under pgf version pgi/osx86-64/10.8/bin/pgf95 on a MAC, and pgf version
linux86-64/10.5/bin/pgf95 on LINUX AMD systems. Later versions should also work --- I simply have not updated my compilers. I have had trouble with some routines (e.g., \$cmfdist/plane/fg\_j\_cmf\_v12.f) with the Intel compiler. OMP instructions can be removed from individual routines by adding an additional ! to the line containing the OMP command.

I recommend users first installing CMFGEN without parallelization. If that works, then install
a parallelized version. Compilers seem to have much more difficulty (i.e., they have more bugs) with
parallelization.


\vfill
\eject
\mhd{Directory Structure}{dir_struc}

It is STRONGLY suggested that the same directory structure be retained when copying the code. This will make upgrades easier. On UNIX systems, the FORTRAN files are generally lower case, and are given the extension �.f� rather than �.FOR�. The �*.INC� files MUST be upper case. The current version is stored in \$cmfdist/� 

\begin{dir_list}{new}

\item[main]     No longer used.

\item[new\_main]   
 Contains the main program and Fortran �include� files for main program. Most {\color{Red} .INC} files have been converted to subroutines.
\blankhalf
\begin{tabular}{ll}
 \hbox{} \hspace{0.2in} cmfgen.f       \hspace{0.5in}\hbox{}      & Main Calling routine \\
 \hbox{} \hspace{0.2in} cmfgen\_sub.f     & Workhorse  \\
 \hbox{} \hspace{0.2in} GET\_J\_CHI.INC  
\end{tabular} 

%    There also several sub-directories with routine specific to this version.

\item[new\_main/mod\_subs]    Contains module definitions for use with cmfgen. At present two versions of mod\_cmfgen.f are supplied. The default uses the new fortran structure which allows allocatable arrays to be used in structures. Some compiler versions (e.g., old intel) did not yet allow this option, so use mod\_cmfgen.f\_intel (although it has not be utilized for some time). No other changes to the code are required.

\item[new\_main/subs] New subroutines. Kept in this location for simplicity. 
\begin{dir_list}{hope}
\item[new\_main/subs/auto] Routines to handle autoionization.
\item[new\_main/subs/chg] Routines handling charge exchange.
\item[new\_main/subs/non\_therm] Routines handling non-thermal ionization and excitation.
\item[new\_main/subs/two] Routines handling two-photon processes for H and He. 
\end{dir_list}

\item[com]   Contains simple scripts to facilitate running models etc.

\item[disp] Contains DISPGEN display package.

\begin{dir_list}{hope}
\item[disp/subs] Auxiliary subroutines for DISPGEN.
\end{dir_list}

\item[misc] Collection of routines useful for a variety of purposes including generating atomic data, operating on departure coefficient files, and generating F\_TO\_S link files.

\item[newsubs] Some new routines that replace routines in subs/. Primarily to do with the new version which uses dynamic memory allocation and records for the atomic populations. Kept here for convenience. The names are distinct.

\item[obs] Routines CMF\_FLUX and OBS\_FRAME used for computing the observed spectrum. CMF\_FLUX is the main routine: OBS\_FRAME is primarily used for testing (it has not been updated/tested for ages), and requires J, $\eta$, and $\chi$ output from CMF\_FLUX.

\item[pgplt]  Plotting packages for DISPGEN and PLT\_SPEC. These routines utilize PGPLOT, which is freely available and runs on many different platforms.

\item[plane] Originally contained subroutines for undertaking radiative transfer in plane-parallel geometry. It now also contains relativistic and time dependent radiative transfer routines, and updated versions of my main
transfer routines.


\item[spec\_plt] Contains routines for displaying theoretical and model spectra (plt\_spec), moments of the radiation field (plt\_jh; plt\_jh\_cur), intensity data (plt\_ip), flux origin (plt\_dfr), and populations as a function of iteration (plt\_scr).

\begin{dir_list}{hope}
\item[spec\_plt/subs] Auxiliary subroutines for PLT\_SPEC.
\end{dir_list}


\item[subs]    Main directory with most of the subroutines required by CMFGEN.

\item[tools]    Directory containing useful tools such as GEN\_IN and USR\_OPTION 

\item[unix]     A few routines (e.g., tune.f) that may be system dependent.
\end{dir_list}

\vfill
\eject
\mhd{Atomic Data}{atomic_data}

\noindent
All atomic data is stored in directories of the form

\hbox{}\indent\indent something/ATOMIC/DUM/XzV/date

\blankhalf \noindent
where date is of the form ddmmyy. Thus each species and each ion has its own directory associated with it. This storage method will facilitate updates, bookkeeping, and transfer of files to collaborators. Several different data sets already exist for some species. Except for H/He, use the latest version. See the README file associated with each ionization stage for possible comments.  In some cases the data is from an alternate source, and it is not obvious which is best. The compilation of the atomic data is a nightmare. NB. For the atomic data directories, we tend to use, I, II, III, IV etc instead of the usual CMFGEN conventions.

With care, the supplied atomic data is sufficiently accurate for computing the atmospheric structure, and for computing abundances. However, it is essential that users take time to understand the limitations of each atomic data set, particularly when determining abundances. In some cases, levels have been combined into a single level, and this may affect both the strength and location of some lines. Fortunately, the atomic structures have generally been chosen so that this is not a major issue. In other cases, line wavelengths are wrong because the observed energy levels and wavelengths are not known. For example, CMFGEN predicts the presence of Fe\,{\sc iv} lines in the optical region of O stars. Unfortunately, NONE of the wavelengths for these transitions are accurately known. This can cause �spurious� features to appear in the spectrum. More importantly, it is possible that some of the Fe\,{\sc iv} lines could lie on top of other lines which are used for abundance diagnostics in the observed spectrum.

For each ionization stage there are 4 to 6 principal data files that are required:

\begin{file_list}{XzV\_F\_OSCDAT}
\item[XzV\_F\_OSCDAT]    File containing energy levels and oscillator strengths. In the level list, only the columns containing the level name, statistical weight, oscillator strength and the ID number are important.
A $-$ve ID indicates that the energy of the level has not been determined experimentally (and hence
can be regarded as unknown) -- in most cases, but not all, this is reliable. In the line section, only the columns containing the f value, and the i-j column are important. For speed, the i-j column is used to assign the transitions to the internal array, not the level names. The format of this file is important. Do NOT edit the file by hand, unless you know what you are doing. To include additional transitions (e.g., intercombination lines) it is best to use rewrite\_osc\_v2.exe. 

\item[XzV\_F\_TO\_S]    File containing information on which levels are going to be grouped together as a super-level. Use WR\_F\_TO\_S to create. File can be edited by hand, and cleaned using WR\_F\_TO\_S.

\item[XzV\_COL\_DATA]    Table of collisional data.


\interlinepenalty=10000
\item[PHOTXzV\_A]    Table of photoionization cross-sections for each level. Look at an existing file to see format. Tabulated data, and specific analytical expressions, are allowed. ``{\bf A}" indicates that the final state is the ground state n (often a super level linking levels in the lowest term) of the next ionization stage.

\item[PHOTXzV\_B]    As for PHOTXzV\_A, but for photoionizations to an excited state. Not all species have such data. In many cases it is included directly into the ``A" cross-sections. 

\item[DIEXzV]    Contains a list of low temperature dielectronic recombination (LTDR) lines from the work of Nussbaumer and Storey. Be careful not to include the lines twice. Some of these lines are automatically included in the Opacity project photoionization cross-sections. File is not necessary if these transitions are not going to be included in the model.

\item[AUTO\_XzV\_DATA] Contains list of autoionization level probabilities for levels above the ionization continuum.

\end{file_list}

\blankline
\shd{Atomic data programs}{atomic_data_progs}

A collection of programs to read various atomic data formats, and to write them out in a format suitable for CMFGEN is available.  These programs are not well documented, and are somewhat messy. They can be made available on request.

\vfill
\eject
\mhd{Useful Scripts}{scripts}

Several useful scripts can be found in the \$cmfdist.com directory. To use these scripts:


\begin{enumerate}
\item
Edit the file /com/aliases\_for\_cmfgen.sh. Replace all occurrences of ~hillier (or jdh) by the directory containing CMFGEN (e.g., /CMFGEN/30JUN00). In some of the scripts you may need to change the source locations of the csh, tcsh and perl.

\item
Put the command 
\blankhalf
\hspace{1 in} source /dist/com/aliases\_for\_cmfgen.sh
\blankhalf\noindent
in your .tcshrc file. You will need to source the file --- afterwards this
will be done automatically when  you login. The aliases\_for\_cmfgen.sh defines the following commands:
\end{enumerate}

\begin{scriptlist}{rm\_all\_links}

\item[clean]
Cleans model directory after model completion. Removes unwanted files such as EDDFACTOR, BAMAT, BAION etc. Only use the command after you are satisfied with the convergence of the model. You may want to remove the deletion of the EDDFACTOR and EDDFACTOR\_INFO files from the command, as this file is sometimes useful when starting other models. To recover a deleted EDDFACTOR file you will need to do a single iteration.

NB: I tend not to remove the SCRTEMP and POINT files, since these make it easy to run additional
iterations in the same directory. However, the commands \$cmfdist/exe/rewrite\_scr.exe and \$cmfdist/com/mvscr.sh can be used to generate a smaller version of SCRTEMP containing only the last N (default is 2) iterations.

\item[rmlinks]    Removes soft links. This option only removes links in the current directory, and it shows what it is doing.
\item[rm\_all\_links]    Removes soft links in the current directory, and in all lower directories.
\item[rmin]    Removes all files of the form *\_IN (no prompt)
\item[rmrrr] Removes all files of the form *PRRR    (no prompt)
\item[dfort] Removes all files of the form fort.*    (no prompt)
\item[dlog] Removes all files of the form *.log    (no prompt)
\item[dscratch] Removes all files of the form *SCRATCH* (no prompt)
\item[dsve] Removes all files of the form *.sve    (no prompt)
%\item[ astxt]
%    Creates soft links to the help files required by DISPGEN, PLT\_SPEC, and WR\_F\_TO\_S.

\item[out2in]    Renames all files of the form *OUT to *\_IN
\item[inc2inc]    Renames files of the form abcde.inc to ABCDE.INC
\item[fortof]    Renames files of the form abcde.for to abcde.f

\item[ full\_diff]    Compares ALL *.f, *.sh, *.INC and Makefile files in one directory with those in another. If only one directory is supplied as a parameter, the current directory is assumed to be the primary directory. To compare the current distribution with the previous distribution do the following in \$cmfdist: \\

\indent
{\color{Brown} \$cmfdist/com/full\_diff.sh \$old\_cmfdist $>$  Error }

or

\indent
{\color{Brown} \$cmfdist/com/full\_diff.sh \$old\_cmfdist $>$\&  Error}

In the first case error messages related to files for which there are no counterparts in the old distribution directory is written to the screen, while in the second case they are written to ``Error''. Error will also congaing other information about the directories examined etc. The file {\color{blue} Diff\_sum} will list all examined files, and show differences when they occur.


\item[ for\_diff]    Compares ALL *.f files in one directory with those in another. If only one directory is supplied as a parameter, the current directory is assumed to be the primary directory.

\item[ INC\_diff]    Compares ALL *.INC files in one directory with those in another. If only one directory is supplied as a parameter, the current directory is assumed to be the primary directory.
\end{scriptlist}

If you have alternative and better methods let me know.
 

\vfill
\eject
\mhd{Model computation}{mod_comp}

\noindent
There are 5 primary control files:
\begin{plt_list}{HYDRO\_DEFAULTS~}
\item[batch.sh]    Controls model execution, and creates soft links to the atomic data.
\item[HYDRO\_DEFAULTS]    Only needed when a hydrostatic iteration is to be done.
\item[IN\_ITS]    Contains the maximum number of iterations to be attempted, and can also be used to force $\Lambda$--iterations.
\item[MODEL\_SPEC]   Specifies atomic models, number of depth points etc.
\item[VADAT]    Principal control which is used to describe the model, and assumptions used for its computation.

\end{plt_list}
These are described more fully elsewhere. Many other data files are also needed.

To generate a new model it is easiest to use the output from a previously converged model. This model should have parameters close to the new model (e.g., a factor of 2 in mass-loss rate). Exactly how close depends on the parameter regime. Some models will converge for much larger parameter changes. To initiate a model,  all the required input data files (discussed below) need to be copied to the directory of the new model.
The simplest method of doing this is as follows:

\blankhalf
 \hspace{1in}      {\color{blue} cpmod r1 r2}
\blankhalf

\noindent
In the above {\bf r1} is the old model directory and {\bf r2} is the new model directory. If {\bf r2} is the same as {\bf r1}, you will be requested to confirm the command.  The above command does the following:

\blankline
\begin{mynarrow}{0.25in}
\begin{tabular}{ll}
cp r1/*OUT   &    r2/ \\
cp r1/VADAT   &     r2/ \\
cp r1/batch.sh   & r2/ \\
cp r1/MODEL\_SPEC &   r2/ \\
cp r1/IN\_ITS   & r2/  \\
cp r1/GAMMAS &   r2/GAMMAS\_IN 
\end{tabular}
\end{mynarrow}

\blankline
\noindent
It also renames the *OUT files to *\_IN by issuing the  command
\blankhalf
 \hspace{1in}      out2in
\blankhalf
If they exist, the following files are also copied. 

\blankline
\begin{mynarrow}{0.25in}
\begin{tabular}{ll}
cp r1/RVSIG\_COL   & r2/ \\
cp r1/HYDRO\_DEFAULTS    & r2/ \\
cp r1/ROSSELAND\_LTE\_TAB   &  r2/ \\
\end{tabular}
\end{mynarrow}
       
\blankline
If the directory {\bf r1} contains JH\_AT\_OLD\_TIME, the model is assumed to be for a supernovae and
the following additional files (JH\_AT\_OLD\_TIME, JH\_AT\_OLD\_TIME\_INFO, NUCLEAR\_DECAY\_DATA , OLD\_MODEL\_DATA, \& SN\_HYDRO\_DATA) are also copied. {\color{red} The cpmod command should not be used to generate input data for a new SN model in a SN time sequence}({\color{blue} use  drad\_cpmod instead}) --- it is simply used to allow a revised SN model to be run at the same time step as the existing model.

\blankline
In general, running a model requires the following steps to be performed. NB: It is not necessary to perform all of these steps for every model. For example, suppose you are computing a small grid of O star models, and you already have one model computed. Since MODEL\_SPEC and batch.sh have already been edited, you may skip steps (1) \& (2). If the abundance are also fixed you can skip step  (7)-- it is also probably reasonable to skip this step if you are not changing the abundances of the most important species.

\begin{enumerate}
\item
Edit the MODEL\_SPEC file so that the correct species and ionization stages
are specified. Set the number of depth points, core rays, and impact-rays for the model. When running sequences of models, such changes will be minimal. Make sure that you have a XzV\_IN file for all ionization stages if running a new model.

\blankline
NB: For continuing an existing model, the populations will be obtained from SCRTEMP --- you do not need to update the XzV\_IN files.

\item
Edit the batch.sh file so it points to the appropriate directories. On a Unix system soft-links ({\bf ln -sf}) are used to point to atomic data files.

\item
Edit the VADAT file to set the parameters of the NEW model. If it is truly a new model, with new stellar parameters, set [LIN\_INT] to F. If you are testing the influence of the number of grid points, or the effects of different SL assignments, etc., set [LIN\_INT] to T (see description of VADAT options). NB: If Ar\,{\sc  iii} has been deleted from the MODEL\_SPEC file, it is not necessary to delete all occurrences of Ar\,{\sc iii} (e.g., FIX\_ARIII) from the VADAT file.

\item
Edit the IN\_ITS file to give the required number of iterations, typically 60. A model will terminate when it has converged (with the convergence requirement specified in VADAT) or after the specified number of iterations.

\item 
If running a hydrostatic model, you may need to update RVSIG\_COL using \$cmfdist/{\-}exe/{\-}rev\_rvsig.exe. An update is needed if you change the mass-loss rate, if you change the core radius of the star, if you change the velocity law, or if you change the number of depth points. A change is not needed for a change in abundances, a change in $\log g$, a change in model atoms, or a change in which species that are included. A RVSIG\_COL file can be created from scratch using \$cmfdist/exe/tlusty\_vel.exe, \$cmfdist/exe/wind\_hyd.exe, or from an old model using DISPGEN.

\item
If running a hydrostatic model, edit HYDRO\_DEFAULTS to indicate the required number of iterations --- this number decreases as the model runs. It may be altered when the program is running. 

\item
If the abundances of important species have changed, if you have changed the turbulent velocity, or have significantly changed the model atoms and species, you may need to update ROSSELAND\_LTE\_TAB.
 
\item
Finally, type {\color{Red} ``batch.sh \&''} to start the model. If the file crashes, check OUTGEN and batch.log. If the model does not converge in the number of iterations allocated (in IN\_ITS), but convergence is likely, simply restart the job with {\color{Red} ``batch.sh \&''}. The model will use the results from the last successful iteration (stored in SCRTEMP) to restart.

\end{enumerate} 



\newcounter{casecount}
\newenvironment{case_list}[1]%
 {\begin{list}{}{ 
 \usecounter{casecount}
 \settowidth{\labelwidth}{0in} %\textbf{#1}}
   \setlength{\leftmargin}{\labelwidth}
   \setlength{\listparindent}{0pt}
   \addtolength{\leftmargin}{\labelsep}
%  \renewcommand{\makelabel}[1]{\centerline{\color{blue} \textbf{Case \arabic{casecount}: [1]}}}}}%
  \renewcommand{\makelabel}[1]{\color{blue} \textbf{\hfill Case \arabic{casecount}: #1}}}}%
  {\end{list}}

 \newcounter{methcount}
\newenvironment{meth_list}[1]%
 {\begin{list}{}{ 
 \usecounter{methcount}
 \settowidth{\labelwidth}{0in} %\textbf{#1}}
   \setlength{\leftmargin}{\labelwidth}
   \setlength{\listparindent}{0pt}
   %\addtolength{\leftmargin}{\labelsep}
   \renewcommand{\makelabel}[1]{\color{OliveGreen} \textbf{\hfill Method \Alph{methcount}: }}}}%
  {\end{list}}
  
\noindent
Some specialized cases for starting a new model are discussed below. These examples assume that the command \\
\hbox{}\hspace{15 pt}{\bf cpmod r1 r2}  \\
has been issued, and that you reside in the {\bf r2} directory. It also assumes \$cmfdist and \$atomic are correctly defined via
aliases\_for\_cmfgen.sh.
 
\usecounter{casecount}

\case{Running a supplied model}
Use the cpmod command to copy the required model files into the new model directory, {\bf r1}. Edit IN\_ITS to set the number of iterations, and set 
[LIN\_INT]=T in the VADAT file. Start the model by issuing batch.sh \&. If the model crashes (or stops quickly), see if OUTGEN has been created.
If it has, check the bottom of OUTGEN to see any error messages. The most likely error is that you will need to add a new keyword (or keywords) to VADAT. If it has not been created, see the compilation section for possible causes.az

\case{Change in number of depth points.}
You wish to run a model which only differs from {\bf r1} in the number of depth points. Use the cpmod command to copy the required files from {\bf r1} to {\bf r2}. Before starting the model you need to edit two lines in MODEL\_SPEC (those containing ND and NP). In the VADAT file, set [LIN\_INT]=T. If using the RVSIG\_COL file, you will need to run REV\_RVSIG to update the number of grid points. You MUST also set the number of hydrostatic iterations in HYDRO\_DEFAULTS if you wish to update the hydrostatic structure.

\case{Changing abundances}
You wish to run a model which only differs from {\bf r1} in the abundances. Use the cpmod command to copy the required files from {\bf r1} to {\bf r2}.  In the VADAT file, set  the new abundances, [LIN\_INT]=F and [IT\_ON\_T]=T. For the abundances, I recommend using mass fractions for all elements more massive than oxygen. If you do this, changes in the H/He ratio etc. will not cause abundance changes of iron group elements whose mass fraction is independent of changes in the H , He and CNO abundances.  Edit IN\_ITS to set the number of iterations. Depending on the size changes, you may, or may not wish to set  [DO\_LAM\_IT]=T, [DO\_LAM\_AUTO]=T and  [DO\_T\_AUTO]=T in IN\_ITS, and [FIX\_T]=T in VADAT.

If utilizing the [DO\_HYDRO] option, you may need to compute a create a new ROSSELAND\_LTE\_TAB file. This file does not affect the solution --- it only aides in converging the hydrostatic structure. Experience suggests that updating this file is probably not necessary if you are changing only the CNO and other metal abundances. If the model has difficulty converging the hydrostatic structure, update the file! Updates will be more necessary when you are close to the Eddington limit since these models are intrinsically difficult to converge.

\case{Revising the number of super-levels or changing the atomic model.}
    You wish to revise the number of super-levels in a model atom, or change the number of levels, from those used in model {\bf r1}. Use the cpmod command to copy the required files from {\bf r1} to {\bf r2}.  Edit the MODEL\_SPEC file to reflect the changes, and set [LIN\_INT] to true in VADAT.  If using alternative data, or changing the assoication of super-levels with full-levels, you may need to change the atomic data assignments in batch.sh. Edit IN\_ITS to set the number of iterations. In general, and particularly if you are including many additional levels and/or making considerable changes to the model atom, you should set [DO\_LAM\_IT]=T, [DO\_LAM\_AUTO]=T and  [DO\_T\_AUTO]=T in IN\_ITS, and [FIX\_T]=T in VADAT.
    
 
\case{Revising $\log g$}
You wish to revise the gravity of an existing model (which has an associated RVSIG\_COL file). Change the gravity in the VADAT file, and set the number of hydrostatic iterations in HYDRO\_DEFAULTS. The gravity is only utilized when a hydrostatic iteration is performed. Be careful --- simply changing ``$\log g$" in VADAT will {\color{red} NOT} change the hydrostatic structure. {\color{red} When starting the model do not forget to edit HYDRO\_DEFAULTS  to indicate the number of hydrostatic iterations to be performed -- as it was copied from a previous model the number of iterations to be performed will most likely be set to zero.} An error is output to OUTGEN when [DO\_HYDRO]=T in VADAT, and when the maximum error in the hydrostatic structure is greater than 20\% (at any depth), or if the root mean square error is greater than 5\%. The star's MASS in VADAT does not need to be changed --- this will be updated by CMFGEN. When starting the model, you can use [LIN\_INT]=T.

\case{Use TLUSTY hydrostatic structure}
You wish to use the hydrostatic structure from a TLUSTY run. To do this a file containing R, V and SIGMA (called RVSIG\_COL for convenience) must be generated using TLUSTY\_VEL.EXE (or some other program). The [VEL\_LAW] is set to 7, [VEL\_OPT] to RVSIG\_COL, and [VINF] to the largest value in
RVSIG\_COL. The TLSUTY file *.11 (e.g., S40000g400v10.11) is required by TLUSTY\_VEL.EXE. A program (\$cmfdist/{\-}exe/{\-}rev\_rvsig.exe) can be used to revise  RVSIG\_COL (e.g., to change number of grid points, mass loss rate, velocity law etc.). 


\case{Changing  L, \Rstar, or \Mdot}
    {\bf r2} is similar to {\bf r1} except that you are changing one, or more, fundamental parameters (e.g., L). Edit the VADAT file to reflect the new value of L and change [LIN\_INT] to .FALSE..

A file (GREY\_SCL\_FAC\_IN), taken from model {\bf r1}, contains T/T(grey) as a function of Rosseland optical depth. When present, this is used to scale T(grey) so as to provide a better estimate of the temperature structure. This is found to give a significantly better flux conservation at the beginning of an iteration. GREY\_SCL\_FAC\_IN is obtained from GREY\_SCL\_FACOUT (see the GREY\_TAU option description in VADAT description for more details), and is generally created by the cpmod command.

Experience suggests that starting with $\Lambda$-iterations provides greater stability. Thus set [DO\_LAM\_IT]=T  and [DO\_LAM\_AUTO]=T in IN\_ITS. The IN\_ITS file will be updated by CMFGEN to have [DO\_LAM\_IT]=F once sufficient LAMBDA iterations have been done.
 
Experience also suggests that holding T fixed for the earlier iterations often provides greater stability, particularly if using GREY\_SCL\_FAC\_IN. Thus I recommend keeping [FIX\_T]\linebreak=T in the VADAT file, and set
[DO\_AUTO\_T]=T in IN\_ITS (NB: default is F). When this is done, initial iterations have T held fixed. Only when the corrections are ``small'', will T be allowed to vary. The VADAT file will be updated by CMFGEN so that it now contains [FIX\_T]=F. No manual intervention is required.

NB: If you are using the hydrostatic option (rather than a fixed velocity law in both the wind and photosphere) you will need to revise RVSIG\_COL using \$cmfdist/exe/rev\_rvsig.exe to reflect the new values. For example, suppose you just change \Mdot. Only the density structure of the wind will change --
the density structure below the photosphere will be fixed. The program \$cmfdist/exe/rev\_rvsig.exe adjusts $V(r)$ in the photosphere to keep the density
constant. A transition velocity (e.g., 10km/s) to describe the switch from photosphere to wind is requested .

\case{Revising \Teff}
You wish to revise the effective temperature, leaving the gravity fixed. This a little more complicated,
especially if you want to leave the V magnitude fixed. If you are not worried about the V magnitude,
then simply adjust the Luminosity according to \\
$$L(\hbox{\rm new})=L(old)[\Teff (\hbox{\rm new})/\Teff(\hbox{\rm old})]^4 \,\,\,.$$
For exploration runs this is fine --- photospheric O star spectra are sensitive to $\log g$ and $\Teff$ only -- they are very insensitive to the physical size of the star. \Mdot\ might need to be scaled slightly to preserve line EWs.

If you wish to preserve the V flux (at least approximately) you will need to scale the radius. I use a simple little program to compute 
$$\Rstar(\hbox{\rm new})=[B(\hbox{\rm old})/B(\hbox{\rm new})]^{0.5}\Rstar(\hbox{\rm old})$$
\blankhalf
 (assuming $F($V$)= $constant $B($\Teff).\Rstar$^2)$, and compute the luminosity from
	$$L(\hbox{\rm new})=L(\hbox{\rm old})\left(\Teff(\hbox{\rm new})\over\Teff(\hbox{\rm old})\right)^4 \left(\Rstar(\hbox{\rm new})\over \Rstar(\hbox{\rm old})\right)^2\,\,\, . $$

Due to the assumption regarding the scaling of $F($V), and since the relationship between \Rstar\ and $R(\tau=2/3)$ will change slightly, this procedure is only approximate. You will need to revise RVSIG\_COL using \$cmfdist/exe/rev\_rvsig.exe to reflect the new value of \Rstar.

\blankline
In both cases you need to set the number of hydrostatic iterations in HYDRO\_DEFAULTS, and set 
[LIN\_INT]=F in VADAT.

NB: Three stellar parameters can be specified in VADAT: $L$, \Rstar, and \Teff. These are not independent! For non-hyrostatic models (i.e., when DO\_HYDRO=F [or is not present] $L$ and \Rstar\ are used to define the model. \Teff\ is not utilized, and the value you have specified in the VADAT file may be far from reality. When DO\_HYDRO=T, and you remember to actually perform hydro-iterations), $L$ and \Teff\ are used to define the model. \Rstar will be updated to reflect changes in the hydrostatic structure below the photosphere. Also note for historical reasons,  \Rstar\ is the radius of the star at the inner boundary -- it is NOT $R(\tau=2/3)$. This can be obtained from $L$ and \Teff, and will be given in the MODEL\_SUM file.

\interlinepenalty=10000
\case{Include additional ion(s)}
You wish to include an ionization stage, or species, not included in model {\bf r1}. This is one of the more difficult changes. Edit the batch.sh and MODEL\_SPEC file to reflect the desired changes. You may also need to edit VADAT if the appropriate key-words have not previously been included. NB: CMFGEN will stop,
writing an error message to OUTGEN, if a required key-word is missing.

To generate the required XzV\_IN file do one of the following:

\vskip 20pt 
\centerline{\color{OliveGreen}  \bf Preferred Method of generating a non-existent *\_IN file}
 
\begin{enumerate}
\item You will need  EDDFACTOR and RVTJ files from the existing model. Since EDDFACTOR is often deleted after the model is completed, you many need to regenerate it by running the existing model for 1 iteration (usually a $Lambda$ iteration) For convenience I no-longer delete EDDFACTOR. 

\item
Use the {\bf cpmod} file to copy the required input files to the new model directory. You will also need to copy EDDFACTOR, EDDFACTOR\_INFO, and RVTJ.

\item
Edit the batch.sh file so that the correct atomic data is linked. Create the soft-links by typing
{\bf batch.sh ass}.

\item
Use \$cmfdist/exe/guess\_dc.exe to generate guesses for departure coefficients of the unknown populations, creating *\_IN files in the process. DO NOT DELETE the EDDFACTOR file. Alternatively, you
could use an existing departure coefficient file from another ionization stage with a similar ionization potential.

\item
Edit MODEL\_SPEC, VADAT (and RVSIG\_COL if necessary) with the desired atomic model changes. In VADAT you may need to add the species abundance, and lines of the form  \\
\hbox{}  \hspace{0.5in}    F,F   \hspace{0.5in}     [DIE\_XZV] \\
For CNO species T, T; F,T; and T,F are also viable options --- the choice depends on whether resonances have already been included in the photoionization cross-section, and hence should not be doubly counted  (in which case the first variable should be F). The second variable should only be true for C\,{\sc iii}, N\,{\sc iii} and O\,{\sc v} for which states forbidden to autoionize in LS coupling have significant autoionization probabilities. This is true even when the  photoionization cross-sections contain resonances, since the 
photoionization cross-sections have generally been computed in LS coupling. In VADAT  set [USE\_FIXED\_J] to TRUE. 

\item
Run the model, starting with $\Lambda$--iterations. Assuming the correct options are set (as indicated below) in IN\_ITS, the model will automatically proceed to perform a complete model iteration:
\begin{enumerate}
\item
$\Lambda$--iterations with old radiation field ([DO\_LAM\_IT]=T in IN\_ITS).
\item
Update temperature structure utilizing grey approximation (if DO\_GT\_AUTO=T (default) in IN\_ITS).
For a minor change (e.g., a low ionization species only important in the wind, or an impurity
species) this step may not be necessary.
\item
$\Lambda$--iterations with revised radiation field.
\item
Full iteration with T held fixed (if [FIX\_T]=T in VADAT, [DO\_LAM\_AUTO]=T in IN\_ITS )
\item
Full iteration with variable T (if [FIX\_T]=F in VADAT, or if [DO\_T\_AUTO=T] (default) in IN\_ITS).
\end{enumerate}
\end{enumerate}

\centerline{\color{OliveGreen} \bf Older method of generating a non-existent *\_IN file}

I tend not to use this method --  it is simply left as a complicated alternative. If it is a lower ionization stage, simply use the lowest existing ionization stage as the input departure coefficient file. For example, to include Fe\,{\sc iii} in a model, use FeIV\_IN for FeIII\_IN also.(It should also be possible to do the TX trick described below, but it is not yet implemented).

If it is a higher ionization stage, use DISPGEN in the {\bf r1} directory to generate files containing excitation temperatures for each level and each species (WRTX option, with the generated files having a similar format to the departure coefficient files). Then use WR\_F\_TO\_S (WRDC option) to create a new DC file. To include Fe\,{\sc v} in a model which has only Fe\,{\sc iv}, run WR\_F\_TO\_S with the Fe\,{\sc v} oscillator file.  For WRDC, use FeIVTX as the input excitation temperature file. 

Alternatively, generate a departure coefficient file from another model, or perhaps from another species with a similar ionization potential. For stability, the following works best (but may not always be necessary).

Include the species in the code with very low abundance (e.g., $10^{-12}$). Perform enough $\Lambda$--iterations until the populations have stabilized. For a species with weak influence, it may be best to set [LIN\_INT] to .TRUE., so as not to affect the temperature structure.

Copy the XzVOUT files to XzV\_IN files, and increase the abundance. Delete (or rename) POINT1, POINT2, SCRTEMP, and EDDFACTOR. Perform enough lambda and full iterations until the populations have stabilized. At this stage, it may be better to set [LIN\_INT] to .TRUE.. Repeat above procedure until the desired abundance has been reached. Initially the changes in abundance can be large, for example, a factor of 10$^4$. I generally go to the final abundance in a single step.


\case{Adding X-rays}
You wish to add X-rays to an existing model which did not include X-rays. You need to edit the batch.sh file to point to the X-ray emissivity file. You will also need to edit the VADAT file to set [MAX\_CF]=1000, and the X-ray parameters: [INC\_XRAYS], [FIL\_FAC\_1], [T\_SHOCK\_1],  [V\_SHOCK\_1] etc. whose meanings are outlined on page \pageref{vadat_xrays}. Since adding X-rays can cause huge changes in the populations of high ionization stages (changes of $10^{50}$!), I would use the [X\_SLOW] option, starting the models with $\Lambda$--iterations and [FIXED\_T]=T. Once the model is close to reaching the desired filling factors you should check the predicted X-ray flux in OBSFLUX. Since the observed X-ray flux will generally not match that required, you will need to adjust [FIL\_FAC\_1] and/or  [FIL\_FAC\_2], set [X\_SLOW]=F, and restart the model. Note that FIL\_FAC\_1], and [FIL\_FAC\_2] in the VADAT file are automatically updated by CMFGEN. The X-ray emission scales as the square of the filling factor.


\vfill
\eject
\mhd{Computational Checklist for Starting a Model}{start_mod}

\begin{enumerate}
\item
Do I need to delete the POINT* and SCRTEMP files? These must be deleted (or renamed) for a new model, or for a model where you wish to use revised estimates of the populations, which will be obtained from the XzV\_IN files?
\item
Should I delete EDDFACTOR --- it must be deleted for a new model (except if [USE\_FIXED\_J]=T), or if changing the depth or frequency grid. It might also need to be deleted if restarting a model which has had problems.
\item
Should I be using [LIN\_INT]=T or [LIN\_INT]=F in VADAT?
\item
 How do I want to compute the initial estimate of the temperature structure?
\item
 Should [XSLOW] be T or F?
\item
 If using x-rays, have you set [MAX\_CF] to $\ge1000$ in VADAT?
\item
 If you have changed \Teff, is there additional species/ionization stages or processes that should be
 included in the model?
\item
 Should I run several $\Lambda$--iterations first? Running $\Lambda$--iterations with new models is r
 recommended. Similarly, should I start with [FIX\_T]=T in VADAT, and [FIX\_T\_AUTO]=T in IN\_ITS?
\end{enumerate}

\vskip 10pt
\mhd{Computational Checklist for Running/Converged Model}{comp_check}
\begin{enumerate}
\item
Is the Rosseland optical depth at the inner boundary $> 50$ (ideally close to 100) to ensure LTE and the validity of the diffusion approximation at depth?
\item
Has convergence been achieved (OUTGEN, CORRECTION\_SUM and STEQ\_VALS files).
\item
Is flux conserved at all depths (OBSFLUX)? Convergence should typically be better than 2\%.
\item
Is the electron energy balance equation satisfied (GENCOOL), particularly in the outer layers of the model? This only needs to be checked when new physics is included, or in new parameter range.
\item
Are there any non-standard error message or warnings in OUTGEN.
\end{enumerate}
 


\vfill
\eject
\mhd{Troubleshooting}{troub_shoot}

CMFGEN is fairly stable, but parameter regimes are encountered where there may be convergence difficulties. Parameters controlling convergence, which are set in VADAT, have �default� values chosen to ensure convergence rather than a high rate of convergence. 
If CMFGEN halts unexpectedly the user should check {\bf OUTGEN} (and {\bf MODEL\_SCR} if it exists) for an error message. Hopefully most error messages are self-explanatory. You should also check {\bf batch.log} for any fortran errors that may have occurred.

\blankline
Convergence difficulties can arise for several reasons:

\vspace{-5pt}
\begin{enumerate}

\item
You are moving from a low-ionization to a higher ionization model. In such cases an ionization front may develop in the wind. Across the ionization front, populations may change by orders of magnitude. In such cases, CMFGEN has trouble eating its way through the front. The solution of the transfer equation across the front is also not reliable.  When the problem arises, it usually occurs with the ground state of a dominant species such as He\,{\sc ii}.

Possible remedies:
\begin{enumerate}
\item
If possible, use higher ionization models as input.

\item
Restart the model with new input data (i.e. XzV\_IN). Remove the fronts from the input files by hand. Use the [LIN\_INT]=T option in VADAT to restart the model. Remember to delete (or rename) the POINT1* and SCRTEMP files.

NB: The WRDC(OWIN=5,7,10,14) option in DISPGEN can be used to write out new DC files (called XzVDC by default) with depths 5 through 7, and 10 through 14 omitted. The absent depths will be recreated in CMFGEN through interpolation. Before using this option, make sure that the RVTJ and POPDUM files have been updated with the latest population estimates. This can be done by running CMFGEN for 0 iterations.
The WRDC file will not overwrite existing XzV\_IN files. An extension of \_IN can  be specified by using
the hidden option EXT [i.e., WRDC(EXT=\_IN)]. As WRDC will not overwrite existing XzV\_IN files, they
need to be deleted, renamed, or moved. Generally I move the existing XzV\_IN files to a subdirectory ``jnk''.
NB: Be careful not to move/delete GREY\_SCL\_FAC\_IN -- this may still be needed when undertaking hydrostatic iterations.

\item
Increase the number of grid points in the model, especially in the neighborhood of the front. To do this you will need to create a file containing the new R grid (same format as DC files) which can be done by hand or by using \$cmfdist/exe/rev\_rdinr.exe. Alternatively you will need to modify the RVSIG\_COL file if it is being utilized. An alternative is to use [INC\_GRID] and related options to insert extra points in the neighborhood of the ionization front. Note that this does not increase the number of points in the model, and hence the memory requirements. It is only for the solution of the transfer equation.
       
Remember: It does not matter how you get the model to converge, only that you do.
\end{enumerate}
 
\item
Problems of convergence in the outer regions of the wind that appear to be temperature related.
Possible remedies are:
\begin{enumerate}
\item
Fix the temperature (i.e., set [FIX\_T]=TRUE in VADAT), and converge the model. Convergence with fixed T is generally rapid. Then set [FIX\_T]=FALSE and try again.
\item
 Ensure that [SCL\_LN]=TRUE in VADAT
\item
 Increase the number of super levels. This seems to be particularly important at low wind densities (e.g., \Mdot$ < 10^{-7}$ \Msunyr).
\end{enumerate}

\item
 Program computes a large negative luminosity, and perhaps crashes when computing the observed flux.

Possible remedies:
\begin{enumerate}
\item
Ensure that [FG\_OPT] is INT/INS.
\item
If the problem occurs on an early iteration, check that input data is correct
and consistent (for example, \Mdot\ in the VADAT file is the same as that used in RVSIG\_COL;
GREY\_SCL\_FAC\_IN has not been corrupted).
\item
 If the problem occurs on the last iteration, change the [FRM\_OPT] to INT.
\item
 Change [METHOD] option to ZERO, although LOGMON should be stable.
\item
 Check validity of input data. Problem most often occurs when the populations associated with some strong resonance transition (e.g., C\,{\sc iv} 1550) are way off. This can occur, for example, when you have used as input a C\,{\sc iv} file in which the 2p state is not split, for a model in which it has been split. Use REWRITE\_DC first (or a macro in an EDITOR) to update the XzV\_IN input file.
\item
Increase the number of depth points and/or use a finer spatial grid at the outer boundary.
\end{enumerate}
 
You can generally return to the original option values (but not the finer spatial grid) once the model has begun to converge.

\item
 Poor model convergence and bad fluxes (as seen in OBSFLUX) at depth.

Possible remedies:
\begin{enumerate}
\item
 If it occurs in a model with new stellar parameters, ensure that [LIN\_INT] was FALSE when the model was started. If it was TRUE, restart the model (deleting EDDFACTOR, POINT*, and SCRTEMP).
\item
 Check that the parameters, such as \Mdot, used to create the RVSIG\_COL file (if used) and in the VADAT file are compatible.
\item
Check that the GREY\_SCL\_FAC\_IN has not been corrupted. This can occur, for example, when you copy the file from a non converged model, particularly a model having convergence difficulties.
\end{enumerate}

\item
 Convergence is proceeding smoothly (with correction possibly smaller than 5\%), but then goes wild, or changes start to systematically increase.

Possible remedies:
\begin{enumerate}
\item
 Do nothing, and continue iterating. The problem could simply arise because information from the optically thick regions of the wind needs time to propagate to the outer regions of the wind.
 \item
  Change the [NEG\_OPAC\_OPT] to SRCE\_CHK. It is possible that the problem arose because some transition (generally some insignificant transition in the IR) has suddenly decided to laze.
\item
 Force the linearization matrix to be recomputed --- This can be done by editing or deleting BAMATPNT. The code will eventually recompute the BA matrix anyway if it continues to detect that the changes are continuing to increase.
\item
 Average the last 2 iterations, which can be useful if the corrections are oscillating in sign.
\end{enumerate}

\item
Poor temperature convergence at a few depths, or T correction goes wild at 1 or 2 depths. This could be a result of [NEW\_LINE\_BA] being set. For a non-SN model this can be set to false, and you should set [SCL\_LN]=T. For a SN model, it may be necessary to set [IND\_BA\_METH] to a slight higher depth index -- the  value should be larger than the depths that have convergence issues.

\item
Poor convergence of a single (few) level(s) at a single (few) depth(s).

Possible Remedies:
\begin{enumerate}
\item
Create a file ADJUST\_CORRECTIONS to provide a factor to scale all corrections at the
specified depths (negative relaxation). In the file you can specify a single depth with the
scale factor (e.g., 20,  0.3) or a range of depths (e.g., 20:22, 0.3) with one specification per
line. This procedure seems to solve many problems, and can be implemented at any time
while the model is running. The scaled corrections are not applied to $\Lambda-$iterations.
\item
Rename the *OUT files to *\_IN, making sure that the *OUT files are for the most recent iteration. Alter the bad departure coefficients more reasonable values and recommence computation of the model. Remember to use the [LIN\_INT]=T option in VADAT to restart the model, and move the POINT* and SCRTEMP files to a temporary subdirectory (in case something goes wrong, and you need them again).
\item
 Include more depth points in the neighborhood of the instability.
\item
Average successive population estimates, since the changes can alter in sign. This can be done using \$cmfdist/exe/do\_ng\_v2.exe.
\end{enumerate}

\pagebreak
\item Asymptotic convergence is poor, with the maximum corrections remaining at the 1\% level. 
Possible Remedies:
\begin{enumerate}
\item
Decrease number of �unimportant� levels. 
\item
Use do\_ng\_v2.exe to average the level populations over the last 2 iterations,
and restart the model.
\item
Pray for guidance.
\end{enumerate}

\end{enumerate}

\vfill
\eject
\mhd{CMFGEN FILES}{cmfgen_files}

\shd{Main Model Control Files}{control}


%Model data

\begin{file_list}{GRE}


\item[IN\_ITS]

Controls the iteration type and cycle.
\begin{plt_list}{xxDO\_LAM\_AUTO}
     \item
     [[NUM\_ITS\hbox{] }]        Number of iterations.
     \item
     [[DO\_LAMBA\_IT\hbox{] }]    If TRUE, do  lambda-iteration.
     \item
     [[DO\_LAM\_AUTO\hbox{] }] Automatically switch from $\Lambda$-iterations to full iterations (Default=T).
     \item
     [[DO\_GT\_AUTO\hbox{] }] Do a grey temperature iteration after (automatically) revising USE\_FIXED\_J (Default=T).
     \item
      [[DO\_T\_AUTO\hbox{] }] Allow temperature to vary when sufficient convergence has been obtained (Default=F).
\end{plt_list}
     
The first two keywords must be in the order given above whilst the last three keywords do not need to be present. The first two keywords can be edited while CMFGEN is running. Entering a 0 for NUM\_ITS will halt CMFGEN after the next complete iteration. Entering a number larger than the value previously in the file will result in an additional \\
\hbox{}\hspace{1cm}
        NUM\_ITS(new) -NUM\_ITS(old) \\
\noindent
iterations. [DO\_LAMBDA\_AUTO] is optional, and is assumed to be TRUE.

\item[HYDRO\_DEFAULTS]    Required if [DO\_HYDRO] is set to TRUE in VADAT. File is used to control iteration of the hydrostatic structure. This may be edited while program is running. Only [N\_ITS] must be present, although as a habit I would include the first 8 options listed below: 

\begin{indented_list}{ATOM\_DENNNN}
\item
[IN\_ITS]     Number of iterations remaining.
\item
[FREQ\_ITS]     Indicates how often hydrostatic       structure is updated. Default is 8.
\item
[STRT\_ITS]   Begin hydrostatic correction after iteration STRT\_ITS. Default is 20, but in many cases I start after the first iteration.
\item
[MAX\_R]  Used to set maximum radius of model defined as MAX\_R$\times R_{\rm c}$ where $R_{\rm c}$ is the connection radius. Note that it is defined in terms of the connection radius (i.e., where the wind joins to the hydrostatic core) -- not in terms of the inner radius.  {\color{red} If this keyword is NOT set, RMAX/\Rstar\ continues to increase (slightly) with each hydrostatic iteration.} 
\item
[ATOM\_DEN]   Density at outer boundary for plane-parallel model. Gives an indication of how far to extend model.
\item
[OB\_OPT]    Indicates how depth grid at outer boundary is specified (DEFAULT, SPECIFY, POW, NONE). If SPECIFY (the preferred option), the grid locations are explicitly specified.
\item
[NOB\_PARS]   Number of grid locations that are added to the ``default'' grid to produce a fine grid at the
outer boundary. The ``default'' grid is chosen to satisfy optical depth and velocity constraints.
\item
[OB\_PN]    Used to specify the optical depth increments at the outer boundary (N=1, 2, ..., NOB\_PARS). If $\Delta\tau=\tau(2)-\tau(1)$ (before extra point insertion) then the revised $\tau$  is $\tau(1+N)=\tau(1)+\Delta\tau$/OB\_PN.  Because a first order boundary condition is used at the outer boundary, the ratio $d\tau(2)/d\tau(1)$ should be large (i.e., Both OB\_P1 and OB\_P1/OB\_P2 [when OB\_P2 is specified] should should be $> 10$). O stars with weak winds require the largest value ($> 30)$.

\item[RES\_REF] Resets reference radius when using old velocity.

\item[GAM\_LIM] Maximum value for the Eddington ratio (defined using the total opacity). Defaults is 0.98. Only effects models which are VER close to  the Eddington limit.t

\item[VC\_ON\_SS] Defines the connection velocity (normalized by the sound speed) where the photosphere joins the wind. 

\item[UP\_GREY\_SCL] This option updates the GREY\_SCL\_FAC\_IN file. This, in principal, means that the T structure is potentially more
accurate after it updates the density structure. Thes is true, provided the T structure in the hydrostatic region is converged. In practice this option can
cause problems when it is not fully converged, since it can cause it be reset to a bad T structure.
Default is Fals.
 
\item[TAU\_REF] Sets the reference optical depth for $g$ and \Teff. The default is 2/3. This option is useful for W-R models in which the normal photosphere, defined by $/tau=2/3$ is located in the wind.

\item
[BETA2] Allows extra freedom with the velocity law around the connection point. Default is BETA (as passed from VADAT), in which
case [BETA2] has no effect.

In the wind we have:

$$V(r) = {{(\Vinf - 2V_c)(1-R_c/r)^{\hbox{$(\beta + (\beta_2-\beta)(1-R_c/r) )$} }}\over { 1+ exp{ (R_c-r)/h) }}}$$

where $h$ is the scale height at the connection point. At the connection point, $\beta_{eff}=\beta$
while in the wind $\beta$ approached $\beta_2$.

\item
[dLOG\_TAU] Logarithmic spacing in Tau for new R grid. Default is 0.25.

\item
[VSCL\_FAC] Minimum V(I+1)/V(I) for new R grid ($<1$). Recall that $V(I+1) < V(I)$. Default is 0.75D0.

\end{indented_list}

In general, options can be found in \$cmfdist/new\_main/do\_cmf\_hydro\_v2.f

\newpage
\item[MODEL\_SPEC]    Parameter file for CMFGEN that specifies the following:

$\bullet$ The number of depth points and core rays. \\
$\bullet$
The species and ionization stages which are to be included in the model. \\
$\bullet$
The number of SUPER-LEVELS, FULL-LEVELS and  IMPORTANT-LEVELS for each ionization stage. \\
$\bullet$
The maximum number of overlapping lines.

The KEY\_WORDS for MODEL\_SPEC are explained on page \pageref{sec_model_spec}.

\item[VADAT] Main file specifying model parameters and options. The various options are explained later in this document (page \pageref{sec_vadat_opts}).

\blankline
\shd{Other Control Files}{other_control}
\vspace{-10pt}
\item[ADJUST\_R\_DEFAULTS]  Control file used with SN models to revise the R grid while the code is running.
 Used when [REV\_RGRID] is set true in VADAT.

\begin{indented_list}{ATOM\_DENNNN}

\item[N\_ITS] Number of revisions to be made to R grid.

\item[STRT\_ITS] First iteration for R grid revision.

\item[FREQ\_ITS] How often R grid revisions are to be performed.

\item[GRID\_METH] Space in which R grid is redefined. TAU\_SPACE uses a combination of $\tau$ and $V$ space to determine the spacing -- there is a maximum $\Delta \log V$ and a maximum $\Delta log \, \tau$. VEL\_SPACE uses constraints on $\Delta \log V$ only.

\item[GRID\_TYPE] Available options are REFINE, UNIFORM. UNIFORM redefines the entire R grid.

\item[OB\_OPT]  Options are DEFAULT or SPECIFY. SPECIFY indicates that parameters are listed to control grid spacing at the outer boundary.

\item[NOB\_PARAMS] Number of outer-bounndary parameters

\item[OB\_PN]   Parameters for the outer boundary: N=1, ..., NOB\_PARAMS. With OB\_P1=30, dTAU(2)/dTAU(1) will
be 30. 

\item[IB\_OPT]  Options are DEFAULT or SPECIFY. SPECIFY indicates that parameters are listed to control grid spacing at the inner boundary.

\item[NIB\_PARAMS] Number of inner-boundary parameters

\item[IB\_PN]   Parameters for the outer boundary: N=1, ..., NIB\_PARAMS. With IB\_P1=30, dTAU(ND-2)/dTAU(ND-1) will
be 30.

\end{indented_list} 

\vfill
\eject
\item[SPECIFY\_IT\_CYCLE]  When this control file is present it can be used to force a particular iteration sequence of iteration types. The cycle is repeated unless the NORM option is found. Specify only one option, per line. The number of consecutive iterations of the same type is specified by a number following the option -- if no number is present 1 is assumed. Do not place [] around the iteration options. There is no check on the validity of
the option ordering. Thus users should not allow T to vary with a fixed BA  if the previous computation geld T fixed. It is all right to follow a LAMBDA-iteration with a full iteration PROVIDED the BA matrix was computed on an earlier iteration (a LAMBDA iteration does not overwrite the BAMAT file). SPECIFY\_IT\_CYCLE  may be changed while the program is running. Unrecognized options are deleted from the file and the default iteration option is used.
The file is read in \$cmfdist/subs/specfy\_it\_cycle.f.

\begin{plt_list}{FULL(FIXT,FIXBA) N}

\item[LAM  N]  Do N lambda iterations.
\item[FULL  N] Do N full iterations -- the BA matrix is computed and the temperature is allowed to vary.
\item[FULL(FIXT)  N] Do N full iterations -- the BA matrix is computed but the temperature is held fixed.
\item[FULL(FIXBA) N] Do N full iterations -- the BA matrix is not computed. Behavior of T variation depends on the previous iteration. 
\item[FULL(FIXT,FIXBA) N] Do N full iterations -- the BA matrix is not computed and the temperature is held fixed.
\item[NORM] Return to a normal iteration cycle. When this record is reached (it is cycled to the top) lower statements are ignore.

\end{plt_list} 

\vspace{-10pt}
\item[ADJUST\_CORRECTIONS]  Optional control file containing parameters that can be used to scale corrections in order to facilitate convergence. This control file may be changed while the program is running. This file is used by \$cmfdist/subs/iddle\_pop\_corrections\_v2.f.

\begin{indented_list}{ATOM\_DENNNN}

\item[RELAX]  A value of Z (say 0.5) indicates that all corrections should be scaled by a factor of 0.5. $Z <1$ is equivalent to under-relaxation, and can help difficult models (especially those oscillating) converge. $Z>1$ corresponds to over-relaxation and might help a model converge faster (or it might cause it to diverge, so
be warned!).

\item[T\_LIM] Maximum fractional change allowed to the temperature.

\item[LST] Beginning depth. If not present, 1 is assumed.

\item[LEND] End depth. If not present, ND is assumed.

\item[CONSIT\_CNT] Checks every CONSIST\_CNT iterations whether adjacent populations are consistent. No consistency check is performed when the parameter is not present, or if it is $\le$ 0. This option may help
models struggling to converge early on but users should note it is still under development.

\end{indented_list}

\vfill
\eject
\blankline
\shd{Model Input Files}{input}
\vspace{-10pt}
\item[GAMMAS\_IN]    Estimate of mean ionization for each atomic species DUM. Generally the GAMMAS file of a previously converged model. The file contains a series of data sets giving the electron density, radius, temperature, and the number of electrons per species in the model.

\item[GREY\_SCL\_FAC\_IN]    Gives T/T(grey) as a function of Rosseland optical depth. Taken from previously converged model (GREY\_SCL\_FACOUT). When present it is used to scale T(grey) so as to provide a better estimate of the temperature structure, and hence give significantly better flux conservation, for the first iteration. 

\item[RVSIG\_COL]    Needed when using [VEL\_LAW=7] \& [VEL\_OPT]=RVSIG\_COL. Contains R, V, \& SIGMA for model in column format. Use  an old model file (which can be revised to accommodate a new \Rstar\ or \Mdot\ using \$cmfdist/exe/rev\_rvsig.exe), or create a new version using \$cmfdist/exe/tlusty\_vel.exe or  \$cmfdist/exe/wind\_hyd.exe.

\item[ROSSELAND\_LTE\_TAB]  Required if [DO\_HYDRO] is set in VADAT. Contains Rosseland mean opacity as a function of T \& Ne. Can be computed using \$cmfdist/exe/main\_lte.exe. Ideally, it should be computed using the same model atoms and abundances as specified in VADAT and MODEL\_SPEC.
Values in this file DO NOT affect the accuracy of a converged model.

\item[XzV\_IN]    Estimates of departure coefficients as a function of depth for each ionization stage. It is generally the XzVOUT file of a previously converged model, although it can be also created by an option (WRDC) in DISPGEN.  Use cpmod to handle their creation from an old model. 

    The first non-blank line of XzV\_IN contains a format date for the file. This is to allow for future format changes. The second line provides brief set of parameters relevant to the model and ionization species: the stellar radius (inner boundary), the luminosity, the number of full levels in the atom, and the number of depth points (ND). There then appears ND data sets, one for each depth, and ordered from the outer boundary to the inner boundary. Each data set is separated by a blank line. The first line of each data set contains the {\color{blue}  radius, the ion density, the electron density, the temperature, the ionization fraction, the velocity, the volume filling factor and an integer depth identifier (1 to ND)}. Only the first 4 quantities and the volume filling factor are utilized when starting a new model (but the other columns must be present). Subsequent lines contain the departure coefficients for each level in the FULL atom, ordered from 1 to NXzV\_F.

\interlinepenalty=10000
\item[SCRTEMP]     SCRTEMP provides the starting populations when it (and POINT1) are present. The file is only valid for the model for which it was created. If you want a model to use revised XzV\_IN files, this file (and POINT1 and POINT2) MUST be deleted. Further discussed under �Scratch Files�.

\item[T\_IN]    Estimate of the T structure of a model. Generally an XzVOUT file from a previous model. NB: The temperature will be adjusted (if the appropriate option in VADAT is set) to allow for changes in the model parameters.

\end{file_list}

\pagebreak
\blankline
\mhd{Atomic data files}{atomic_files}
 
\blankhalf
\centerline{\color{blue} \bf Generic data files}
\vspace{-15pt}
\begin{file_list}{XzV\_F\_OSCDAT}

\item[HYD\_L\_DATA]    Hydrogenic photoionization cross sections for `{\it l}\,' states.

\item[GBF\_N\_DATA]    Bound-free gaunt factors for hydrogen.

\item[TWO\_PHOT\_DATA]    Data giving atomic data for 2-photon process. One file contains data for all species.

\item[CHG\_EXCH\_DATA]    Data giving charge exchange reaction rates. One file contains data for all species.

\item[XRAY\_PHOT\_FITS]    X-ray photoionization cross-sections. One file contains data for all species.

\item[RS\_XRAY\_FLUXES]    Fluxes as a function of shock-temperature. Used when including x-rays in the model. Original data was calculated using the old Raymond \& Smith code. Newer data has been computed using APEC (Astrophysical Plasma Emission Code) which is a heavily revised version of the Raymond \& Smith code. Several data files exist: The file (\$cmfdist/misc/rs\_xray\_flux\_sol.dat) is the data used with most existing models, but is nolonger the preferred data file.  Three newer data sets are available: for the galaxy (gal\_xray\_hr.dat), for the LMC (lmc\_xray\_fluxes.dat) and for the SMC (smc\_xray\_fluxes.dat). They were computed by Janos Zsargo using solar, or scaled solar, abundances.

\item[arnaud\_rothenflug.dat] List of fits to photoionization cross-sections for elements up to Ni. Used when computing the influence of non-thermal ionizations in supernovae models. Beware: not all ions are presently included in the supplied file.

\item[NUC\_DECAY\_DATA] Nuclear decay data for supernovae models. File contains isotope masses and half-lives. The fill also list the gamma-rays and their emission probabilities generated by the different decay chains. Beware: some ``irrelevant" species can be treated as stable. At present, only 1 and 2-step decay chains are treated.


\end{file_list}

\vfill
\eject
\blankline
 \centerline{\color{blue} \bf  Species/ion dependent data files}
\vspace{-15pt}
\begin{file_list}{XzV\_F\_OSCDAT}

\item[XzV\_F\_OSCDAT]    Energy levels and oscillator strengths.

\item[XzV\_COL\_DATA]    Collisional data. Tabular format.
             
\item[PHOTXzV\_A]    Photoionization data. Tabular format.
     
\item[XzV\_F\_TO\_S]    Provides the links between the full-levels and super-levels. Only required when NS $<$ NF.  Use WR\_F\_TO\_S to assist in creating these files.

\item[DIEXzV]    Dielectronic data: Presently taken from Nussbaumer and Storey.

\item[AUTO\_XzV\_DATA] Contains a list of autoionization level probabilities for levels above the ionization continuum. Only needed for some ions.

\end{file_list}

\vfill
\eject
\mhd{Output Files}{output_files}

\begin{file_list}{OBSFLUX}

\item[MODEL]     Contains MODEL information, data from VADAT, and headers from the atomic data files. If code halts, check the MODEL\_SCR and OUTGEN files for error messages. MODEL\_SCR is converted to MODEL after a few seconds/minutes.

\item[MOD\_SUM]     Brief formatted summary of the model. Useful for bookkeeping and archival purposes. It is created/updated on the last iteration.

\item[OUTGEN]     Summary of the results and corrections for each iteration. Look at this file to check on the progress of a model, and for warnings and error messages. Some warnings are generated all the time, and are purely informational. For example, the routine contains an indication of how many (very) weak transitions have been neglected (because of switch settings in VADAT). At the beginning of the model, and on the last iteration, CMFGEN will also output a list of ions that could be deleted, or might be added, to the model. These are a guide only -- some ions suggested by CMFGEN may only affect the temperature structure at $\tau=100$, and hence have no effect on the spectrum. For hot W-R models, it is possible that you may need to include 10 ionization stages of Fe, although in practice (and depending on model purpose) this is usually not necessary for spectral modeling.

OUTGEN also provides a quick means of checking on the progress of a model. The command \\
\hbox{}\hspace{1in}        {\color{OliveGreen} \bf grep  Maximum OUTGEN} \\
will return a list of the maximum correction as a function of iteration. Similarly, \\
\hbox{}\hspace{1in}         {\color{OliveGreen} \bf grep  Lum OUTGEN} \\
will return the radiative luminosity (in the comoving-frame) at the outer and inner boundaries as a function of iteration.

Since the maximum correction is not always representative of the corrections, OUTGEN also lists for each iteration the top 10 fractional corrections which decrease (DEC\_VEC) and increase (INC\_VEC) variables. 

\item[XzVOUT]     Final departure coefficients for each model. Only output after final iteration. Can be created by running CMFGEN for 0 iterations. Can also be created from RVTJ and POPDUM files using DISPGEN. Same format as XzV\_IN discussed earlier.

\item[GAMMAS]     Estimate of mean ionization for each atomic species DUM. Used for initiating new models.

\item[RVTJ]     Contains main atmospheric structure vectors, e.g., R, V, T, Ion and Atom population, Rosseland mean optical depth etc.. As with XzVOUT and the POPDUM files, it is only output after the final iteration. It can be created by running CMFGEN for 0 iterations. However in this case the mean opacities will be zero. This file is required by DISPGEN and CMF\_FLUX.  The format of the file is fairly obvious. For historical reasons, the populations of H \& He are also output to this file, but never used. 

\item[POPDUM]     Level populations for all (computed) ionization stages of species DUM (e.g., POPCARB, POPHYD). These files are required by DISPGEN and CMF\_FLUX. The file format is fairly obvious with the following clarifications. The oscillator date is only used to provide a check that the correct atomic data is being used when running DISPGEN or CMF\_FLUX. The populations of the ionization stages are listed sequentially in the file. Populations are listed for each level, then for each depth (d=1 is listed first). The final set of ND numbers associated with each ionization stage is the (SL) ion density, which is only used if the following ionization stage is unavailable. To read these files into another program, look at the read routines used by DISPGEN or PLT\_SPEC.

\item[OBSFLUX]     Contains frequencies and observed fluxes. These are displayed by PLT\_SPEC. OBSFLUX also contains information on how the comoving-frame luminosity varies with depth. This provides a check on the accuracy of the flux conservation. For stellar models the luminosity only enters into the transfer solution at the inner boundary (diffusion approximation) -- thus the constancy of the luminosity with depth provides a realistic check on the accuracy of the model. The spectrum in this file SHOULD NOT be considered as the final observed spectrum --- use CMF\_FLUX to compute the observed spectrum.

                  The file contains a list of frequencies (in $10^{15}$\,Hz) and then the list of the corresponding fluxes (observer''s frame) in Jansky�s (assuming $d=1$\,kpc).  The file then lists the radiative luminosity, dielectronic and implicit recombination line emission (if dielectronic lines are treated as individual lines), the total line emission (if not using blanketed option), the mechanical luminosity (if radiative equilibrium holds (L is not conserved in the CMF and this gives the correction to be made to the radiative luminosity), the total radiative luminosity, and the X-ray luminosity. Ideally, the total radiative luminosity (radiative is the wrong word) should be constant -- in practice there will be small variations at the 1\% level. Extensive testing has shown that errors of ~1\% have no influence on the results. The radiative luminosity is given at each depth, while other terms are evaluated only for the shell centered at each depth.  Finally we give a summary of the total X-ray luminosity emitted by the gas and the observed X-ray luminosity (which will be affected by absorption and by any intrinsic X-ray emission coming from the star and its wind).

Note: The Luminosity output to OBSFLUX is the luminosity in the COMOVING frame. For normal stellar work, this will be very close to the luminosity of the star in the observer's frame. However, for time-dependent SN calculations the CMF luminosity  will generally be lower than that computed by integrating over the observed spectrum [since for a static atmosphere $H_{obs}=H_{cmf} + \beta(J_{cmf} + K_{cmf})$]. However, when the luminosity is declining rapidly, the CMF luminosity may actually go above the observer's frame luminosity. This occurs because the CMF explicitly includes time dependence effect and if RMAX is large, you are effectively computing the luminosity at an earlier epoch (there are also additional issues). Fortunately issues related to explicit time dependence are not crucial, and obs\_fin should provide the most reliable spectra and luminosities.


\end{file_list}

\vfill
\eject
\shd{Diagnostic output files}{diag_files}


\begin{file_list}{CORRECTION\_SUM}

\item[BA\_ASCI\_N\_DN] This file contains the statistical equilibrium equations for a single depth N. N is
set/changed by editing \$cmfdist/new\_main/subs/generate\_full\_matrix\_v3. It can be used with \$cmfdist/misc/tst\_ba\_mat\_sol.f (and a diagonal operator) to check the solution of the rate equations, and to check for near singularities. These files can be deleted at the end of a model run and are irrelevant for most users.

\item[CORRECTION\_SUM]     This file lists the number of corrections greater than 100\%, 10\%, 1\%, 0.1\%, 0.01\% and 0.001\% as a function of depth. This file is very useful. In most cases, the file will show that each depth has a similar distribution of corrections, and that any differences change fairly smoothly with depth. However in some models a few populations may behave unusually. For example, in the CORRECTION file, all corrections may be small except at one depth. This may indicate a convergence problem which can be corrected, either by hand, or using one of the correction procedures that are available.

\item[CORRECTION\_LINKS]    Prints the 5 largest corrections at 5 to 10 depths, and the identification of those
corrections with the atomic ionization stage and super-level to which each applies. Originally this had to be done using the STEQ\_VALS and MODEL files. For use when having problems with convergence.

\item[EWDATA]     Equivalent widths computed when using CMF or SOB modes. File is empty in BLANKETING mode. Generally empty with CMFGEN calculations, but often used with CMF\_FLUX calculations for W-R stars, CSPNs. Data is really only applicable to emission
lines (especially with SOB option).

\item[CFDAT\_OUT]     Continuum frequencies ---  primarily used by the author for diagnostic purposes.

\item[GENCOOL]     File containing HEATING and COOLING rates of the electrons for all the various processes (e.g., photoionization, recombination, collisional excitations and ionizations, X-ray cooling, adiabatic cooling). The net cooling rate should ideally be zero if the code is working correctly, and when the model is converged. In practice the use of super levels prevents this from happening (Hillier 2003) . Increasing the number of super levels and/or altering the level assignments should/can reduce the net cooling rate. Generally errors of 20\%, or less, appear not to be important for most models. Ideally, the effect of these errors should be tested for models in the relevant parameter range. Typically the error is less than a few percent. At the inner boundary the errors may be larger but this does not matter providing the luminosity has been conserved (check OBSFLUX). The larger error at the inner boundary primarily occurs because of limited convergence, and because it is computed using quantities that have already suffered cancellation. The file format is somewhat similar to XzVPRR. This file should be checked occasionally as it can reveal problems with the models or bugs in the code.  The program {\color{OliveGreen} \$cmfdist/exe/mod\_cool.exe} can be used to create a smaller summarized version of GENCOOL (page \pageref{sec_mod_cool}).

\item[HYDRO]     Summary of the hydrostatic terms, and radiation force important for driving the wind. Output depends on the stellar mass as input through VADAT. At present, the [MASS] in VADAT only affects this file. [MASS] in the VADAT file is automatically updated after a hydrostatic iteration has been performed. HYDRO can be rewritten for a new stellar MASS using {\color{OliveGreen} \$cmfdist/exe/rev\_hydro\_turb.exe}. This file can be used to check for consistency between the adopted mass-loss rate and velocity law, and the radiation force.

\item[J\_COMP]     Contains the value of the mean intensity at the inner and outer boundary as a function of frequency. Two mean intensities are output. One is computed using the ray by ray solution, the other using Eddington factors. Ideally they should be identical;  in practice they differ by around 1\% at the outer boundary (with larger values at some frequencies). This file is only created on the final iteration. If a major crash occurs, it is often useful to rerun the code for 1 iteration so that this file is created. For very low mass-loss rate models, with large Rmax, J\_COMP might reveal larger discrepancies in the outer boundary fluxes, particularly at infrared wavelengths. If the later occurs, a warning message is output to OUTGEN. In such cases you need to use a smaller step size at the outer boundary. The program {\color{OliveGreen}\$cmfdist/exe/plt\_jc\_comp.exe} can be used to plot J and the errors at the outer or inner boundary.

For time-dependent SN model calculations the difference in J(Edd) and J(ray) may be larger. This occurs because J(Edd) depends explicitly on the time while J(ray) only
has an implicit dependence (throuh a dependance on a time-dependent scattering term),

\item[JEW]     Used when computing line equivalent widths. File is empty in blanketing mode.

\item[LINEHEAT]     Net cooling rates for each bound-bound transition. Defined as $h\nu N_u A_{ul} Z_{ul} /4\pi$, and it is thus the cooling per steradian. Because of the choice of program units, it is scaled by a factor of $10^{10}$. In addition this file contains two running sums of the net cooling rates -- the sums differ in how the rates are scaled. I use this file when there are issues with the accuracy of the electron heating/cooling
convergence, as indicated in the GENCOOL file, to work out lines that are potentially causing problems. \$cmfdist/exe/plt\_ln\_heat.exe can be used to plot data in LINEHEAT. This file is only created on the last iteration, is VERY large, and generally should be deleted after a model is finalized. 


\item[MEANOPAC]     Summary of various optical depth scales (Rosseland, Flux, and electron scattering) as a function of depth. The optical depth scales have been explicitly corrected by the volume filling factor. The optical depth scales can be computed from the listed opacities using $d\tau = -f \chi \, dz$ where $f$ is the volume filling factor.

\item[NEG\_OPAC]     Lists each frequency, and the transition, for which a negative optical depth occurred. These negative optical depths arise due to population inversions, and generally (but not always) occur in the infrared. We have implemented automatic procedures in CMFGEN to handle these occurrences. They tend to occur more often when the model is unconverged.

\item[NETRATE]     Diagnostic file containing the net rate, $Z_{ul}$ $(=1-J_{ul}/S_{ul})$, for each bound-bound transition, ordered by increasing frequency. For each line, the net rate as a function of depth (1 to ND) is given. This file is only created on the last iteration, is VERY large, and thus is generally deleted after a model is finalized. 

\item[PRRRXzV]     Check file containing recombination and ionization rates for species XzV. If the code is working correctly, the total RATES for each species should be 0 (i.e.,  NET\_RATE / TOTAL\_RATE $<< 0.01$ for a converged model). If this condition is not met, there is an ERROR.

\item[STEQ\_VALS]     On each iteration, 4 arrays are printed out. The first array, of length ND, is
simply the radiative equilibrium equation. The second array is the value of each RATE (i.e., statistical equilibrium) equation at each depth. Tabulated as �STEQ�, with equation depth listed horizontally, and rate equation vertically. The first data set lists the statistical equilibrium equations for each ionization stage in sequential order. For a species with NXzV levels there at least NxZV+N where N is larger than 2. These extra equations relate to the ionization equilibrium, species conservation, and if present, ionizations to excited states and Auger ionizations. The third data set lists the merged rate equations in sequential order where degenerate equations have been combined/removed (matrix of size NT $\times $ ND). Also output, under ``STEQ SOL'', is the fractional correction to each population --- {\color{red} for historical reasons a $-$ve correction corresponds to an increase in the variable}. This file can be very useful when diagnosing convergence difficulties.    To facilitate interpretation of this file, two additional diagnostic files, CORRECTION\_SUM and CORRECTION\_LINKS, are created. 

    While the STEQ\_VALS file is not user friendly it is very useful. For example, to find a summary of corrections to T, check the value of NT in the file MODEL. If this is 600, the command
\blankhalf
\hspace{1in}
        { \color{OliveGreen} \bf grep �600(31)\#� STEQ\_VALS}
\blankhalf
will list the corrections to T  for depths 31 to 40 as a function of iteration (corrections are grouped in blocks of 10). Note: If ND~$ > 99$ the command is 
\blankhalf
\hspace{1in}
       { \color{OliveGreen} \bf grep �600( 31)\#� STEQ\_VALS}
\blankhalf
(note the space before the 3). In general MODEL can be used to find which equation corresponds to which species and super level, and with a little more work, which atomic level. Similarly
\blankhalf
\hspace{1in}
            {  \color{OliveGreen} \bf grep �600(31)\textbackslash*� STEQ\_VALS}
\blankhalf
will list the radiative equilibrium equation which, ideally, should be zero. In practice, the final rate should be $<<$ than the initial value. NB:  The radiative equilibrium equation is set to zero in the STEQ\_VALS array when doing a $\Lambda$--iteration --- this is the reason for the radiative equilibrium equation being printed as a separate vector.

\item[TRANS\_INFO]     Summary of all bound-bound transitions, in increasing wavelength order. The wavelengths are vacuum for  $\lambda < 2000$\,\AA, and in air for $\lambda > 2000$\,\AA. Be careful --- some files list vacuum for $\lambda > 2000$\,\AA.

\item[TOTRATE]     Total rate, $N_u A_{ul} Z_{ul}$, for each bound-bound transition. The statistical equilibrium equations, after the collisional terms are evaluated, are also output. Only needed for
diagnostic purposes, and can be deleted when a model is finalized.

\end{file_list}


\vfill
\eject
\shd{Scratch Files}{scratch_files}

These files are required while a program is running. They should be retained until a MODEL is fully converged.

\begin{file_list}{EDDFACTOR}

\item[POINT1]     ASCII pointer file for scratch file, SCRTEMP. It points to the iteration that will be read in when a model is restarted. In the event of convergence difficulties, it can be edited to point to an old iteration. Say the model has gone crazy at iteration 25. You can return to iteration 23 (or some other iteration number) by simply changing the number above IREC and NITSF to 23. POINT1 is also used by the Ng acceleration routines to indicate when the last  Ng acceleration occurred. The file is updated after each iteration.

\item[POINT2]     Backup copy of POINT1

\item[SCRTEMP]     Main scratch file. It saves all atomic populations for each iteration. Thus only results
for the current iteration are lost with a code or computer crash.  {\color{Red} \bf  SCRTEMP  is used to restart an existing model} and by the  Ng acceleration routine. FREE format, direct access file which may not be portable without modification (although I routinely port it across systems). Two formats are in use. The newer format allows a revised R grid to be stored for each iteration. For new models, this will be transparent to the user. However for some applications (using the DO\_HYDRO option with an existing model) it may be necessary to rewrite the file into the appropriate format using REWRITE\_SCR (or start as a new model). SCRTEMP can be examined using {\color{OliveGreen} \$cmfdist/exe/plt\_scr.exe}. I tend not to remove the SCRTEMP and POINT files when a model is completed, since these make it easy to run additional iterations in the same directory. However, the commands \$cmfdist/exe/rewrite\_scr.exe and \$cmfdist/com/mvscr.sh can be used to generate a smaller version of SCRTEMP containing only the last N (default is 2) iterations.

\item[EDDFACTOR]     File containing the mean intensity J (RJ in program notation) for each frequency at each depth. Originally the file contained the Eddington f values, hence the name. It and ES\_J\_CONV can be plotted using PLT\_JH. Free format, direct access file which is (generally) NOT portable, although I
routinely transfer the file between LINUX systems, and my Intel MAC.. 

\item[ES\_J\_CONV]     Convolution of the mean intensity J with the electron scattering redistribution function. Created when INCOHERENT electron scattering is assumed. Free format, direct access file which may not be portable. 

\interlinepenalty=10000
\item[BAMATPNT]     Pointer and information file for BAMAT. If BAMAT and SCRATCH files are available, this file may be edited by hand so that CMFGEN does not recompute BAMAT etc. Useful when starting a stopped job. Normally, CMFGEN automatically sets the control switch.

\item[BAIONPNT]     Pointer and information file for BAION. This and BAION are no longer used.

\item[BAMAT]     Huge file which contains the LINEARIZED statistical equilibrium equations. CMFGEN uses data stored in this file in when a model is nearly converged -- this reduces the model computation time considerably. Remember to delete the file after a model has completed. FREE format and generally NOT portable.  

\item[?\_INFO]     Files indicating the size of direct access records in files such as EDDFACTOR, ES\_J\_CONV etc. They are needed by PLOT routines such as PLT\_RJ, PLT\_JH etc.

\item[\#SCRATCH\#\#\#]     Scratch file containing BAMAT inversion data. There is one file (DIAG operator) or 3 files (TRIDIAG operator) for each depth. They are used by cmfgen\_dev.exe to facilitate the solution of the large block of simultaneous equations which yield the corrections to the populations. They are only output if [WR\_ PRT\_INV] is set to TRUE in VADAT. At the end of an iteration, the files can be deleted using the {\bf dscratch} command (be careful that you don't have other files containing SCRATCH in their name).

\end{file_list}

\vfill
\eject
\mhd{Explanation of fields in MODEL\_SPEC}{model_spec}

MODEL\_SPEC is the top level control file for CMFGEN, and defines most of the main variables (see page \pageref{sec_main_var}).  With VADAT, it controls the nature and the type of model to be run. All parameters need to be included in this file, and every model must include either H or He.
For clarity, species should be included in order of increasing atomic number, while ionization stages should be included in numerical order. 

For each species the final ionization stage (e.g., H\,{\sc ii}, 
He\,{\sc iii}, C\,{\sc v}) should NOT be listed in MODEL\_SPEC. The single level associated with the final ionization stage is automatically included by CMFGEN.

\begin{mylist}{NLINE\_MAX]}

\item[ND]
This sets the number of depth points in the model atmosphere calculation. In most cases, the grid is computed internally. For W-R stars ND could be as low as 40 although 60 (or higher) is recommended. For O stars, at least 60 points is recommended. When using VEL\_OPT=7 (i.e., reading in a TLUSTY core hydrostatic structure) ND must be identical in MODEL\_SPEC \& RVSIG\_COL. For O stars with low mass-loss rates, ND should be larger than 60 since you need at least 5 points per decade of $\tau$ in the photosphere, and at least 30 points in the wind. 

\item[NC]    Number of rays striking the core. Used for the angle integration. I typically use 10 to 15.

\item[NP]   Total number of rays. This is generally ND+NC. For some old options this must be ND+NC$-2$.

\item[NUM\_BNDS]   Number of bands to be used for the linearization matrix. Choose 1 for a DIAGONAL operator and 3 for a TRIDIAGONAL operator.

\item[NCF\_MAX]    Maximum number of continuum frequencies. In BLANKETING mode, these frequencies also include lines. If NCF\_MAX is too small CMFGEN will halt, and an error will be output to OUTGEN.

\item[MAX\_SIM]    The maximum number of lines whose profiles overlap. Generally set to 70  for smallish models--- models with larger values may take longer to compute. It must be larger than the maximum number of lines whose Sobolev resonance zones (i.e. intrinsic line profiles) simultaneously overlap.  If MAX\_SIM is too small CMFGEN will halt, and an error will be output to OUTGEN.

\item[NLINE\_MAX]    The maximum number of bound-bound transitions. In the original version, CMFGEN would halt if NLINE\_MAX was too small. In CMFGEN\_DEV, \break
NLINE\_MAX is computed internally, and [NLINE\_MAX] in MODEL\_SPEC is obsolete.

\item[NLF]    Number of frequencies to be used across the Doppler core. Not used in BLANKETING mode --- only by the CMF section.

\interlinepenalty=10000
\item[XzV\_ISF]    It indicates the level structure to be used for the atomic species XzV for CMFGEN\_DEV. It requires 3 parameters and has the format
      10,20,30
where
   10 is the number of important levels (= NXzV\_IV $\le$NXzV),
   20 is the number of super levels (= NXzV $\le$ NXzV\_F) in the  model atom, and
   30 is the number of atomic levels (=NXzV\_F) in the model  atom.
    For H, He, CNO, and perhaps Fe, it is a good idea to have the number of important levels identical to the number of super levels. For true impurity species, set number of important levels to 0.

\optdesc{[XzV\_NSF]}    It is an obsolete reference used in MODEL\_SPEC with an older version of cmfgen. If present in an old model file it should be updated with [XzV\_ISF]. [XzV\_NSF] indicates the level structure to be used for the atomic species XzV for CMFGEN. It has two parameters and has the format
      20,30
where
   20 is the number of super levels (= NXzV $\le$ NXzV\_F) in the model atom, and
   30 is the number of atomic levels (=NXzV\_F) in the model atom. \\
   \\
   If a weird error occurs when running CMFGEN, and/or it stops immediately, check that the usage of XzV\_ISF (and XzV\_NSF) is correct. 

\item[FL\_OPT]
Optional keyword to automatically change the number of full levels in all species (whan absent, NXzV\_IF is not changed). This, and the next two keywords were introduced to allow rapid testing of the influence of  the model atoms on model \linebreak results without the need to physically edit NXzV\_F etc in MODEL\_SPEC, and change the atomic data assignments in batch.sh. Other options could be added to \linebreak \$cmfdist/subs/fdg\_f\_to\_s\_ns\_v1.f. When [FL\_OPT] is set, NxZV\_S is reset to that required by the assigned F\_TO\_S file -- this might be adjusted if the [SL\_OPT] is also set.


\begin{plt_list}{DO\_ALL\_LEVS --}
\item[DO\_ALL\_LEVS --]  Sets NXzV\_F to use all levels for each ion.
\item[SET\_TO\_N --]  Sets NXzVF to Min(N,NXzV\_F). NB: N should be replaced a numeric value, e.g., 100.
\end{plt_list}

\item[SL\_OPT]
Optional keyword to automatically change the number of super levels in all species.
\begin{plt_list}{DO\_ALL\_LEVS --}
\item[SPLIT\_LOW\_N --]  Splits the lower N super levels into individual levels. NB: N should be replaced a numeric value, e.g., 10.
\end{plt_list}


\item[IL\_OPT]
Optional keyword to automatically change the number of impurity levels in all species.
Must be present when [FL\_OPT] or [SL\_OPT] is present. 

\begin{plt_list}{DO\_ALL\_LEVS --}

\item[USE\_ALL\_SL --]  Sets NXzV\_IV to NXzV\_S.

\item[SET\_TO\_N --]  Sets NXzV\_IV to Min(N,NXzV\_S).

\item[UPDATE --] Sets NXzV\_S to Min(NXzV\_S, NXzV\_IV + (NXzV\_S-NXzV\_S(old)])

\item[NOCHANGE --] NXzV\_IV remains as the value read in from MODEL\_SPEC.

\end{plt_list}

\end{mylist}

\vfill
\eject
\mhd{Explanation of options in VADAT}{vadat_opts}

VADAT is the main driver file for CMFGEN. Some parameters need to be included even when they are not utilized, while other parameters need only be included when another specific parameter has been set to a specific value. For example, the parameters for the velocity law change depending on the choice of [VEL\_LAW].

All keywords in the following text are specified in bold between square brackets. Some parameters are checked for validity --- others are not. It is the responsibility of the user to ensure that the parameters are valid. Check the MODEL file after a job begins.

Keywords need not be in order, although it is recommended that the ordering of keywords not be changed dramatically from that provided. If a keyword can't be found, an error is output to OUTGEN, and program execution stops. Superfluous keywords (e.g., FIX\_CIII when C\,{\sc iii} is not included) are ignored.

Additional keywords are continually added to improve the accuracy, convergence properties, and applicability of the code. This will necessitate that the VADAT file of old models be revised, if you wish to re-run the same model. This generally presents no difficulty, however, since the options are unique, and the code will inform you through the OUTGEN file when they are not present. 

Options in VADAT are read in via the subroutine 
\blankhalf
 \hbox{ } \hspace{0.5in}  new\_main/mod\_subs/rd\_control\_variables.f. 
\blankhalf
This routine can be checked for option ordering, and additional info.  For logical variables, F can be used instead of FALSE, and T for TRUE. Also please use spaces, rather than a tab, to separate values from their associated keyword.


\blankline
\blankline
\shd{Options for Atmospheric Structure}{atm_struct}

\vspace{-10pt}
\begin{mylist}{longest label}

\item[RSTAR]    Radius of star (i.e. inner boundary of model) in program units of $10^{10}$\,cm. In these units the radius of the Sun is 6.96. The optical depth at [RSTAR] is velocity and mass-loss rate dependent, and can be found using DISPGEN. The user should check that the code extends sufficiently deep  (i.e., \tauross=10 to 100) so that LTE is recovered. Going unnecessarily to very large $\tau$ (e.g., $\tauross > 1000$) may cause convergence difficulties.  \tauross(max) is now output to CMFGEN in the first iteration when the [LIN\_INT] option is false.

\item[RMAX]    Outer radius of star in units of [RSTAR]. For low density atmospheres [RMAX]=100 is adequate, while for W-R stars  [RMAX]=200 is suggested. The strength of some lines (e.g., He\,{\sc i} 5876, 10830) may be weakly affected by the choice of [RMAX]. The strength of forbidden lines formed at low densities may be substantially affected.

\item[DO\_HYDRO]    Indicates that the density structure in the photosphere is to be iterated on so as to better satisfy the hydrostatic equation. If TRUE, HYDRO\_DEFAULTS must be present as a separate file, and [VEL\_LAW] must be 7. In this case both [VINF] and [BETA] must also be given in VADAT. When set, R and V are written to SCRTEMP for each iteration.
{\color{red} NB: When set, you must also edit  HYDRO\_DEFAULTS to specify the number of iterations.}

\vfill
\eject
\item[VEL\_LAW]    Indicates which velocity law to adopt. New velocity laws can be added to the code by utilizing this option. Each velocity law option has its own set of compulsory keywords.

{ \hfill  \color{Brown}  \bf [VEL\_LAW]=3 \hfill } \\
Standard $\beta$-velocity modified at depth so that it approaches a hydrostatic structure. The velocity law has the form:
  \begin{equation}
     v(r)= [ \Vo + (V_\infty-Vo)(1-R_*/r)^\beta ] / [ 1+\Vo/\Vcore exp([\Rstar-r]/\Psub{h}{eff}) ]
  \end{equation}

\begin{myinlist}{longest label}
\item[VCORE]    Velocity in km\,s$^{-1}$ at \Rstar\ when \Vo$ >> $\Vcore. Typically less than 1\,km\,s$^{-1}$.

\item[VPHOT]    Generally referred to as the photospheric velocity, but its exact meaning depends on the values adopted for the other parameters such as \Vo\ (in km\,s$^{-1}$).

\item[VINF]    Terminal velocity of the flow in km\,s$^{-1}$.

\item[SCL\_HT]    Scale height (heff) of photosphere in units of RSTAR. Specifies density structure at low velocities. Can be used to specify a velocity structure which is approximately hydrostatic.

\item[BETA]    Speed of velocity law. This is $\beta$ in the usual $\beta$-velocity law. Typical values are in the range 0.5 to 4.
\end{myinlist}

{ \hfill  \color{Brown} \bf [VEL\_LAW]=4 \hfill}\\
    Reads in R,V \& SIGMA from a file (simple column format). No options. \\

{ \hfill  \color{Brown} \bf  [VEL\_LAW]=6 \hfill} \\
    $\beta$-velocity modified at depth so that it approaches a hydrostatic structure. In the outer regions it is modified so that it can have an extended acceleration zone. This velocity law is an extension of velocity law 3.  The TSTV option in DISPGEN can be used to examine the shape of the velocity law for different parameter combinations. The velocity law has the form:
\begin{equation}
  v(r)={ { \Vo + (V_{\infty}-\Psub{V}{ext}-\Vo)(1-\Rstar/r)^{\beta_1} + \Psub{V}{ext}(1-\Rstar/r)^{\beta_2}  } \over 
                    {(1+\Vo/\Vcore \, \exp([\Rstar-r]/\Psub{h}{eff}) }  }
\end{equation}

\begin{myinlist}{longest label}
\item[VCORE]    Velocity in km\,s$^{-1}$ at R$_*$. Typically less than 1\,km\,s$^{-1}$.

\item[VPHOT]    Photospheric velocity (\Vo) in km\,s$^{-1}$.

\item[VINF1]    Terminal velocity of flow in km\,s$^{-1}$, of the first component of the velocity law. For the above, [VINF1]$=V_\infty-$\Psub{V}{ext} and [VINF2]=[VINF1]$ + $\Psub{V}{ext} 

\item[SCL\_HT]    Scale height (heff) of photosphere in units of RSTAR. Specifies density structure at low velocities. Can be used to specify a velocity which is approximately hydrostatic.

\item[BETA1]    Speed of velocity law. This is $\beta$ in the usual $\beta$-velocity law. Typical values are in the range 0.5 to 4.

\item[EPPS1]   Needed when $\beta < 1$. Chosen so that $(1-\epsilon \Rstar/r)$ is greater than zero at the core, and hence gives a finite dv/dr.

\item[VINF2]     Terminal velocity of flow in km\,s$^{-1}$. If [VINF2]=[VINF1], the second component of the velocity law has no effect.

\item[BETA2]    Speed of the second component of the velocity law. This is $\beta$ in the usual $\beta$-velocity law. Typical values are in the range 5 to 50.

\item[EPPS2]   Needed when $\beta_2 < 1$. Chosen so that $(1-\epsilon \Rstar/r)$ is greater than zero at the core, and hence give a finite dv/dr.
\end{myinlist}

{ \hfill  \color{Brown}  \bf [VEL\_LAW]=7  \hfill}\\
  R, V and SIGMA (= dlnV/dlnr-1) are read in from a file. \Psub{R}{max}/\Rstar\ should agree with the value specified in the VADAT file. \Rstar is scaled to agree. Presently no interpolation is done, and the R values read in define the adopted radius grid. For most models, a simple
  $\beta$-velocity law is appended to the hydrostatic structure. A more complicated velocity law can be adopted by setting the key word
  [BETA2] in HYDRO\_DEFAULTS.
 
\begin{myinlist}{longest label}
\item[VEL\_OPTION]
    Indicates from which file velocity data is to be read. Two options available: RVSIG\_COL (simple column format) and deKOTER (file in Alex deKoter's format). See newsubs/rd\_rv\_file\_v2.f for further details. A new RVSIG\_COL can be created from an old RVSIG\_COL file using \linebreak \$cmfdist/exe/rev\_rvsig.exe. Alternatively a new RVSIG\_COL  file can be created using  \$cmfdist/exe/tlusty\_vel.exe or \$cmfdist/exe/wind\_hyd.exe.

\item[BETA]    Speed of velocity law. That is $\beta$ in the usual $\beta$-velocity law. Typical values are in the range 0.5 to 4.

\item[VINF]    Terminal velocity, in km\,s$^{-1}$, of the wind. Should be similar to value in file.

\item[VAR\_MDOT]   Logical variable indicating whether the mass-loss rate varies with time, and hence radius. If true, the density structure must be supplied. Used for modeling LBVs, such as AG Car, and for treating winds with a specified density (usually clumped) structure. 

\item[VM\_FILE]    File with density and clumping info for model with variable MDOT. Same format
as RV\_SIG file except it has two extra columns -- the density and the clumping factor. This file could
also be used to set R, V,  and SIGMA (i.e., utilized as an RVSIG\_COL file)

\end{myinlist}
 
{ \hfill  \color{Brown} \bf [VEL\_LAW]=10 \hfill} \\
   Velocity law for SN models. A power law in r (generally a Hubble law) is assumed. 

\begin{myinlist}{longest label}
\item[VCORE]    Velocity at inner core (km\,s$^{-1}$).

\item[BETA1]    Exponent for velocity law. Use 1 for a Hubble Law.

\item[RCUB\_RHO]    Density(core radius)$^3$. The density is in gm/cm$^3$, while the core radius is assumed to be in program units (i.e., 10$^{10}$ cm).

\item[N\_RHO]    Exponent for power law variation of density. For SN, N\_RHO should be of order 10.
\end{myinlist}

{ \hfill  \color{Brown} \bf [VEL\_LAW]=11 \hfill}  \\
  For SN models using a hydrodynamic structure.
 
\begin{myinlist}{longest label}
\item[VINF]    Required terminal velocity, in km\,s$^{-1}$, of wind (at infinity).
\end{myinlist}

\item[MDOT]    Mass-loss rate in units of \Msun/yr.

\item[LSTAR]    Total luminosity of star in units of \Lsun. This is a fundamental parameter of the model.

\item[TEFF]    Effective temperature (in units of 10$^4$\,K) of star. Only used when hydrostatic structure is updated. For a spherical star, [TEFF] is specified at $\tauross=2/3$. For a plane-parallel model, RSTAR is used to define TEFF. {\color{red} \bf [TEFF] only has an effect when the hydrostatic structure is updated -- models in which this is not done will generally have inconsistent luminosities, radii, and [TEFF]. When starting the model do not forget to edit HYDRO\_DEFAULTS  to indicate the number of hydrostatic iterations to be performed.}

\item[LOGG]    Log of surface gravity (cgs units). Only used when hydrostatic structure is updated. For a spherical star, $\log g$ is specified at $\tauross=2/3$. For a plane-parallel model, RSTAR is used to define $\log g$.

\item[MASS]    Mass of star in units of \Msun. Presently this parameter is only used for output to HYDRO. If the hydrostatic structure is updated, it will be automatically be edited in VADAT by CMFGEN to be consistent with log g and R(Teff).

\item[DO\_CL]    Switch to turn clumping on/off. A simple filling factor approach is used. Clumping currently ONLY works in BLANKETING mode. Incorrect results will be obtained if clumping is used in CMF or SOB mode, or if dielectronic lines are included as pure lines. In order to generate a FLUX spectrum, the code does not halt when [FLUX\_CAL\_ONLY] is specified with [SOB] for some (or ALL) lines.

\item[CL\_LAW]    Character string which specifies how to evaluate the clumping factors. Currently two laws are implemented:
EXPO and REXP. No effect if [DO\_CL]=FALSE.

EXPO: \hspace{10pt}  $f(r)=f_1+(1-f_1-f_3)\exp(-V(r)/f_2)+f_3 \exp(-V(r)/f_4)$ 

REXP:  \hspace{10pt} $f(r)=f_1+(1-f_1) \exp(-V(r)/f_2)+(1-f_1) \exp( [V(r)-V{\rm max}]/f_3)$

\blankhalf
In the above $f_1 = $[CL\_PAR\_1], $f_2 = $[CL\_PAR\_2 ]etc. For  the [EXPO] law, $f_4=1$ if $f_3=0$. 


\item[N\_CL\_PAR]    Number of clumping parameters. For the [CL\_LAW]=EXPO, this value should be 2. No effect if [DO\_CL]=FALSE.

\item[CL\_PAR\_1]    Clumping parameter (i.e., volume filling factor) at $V_\infty$. I typically choose 0.1. To preserve the same spectrum as an unclumped model, the mass-loss rate should be multiplied (and hence reduced) by a factor (\hbox{\rm [CL\_PAR\_1]})$^{0.5}$. No effect if [DO\_CL]=FALSE.

\item[CL\_PAR\_2]    Indicates how rapidly clumping should be switched on. It is assumed to be damped at low velocities. Parameter is specified in km\,s$^{-1}$. Typically we adopt a few hundred km\,s$^{-1}$ for a W-R star. No effect if [DO\_CL]=FALSE.

\item[DUM/X]    When $+$ve, it gives the abundance of species DUM, by number, relative to some arbitrary species X. This mode tends to be most useful for species modified by nuclear processes (e.g., H, He, C). When $-$ve, it specifies the mass fraction of the species. This mode is useful for species whose mass fraction is not affected by evolution (e.g., Fe). The two modes may be mixed. DUM should be replaced by HE, CARB, NIT etc. i.e., the DUM variables specified in the MODEL\_SPEC file, and in CMFGEN. 
\end{mylist}


\shd{Options for continuum frequency grid}{cont_freq_grid}

\begin{mylist}{longest label}

\item[RD\_CF\_FILE]    Read in continuum frequencies from file, perhaps generated by an earlier model. Option rarely utilized since blanketing was installed.

\item[MIN\_CF]    Minimum continuum frequency (if calculating) in program units of $10^{15}$\,Hz. Bound-bound transitions at lower frequencies are neglected in the blanketing calculations. Example, for W-R star use: $3.49897\times 10^{-3}$.

\item[MAX\_CF]    Maximum continuum frequency (if calculating) in program units of $10^{15}$\,Hz. This value should be 1.5 or more times the highest ionization edge of all included species. For W-R stars typically adopt a value of 50 to 100. A much higher value (1000) needs to be adopted when X-rays are included.

\item[FRAC\_SP]    Fractional spacing for small frequencies. Typically adopt 1.1. Thus \\
\hbox{} \hspace {3cm}  $\nu_{i+1} = \nu_i$ / [FRAC\_SP].


\item[AMP\_FAC]    Amplification factor for large frequency ranges --- typically adopt 1.05. Thus \\
\hbox{} \hspace {3cm} $\nu_{i+1} = \nu_i \times $[AMP\_FAC].

\item[MAX\_BF]    Maximum frequency spacing in program units of $10^{15}$\,Hz close to bound-free edges for frequencies $> 10^{15}$\,Hz. Typically we adopt 0.10. Too large a value will influence the accuracy with which recombination rates are computed.

\item[FR\_GRID]. When non-zero, it allows earlier techniques to be used to compute the frequency grid. When specified, 0 uses the latest grid option (default).

\item[DO\_DIS]    Allows for level dissolution. This leads to better behavior near bound-free edges, and is more realistic. Usually set to TRUE.

\item[dV\_LEV]    Spacing in km\,s$^{-1}$ on low side of bound-free edge. This option is to insure good frequency spacing when level dissolution is in effect. Typically adopt 200.0 km\,s$^{-1}$.

\item[AMP\_DIS]    Amplification factor on low side of bound-free edge. As we move away from the bound-free edge (to lower frequencies) the frequency spacing increases by [AMP\_DIS] until next continuum frequency is reached (Typically I use 1.4).

\item[MIN\_DIS]    Minimum frequency in program units of $10^{15}\,$Hz for level dissolution. Below this frequency we make no special allowance for level dissolution in choosing the continuum frequencies. Use 0.1.

\item[CROSS]    If TRUE, continuum cross-sections are evaluated at all frequencies. If FALSE, continuum cross-sections are evaluated at the pure continuum frequencies, and additional frequencies as determined by [V\_CROSS]. The FALSE option can improve the speed of the code by a factor of 3 (or more). Tests indicate that setting [CROSS]=FALSE does not generally affect the accuracy of the model calculation. FALSE is the preferred setting.

\item[V\_CROSS]    Maximum separation, in km\,s$^{-1}$, between evaluations of the continuum opacity and emissivity. Typically adopt 750.0 km\,s$^{-1}$. This value should be compatible with the smoothing adopted for the photoionization cross-sections and [SIG\_GAU\_KMS] -- i.e.,, it should be at least a factor of 2 smaller. It is also used in defining the frequency grid.

\item[SIG\_GAU\_KMS]    Sigma of Gaussian (in km\,s$^{-1}$) used to smooth the photoionization cross-sections. Some unsmoothed, or low smoothed, photoionization files are available. Eventually all the photoionization data will be unsmoothed so as to provide greater flexibility. If the data has already been smoothed with a resolution of 3000 km\,s$^{-1}$, this option will not have any effect if [SIG\_GAU\_KMS] is set below 3000 km\,s$^{-1}$, since [SIG\_GAU\_KMS] in CMFGEN is adjusted for the existing smoothing for each ion.

\item[FRAC\_SIC\_GAU]    Hidden parameter which indicates the spacing (in sigma) used for the numerical smoothing. Default is 0.25.

\item[CUT\_ACCURACY]    Hidden parameter: After smoothing, unnecessary grid points in the photoionization cross-sections are deleted. Points are only deleted when the cross-section can be interpolated, using linear interpolation, from neighboring points with a fractional accuracy of [CUT\_ACCURACY]. The default value is 0.02, which is smaller than the accuracy of the photoionization cross-sections. 

\item[ABV\_EDGE]    (Hidden). Uses only data above photoionization edge when smoothing. Default is TRUE.


\item[EXT\_LINE\_VAR]    Extent of variation region beyond resonance zone measured in terms of $V_\infty$. I typically adopt 0.5.

\item[ZNET\_VAR\_LIM]    Iterate on net rates at those depths where ABS(ZNET-1)$ <$ ZNET\_VAR\_LIM. Typically set to 0.01. When $|$ZNET-1$|$ is small, the line is essentially optically thin.

\item[WNET]    If TRUE, we iterate on the net rates (rather than use full linearization) for weak lines as set by ZNET\_VAR\_LIM.

\item[WK\_LIM]    Used as a control parameter for WNET. When set to 0.1, lines with maximum line opacity to continuum opacity of $< 0.1$ are treated using net rates.
\end{mylist}

\shd{Computation of radiation field}{comp_rad_field}

\begin{mylist}{longest label}

\item[PP\_NOV]    Hidden logical variable: When true CMFGEN uses a plane-parallel geometry WITHOUT a velocity field. Default is FALSE (i.e., use a spherical model).

\item[PP\_MOD]    Hidden logical variable: When true, CMFGEN, uses a plane-parallel geometry WITH a velocity field. Default is FALSE (i.e., use a spherical model). 

\item[INCL\_DJDT]    Hidden logical variable: Includes DJDt terms in transfer equation for SN models. For Hubble flow only. Default is FALSE. When running a SN time-sequence,  
[INCL\_DJDT], [USE\_DJDT\_RTE] and [USE\_FRM\_REL] should all beTRUE,

\item[USE\_DJDT\_RTE]    Hidden logical variable: Allows the user to force the DJDt transfer routines to be used, even when the DJDt terms are not to be included. Useful for testing, and can also be used for the first model of a time sequence.

\item[DJDT\_RELAX]    Hidden logical variable: Relaxation parameter to scale DJDt terms to assist initial convergence. Default is 1 (i.e., no relaxation.). Parameter was used for initial model development.

\item[USE\_J\_REL]    Hidden logical variable: Uses MOM\_J\_REL\_VN to solve the momemt transfer equations in the comoving-frame. All relativistic terms may be included. Default is FALSE. 

\item[USE\_FRM\_REL]  When true, the fully relativistic ray routine is used to compute the
Eddington factors. The default value is TRUE is either [USE\_J\_REL] or [USE\_DJDT\_RTE] is true, otherwise it is false.

\item[INC\_REL]    Hidden logical variable: Includes relativistic terms in the transfer equation when USE\_J\_REL is TRUE. Default is FALSE.

\item[INC\_ADV\_TRANS]    Includes the advection terms in the transfer equation. Default is generally FALSE, but it is TRUE when both USE\_J\_REL and INC\_REL are TRUE.
 
\item[USE\_FIXED\_J]    Logical variable that tells CMFGEN to use a previously computed J as stored in EDDFACTOR. This can only be TRUE when doing a LAMBDA iteration, and the EDDFACTOR file must exist. It should be set to TRUE when installing new-species, since this allows populations that are consistent with the radiation field to be rapidly computed.  USE\_FIXED\_J is also used in CMF\_FLUX to save computational effort (where you use an old J to provide an estimate for the electron scattering source function).

\item[JC\_W\_EDD]    Use Eddington factors to compute JBAR for lines. For use with the CMF option. Set to TRUE. Not utilized in pure SOB or pure BLANKETING modes. Alternative modes, which may still work, are obsolete and are slower.

\item[NOV\_CONT]    In non-blanketed mode, ignores velocity terms when computing continuum.

\item[JBAR\_W\_EDD]    Computes mean line continuum intensity using Eddington factors. Only used with CMF option (non-blanketed mode).

\item[DIF]    Indicates whether the diffusion approximation is to be used at the inner boundary. An alternative for stellar work is a Schuster-like boundary condition that hasn't been utilized for ages, and may no longer work.

\item[OB\_METH] Option to specify the differencing of the outer boundary condition. Used mainly for development work.
\begin{plt_list}{HOLLOWCORE --}
\item
[HONJ]  Default (iterate on H/J)
\item
[HALF\_MOM] Uses half moments to handle the outer boundary condition.
\end{plt_list}
  
\item[IB\_METH]  Optional keyword that supersedes [DIF], but bust be consistent with [DIF] if both are present. Primarily for use with SN models.
\begin{plt_list}{HOLLOWCORE --}
\item
[DIFFUSION] Equivalent too DIF=T
\item
[ZERO\_FLUX] Sets the flux to zero at the inner boundary. For the moment solution, this is (almost) equivalent to the diffusion approximation with an inner boundary luminosity of zero.
However for the ray-ray solution, we set $I^+ = I^-$ (at the inner boundary) compared with $I^+=B(T)$. Thus the ZERO\_FLUX option will work when the core becomes transparent.
\item
[HOLLOW\_CORE]  Assumes the core is hollow and allows for the velocity shift between
the near and far sides of the core. This option may be convergence issues since this may cause large changes in some populations close to the inner boundary necessitating a finer grid. This is similar to the ZERO\_FLUX option when V(ND)=0.
\end{plt_list}

\item[COH\_ES]    Switch for coherent electron scattering. For best convergence set key to TRUE. For flux calculations, with CMF\_FLUX, set to FALSE. Work on improving the convergence in NON-COHERENT mode is ongoing. To test sensitivity of CMFGEN model  
to this option, first converge model with [COH\_ES]=T. Then run a new model with [COH\_ES]=F. Most likely will converge, although issues can occur with the the temperature correction.
A partial test can be done by setting [FIX\_T]=T.

\item[OLD\_J]    Obsolete: Use old J to provide estimate for J\_ES.

\item[MIX\_COH]    Mix coherencies in variation of J. Tells CMFGEN to assume that the electron scattering is coherent at some depths, as determined by [ES\_FAC], when computing the variation of J. Under development.

\item[ES\_FAC]    Determines how close RJ (mean intensity) and RJ\_ES (scattered mean intensity) have to be to use the coherent approximation when computing the variation of J. Use [ES\_FAC]=0.1. Under development.
 
\pagebreak
\item[N\_TYPE]    In the computation of J using the moment equations the moment N (in the moment set J, H, K, N) must be specified. This could be done using the Eddington factor N/H but this is unstable as H may go through zero.
\begin{plt_list}{GG\_ONLY}
\item
[N\_ON\_J] Use the Eddington factor N/J at all depths. Preferred option in CMFGEN.

\item
[MIXED]Use the Eddington factor N/H except at those depths where it is undefined (e.g., where H is close to zero). At those depths, N/J is used.

\item
[G\_ONLY] Use the Eddington factor g = N/H at all depths. In blanketed models this option can CRASH the code. I tend to use this option in CMF\_FLUX.
\end{plt_list}


\item[FG\_OPT]    Determines the method used to solve the radiative transfer equation (RTE) in the (p,z) coordinate system. The solution of the RTE is used to compute the Eddington factors f and g needed to solve the moment equations.

\begin{plt_list}{GG\_ONLY}
\item
[INT/INS] Preferred setting. Uses the integral short characteristic approach (INT) to solve the RTE. `/INS' indicates that extra points are inserted near z=0. It may be omitted. `INT/INS' has greater stability than `DIFF/INS' since the intensity, I, is always guaranteed to be greater than zero.

\item
[DIFF/INS] Uses the difference approach to solve the RTE. Original approach. Generally fairly stable but can get problems en route to a converged model. 
\end{plt_list}
\end{mylist}

\shd{Accuracy options}{act_opt}

\begin{mylist}{longest label}
 \item[METHOD]    Optical depths in CMFGEN are estimated by the trapezoidal rule with a correction based on the first derivatives using the Euler-McLaurin summation rule. Method determines how the derivatives are estimated. .

\begin{plt_list}{LOGMON}

\item
[ZERO]
 Indicates that the derivatives are zero, and hence the optical depths are to be estimated using a pure trapezoidal rule. Can be a useful option if model is suffering convergence difficulties because of an ionization front.
\item
[LOGMON]
Preferred option. Optical depths estimated by the trapezoidal rule with a correction based on the first derivatives using the Euler-McLaurin summation rule. The derivatives are estimated using a cubic in log-log space that is forced to be monotonic over each integration interval.

\item
[LOGLOG]
Optical depths estimated by the trapezoidal rule with a correction based on the first derivatives using the Euler-McLaurin summation rule. The derivatives are estimated using the stable approximation
                         
\begin{equation}
\left({d\chi \over dr} \right)_i = \left({\chi_i \over r_i}\right) \left( {\ln \chi_{i-1} - \ln \chi_{i+1} \over \ln r_{i-1}- \ln r_{i+1} } \right)
\end{equation}

\item
[LOGLIN]
As for LOGLOG except derivatives estimated using
                         
\begin{equation}
\left({d\chi \over dr} \right)_i = \chi_i \left( {\ln \chi_{i-1} - \ln \chi_{i+1} \over (r_{i-1}  - r_{i+1} }\right)
\end{equation}

\item
[LINLIN]
As for LOGLOG except derivatives estimated using
                         
\begin{equation}
\left( {d\chi \over dr}\right)_i = { \chi_{i-1} - \chi_{i+1} \over r_{i-1}  - r_{i+1} }
\end{equation}

\end{plt_list}

 \item[VFRAC\_FG]    Used to facilitate the inclusion of extra points in FG\_J\_CMF\_V11 --- the default procedure used to compute the Eddington factors. FG\_J\_CMF\_V11 solves for J \& K using the ray method. This option was installed to overcome numerical difficulties, and is primarily for CMF\_FLUX. If set to {\bf m} (for example), extra points are inserted along each ray to ensure that the velocity step size is less than  {\bf m} local Doppler widths. In CMFGEN I usually omit by setting to a large value (e.g., 2000). This option can be used with the current version of cmfgen\_dev.exe.

\item[VFRAC\_MOM]    Used to facilitate the inclusion of extra points in MOM\_J\_CMF\_V6. MOM\_J\_CMF\_V6 solves for J \& K using Eddington factors. This option was installed to overcome numerical difficulties, and is primarily for CMF\_FLUX. If set to m (for example), extra points are inserted along the radius grid to ensure that the velocity step size is less than m local Doppler widths. In CMFGEN I usually omit by setting to a large value (e.g., 2000). It is unclear whether convergence would be obtained with the current version of cmfgen\_dev.exe, since the linearization section presently does not include extra points.

\item[THK\_CONT]    Switch to utilize the thick boundary condition for the computation of the continuum. In CMF mode, this option only applies to the true continuum, where as in blanketing mode, where there is no ``distinction" between lines and continuua, it applies to all frequencies. In blanketing mode the atmosphere is extended by extrapolation. In other modes a crude approximation is utilized. Preferred default is TRUE --- this generally gives a smooth and more realistic run of properties at the outer boundary. Do {\color{red} NOT} confuse this option with [THK\_LINE]-- that option is only refers to lines when using the CMF mode.

\item
[REXT\_FAC]  -- Factor to extend outer radius for a thick continuum model. Default is
 model dependent.
 
\item[TRAP\_J]    Method to use for the computation of weights for the calculation of the moments J, H, K, and N. [TRAP\_J]=T is the preferred option. This indicates to assume a linear approximation for the specific intensity I between each node. For J this is equivalent to a trapezoidal rule. For other moments the appropriate weight $(\mu, \mu^2, \mu^3)$ is specifically taken into account when computing the weight.

\item[OBC\_TYPE]    Hidden logical variable: Outer boundary condition type -- used by me for development work. Default is 1.

\item[BC\_PAR1]    Hidden logical variable: Frequency (10$^{15}$\,Hz) below which new boundary condition is adopted -- used by me for development work. 

\item[INCID\_RAD]    Hidden logical variable. Include incident radiation on outer boundary (for use with plane-parallel models).
\end{mylist}

\shd{Line options}{line_opts}

\begin{mylist}{longest label}

\item[FIX\_DOP] Default is TRUE. Indicates the Doppler profiles of fixed width should be
used when computing intrinsic line-broadnening profiles.

\item[TDOP]    Presently the line absorption profile is assumed to be a DOPPLER profile that is independent of depth and the atomic species. The width of the Doppler profile depends on 3 parameters --- the adopted electron temperature [TDOP] in program units of 10$^4$\,K, the adopted atomic mass  [AMASS\_DOP] in a.m.u., and the turbulent velocity [VTURB] in \kms. One day  I will update CMFGEN to use variable Doppler widths etc.

\item[AMASS\_DOP]    Atomic mass (in a.m.u) used to compute the Doppler profile. At present this is the same for all species at all depths. The best approach is probably to set AMASS\_DOP=1.0D+40. Then use VTURB to control the width of the line absorption profile when [FIX\_DOP]=TRUE.

\item[VTURB]    Turbulent velocity, in km\,s$^{-1}$, used to compute the line absorption profile. At present this is assumed to be depth independent, and the same for all species. Only used when
[FIX\_DOP]=TRUE.

\item[VTURB\_MIN] - Minimum turbulent velocity (only used when [FIX\_DOP]=FALSE).

\item[VTURB\_MAX] - Maximum turbulent velocity (only used when [FIX\_DOP]=FALSE). A linear interpolation in V(r) is used to determine the turbulent velocity at other radii.

\item[GLOBAL\_PROF]    Indicates method for determining intrinsic line absorption profiles when
[FIX\_DOP]=FALSE. Options are:
    
\begin{plt_list}{LIST\_VOIGT}

\item[NONE]    Options for each species determined by [PROF\_XzV]. 
\item[DOP\_FIX] Fixed Doppler width for all species. 
\item[DOP\_SPEC] Species dependent Doppler width but depth independent.
\item[DOPPLER]    Species and depth dependent variable Doppler width.
\item[HZ\_STARK]    Stark profile (convolved with Doppler profile) for H I and He II. 
\item[LIST]    Profile options for individual lines are specified in the file FULL\_STRK\_LIST.
\item[LIST\_VOIGT]  Use the option in FULL\_STRK\_LIST when available. Uses a Doppler profile for �weak� lines. Uses VOIGT profile with radiative damping for strong lines.

\end{plt_list}

\item
[OPAC\_LIMS]  Set prof limits by line to cont. ratio?'

\item 
[DOP\_PROF\_LIMIT] Edge limits for Doppler line profile when [OPAC\_LIMS]=TRUE.
Given as a ratio of line opacity to continuum opacity.
           
\item
[VOIGT\_PROF\_LIMIT] Edge limits for Voigt line profile when [OPAC\_LIMS]=TRUE.
Given as a ratio of line opacity to continuum opacity.

\item
[MAX\_ED\_PROF] Sets the maximum electron density that will be used in the evaluation of Stark profiles
computed via the Griem routine. This is useful, for example, to limit the size of the profile at depth, especially in CMFGEN calculations. NB: Current tabulated stark profiles often have hard wired electron density limits of $10^{17}$\,cm$^{-3}$. Further, there are issues with He\,{\sc i} 4388 above an electron density of $10^{16}$\,cm$^{-3}$.

\item
[V\_PROF\_LIM] Sets the the profile limits (=1/2 full width) of the intrinsic line profile. Profile is limited to the range set by [V\_PROF\_LIM], and the integration for rates in integrated over this limit. Does not effect Stark profile,
and it is superseded by any limit in file listing line profiles. For CMF\_FLUX\_V5 the default is 5000\,\kms, while for CMFGEN it is 3000\,\kms. Increasing this limit will often remove (or decrease) errors in OUTGEN or OUT\_FLUX
regarding profile normalization at high electron densities.

\item
[NORM\_PROF]: Indicates that the profile (non-Doppler) area should be normalized to unity. Essential for CMFGEN, and is the default option in both CMFGEN and CMF\_FLUX\_V5.

\item[MAX\_DOP]    Maximum half-width of resonance zone in Doppler widths. Because of the possibility of very strong lines, a value of 6 is used. For weak lines a smaller value could be utilized, but at present all lines use the same value (when FIX\_DOP=T).
    
\item[FRAC\_DOP]    Spacing in CMF resonance zone in Doppler widths. For statistical equilibrium calculations, particularly model grid calculations, use 1.0. Because of numerical instabilities, a smaller value may not give increased accuracy. The presence of instabilities depends on both the spatial and frequency grid scales, and the adopted Doppler (thermal + turbulent) velocities. We are trying to develop automatic methods to better handle the instabilities.  For the observed flux calculations use 0.5, together with VFRAC\_FG and VFRAC\_MOM set to 1.  This is generally more stringent than needed, but it does give more accurate profiles with less bleeding to the red (for OBSFLUX). Note that VFRAC\_FG and VFRAC\_MOM only directly affect the calculation in the comoving-frame. In the observer's frame it only affects the computed spectrum via the electron-scattering emissivity.

\item[dV\_CMF\_PROF]    Frequency spacing in km\,s$^{-1}$ across the CMF profile (i.e., from the red edge of the resonance zone to +2$V_{\infty}$). Utilized primarily in the statistical equilibrium calculations. Use a value a few times that of the DOPPLER width. When the final flux calculation is performed this is reduced to FRAC\_DOP$\times$DOPPLER\_WIDTH (if [INS\_F\_FORM\_SOL] is TRUE).

\item[dV\_CMF\_WING]    Frequency spacing in km\,s$^{-1}$ in the electron scattering line wings of CMF profile. Generally use a few hundred km\,s$^{-1}$.

\item[ES\_WING\_EXT]    Extent of non-coherent electron scattering wings beyond the resonance zone in km\,s$^{-1}$. 2500 km\,s$^{-1}$ is satisfactory.

\item[R\_CMF\_WING\_EXT]    Extent of coherent electron scattering wings beyond resonance zone (in units of $V_{\infty}$). When electron scattering strongly influences the profiles, a value of 3.0 should be used. The option allows for the fact that in an expanding atmosphere electron scattering always increases the wavelength of a photon (assuming COHERENT scattering only).

\end{mylist}

\blankhalf
\shd{Options for calculation of spectrum in observer's frame.}{obs_spec}

\vspace{-10pt}
\begin{mylist}{longest label}
\item[OBS\_EXT\_RAT]    Half-width of observed profile in units of $V_{\infty}$. Value must be $> 1$. Typically adopt 1.1

\item[dV\_OBS\_PROF]    Frequency spacing in km\,s$^{-1}$ across observed profile. Typically one Doppler width.

\item[dV\_OBS\_WING]    Frequency spacing in km\,s$^{-1}$ in electron scattering line wings. 200 km\,s$^{-1}$ is reasonable.

\item[dV\_OBS\_BIG]    Frequency spacing in km\,s$^{-1}$ between lines i.e., in the continuum. Your choice. Typically we adopt 2000 km\,s$^{-1}$.

\item[FLUX\_CAL\_ONLY]    Switch to indicate to CMFGEN that it should undertake a pure FLUX calculation i.e., it should compute the observed spectrum. No corrections are made to the atomic populations in this mode. To compute a CONTINUUM spectrum, set this option to TRUE and [GLOBAL\_LINE] to SOB. To compute the He2 (i.e. He II) spectrum set this option to TRUE, [GLOBAL\_LINE] to NONE, [TRANS\_He2] to BLANK, and [TRANS\_XzV] to SOB (all other species). This option, and the following 4 options, are now obsolete in CMFGEN. These tasks are now performed in CMF\_FLUX. This option must be FALSE when computing a full model. Options have the same meaning in CMF\_FLUX\_PARAM file for CMF\_FLUX.

\item[EXT\_FRM\_SOL]    Extend formal solution, by extrapolation of the opacity and emissivity, a factor of 10 in R. Useful to see whether the choice of RMAX is influencing the OBSERVED spectrum.

\item[INS\_F\_FRM\_SOL]    Insert extra frequencies into the formal solution. These extra frequencies are inserted to improve the line profile computation. Set to TRUE.

\item[FRM\_OPT]    Method for evaluating OBSERVERS frame spectrum using a pure comoving-frame approach. Options are INT and DIFF, which have the same meaning as for [FG\_OPT]. INT is slower, but more stable.

\item[DO\_SOB\_LINES]    Calculate line equivalent widths when doing Sobolev calculation.

\item[SOB\_FREQ\_IN\_OBS]    Consider lines treated in Sobolev mode when computing the continuum frequency set.

\item[LAM\_SET]    Switch to SOBOLEV option for lines with $\lambda$ smaller than
[F\_LAM\_BEG] or greater than [F\_LAM\_END]. We use the GLOBAL option for lines with [F\_LAM\_BEG] $< \lambda < $ [F\_LAM\_END]. For this option, [F\_LAM\_BEG] is normally zero. Don't use with CLUMPING. Useful for computing the observed flux of a small spectral region with CMF\_FLUX.

 \item[F\_LAM\_BEG]    If  [FLX\_CAL\_ONLY] is TRUE, it is the wavelength (in Angstroms) at which we begin the flux calculation with lines. It is also used with the [LAM\_SET] option. Note that the continuum flux is computed at wavelengths shorter than [F\_LAM\_BEG].

 \item[F\_LAM\_END]    If [FLX\_CAL\_ONLY] is TRUE, it is the wavelength (in Angstroms) at which we end the flux calculation. It is also used with the LAM\_SET.

\item
[VERBOSE]    Hidden logical keyword: Indicates that additional error messages, warnings etc are to be output. Option is not fully implemented. Default is FALSE.

\item
[WRITE\_RATES]   Defalut is FALSE. Indicates whether the very large ascii fles, NETRATE, TOTRATE, and LINEHEAT should be created.


\end{mylist}
%\end{description}


\vskip 20pt
\shd{Options controlling the treatment of bound-bound transitions.}{control_bb_trans}

\begin{mylist}{longest label}

\item[GLOBAL\_LINE]    Switch to indicate which modes are to be used to compute the net-rates (etc) for individual line transitions. 

\begin{plt_list}{GG\_ONLY}

\item
[BLANK]   All lines treated in blanketing mode (preferred mode).

\item
[SOB]    All lines treated using the SOBOLEV approximation. Use this model for calculation of the continuum spectrum, or a Sobolev model.

\item
[CMF]    All lines treated in the CMF but in NON-BLANKETED mode. This option is obsolete. Use SOB option for fast-dirty models, and the BLANK option for sophisticated modeling.

\item
[NONE]    The computation mode for lines from EACH species XzV is specified by [TRANS\_XzV]. Use this mode to examine the effect of lines due to a particular species (on a model or on the observed spectrum). The direct influence of individual lines on the observed spectrum is now calculated better with CMF\_FLUX.

\end{plt_list}

This option has addition features in CMF\_FLUX (see \S~\ref{cmf_flux_gl}).

\item[GF\_CUT]    Omit lines with $gf < $[GFCUT] and lower level greater than GF\_LEV\_CUT. At present used only for species with an atomic number $>$ [AT\_CUT]. Useful for neon, and higher elements.

\item[GF\_LEV\_CUT]    Only omit transitions to levels, in each ion, if their lower level sequence number $> $[GF\_LEV\_CUT]. Normally set to a low number like 10, although 0 is valid. This option has been partially superseded by [MIN\_TRANS]. We generally don't want to omit transitions to the ground term, since even
weak transitions can be important.

\item[AT\_CUT]          Allows lines meeting the [GF\_CUT] criterion to be omitted provided the elemental atomic number is $\ge$ [AT\_CUT].

\item[MIN\_TRANS]    This gives the minimum number of downward transitions from a level that MUST be included before any transitions are omitted from the calculation. A non-zero value for this option ensures that no important transitions are omitted, even if they are very weak. Typically set to 10, but could be lower.

\item[THK\_LINE]    Use thick line boundary condition in CMF mode. Preferred option is TRUE.

\item[CHK\_L\_POS]    Check for negative line opacity in CMF and SOBOLEV modes. Necessary as the radiative transfer equation, as formulated, is unstable to switches in the sign of the opacity. If TRUE, the line opacity is adjusted to ensure that the total opacity remains positive. Usually affects only weak lines in the IR. Preferred option is TRUE, except (perhaps) in SOBOLEV mode.


\item[NEG\_OPAC\_OPT]    Option for treating negative opacities in BLANK mode. Two options presently available: SCRE\_CHK and ESEC\_CHK. ESEC\_CHK was the previous default. Basically it sets the total opacity to 0.1*ESEC (electron scattering opacity) when, as a consequence of a negative line opacity, it falls below this value. SRCE\_CHK was introduced for some O star models because the line opacity for some far IR lines [e.g., H(9-8)] could become negative (or very small) during the iteration process, while their (absolute) Sobolev optical depth was still large (i.e., $10^5$). In such cases the ESEC\_CHK option caused problems because the SOURCE function became much too large. ESEC\_CHK is preferred option for W-R stars, except when there is a problem (see trouble shooting section).
For O stars I tend to use SRCE\_CHK. 

\item[He2\_RES=0]    Logical variable: Sets the rates in He\,{\sc ii} resonance lines to zero. Obsolete option not utilized. Set to FALSE.
                
\item[ALLOW\_OL]    Include line overlap in SOBOLEV approximation. Line overlap is included only crudely.

\item[OL\_DIF]    Maximum velocity difference (km\,s$^{-1}$) for overlapping lines. Typically adopt 50 km\,s$^{-1}$. Only of use with the SOBOLEV approximation.

\vskip 20pt
\shd{Options controlling the heating and cooling due to line terms}{line_cool}

\item[SCL\_LN]    Scale the line cooling/heating rates before adding to the radiative equilibrium equation. Set to TRUE for non-supernovae models. This option adjusts for the fact that the energy levels in a super level do not have exactly the same energy. Forcing the levels in a SL to have the same departure coefficient can introduce an artificial heating/cooling term. The cooling/heating rates are scaled by the ratio of the mean wavelength to the actual wavelength. Especially important in O star winds where scattering dominates over collisional processes. It can also be important in cases where only a few super levels per ion are used.

\item[SCL\_LN\_FAC]      Performs the [SCL\_LN] scaling only if the lines lie within a factional spacing of [SCL\_LN\_FAC]. Set to 0.5.

\item[SCL\_DEN\_LIM]    Hidden: switches off the scaling of the line cooling rates when the electron density exceeds SCL\_DEN\_LIM. Default value is 10$^{30}$ (i.e., don't switch off).

\item[SCL\_SL\_OPAC] Alternative option to [SCL\_LN] which was developed to handle Type II SN in the inner regions where the  optical depth can be very high (e.g., $> 10,000$). In this approach the line opacities and emissivities are both scaled (scaling preserves ratio of $\eta/\chi$) so that the line cooling from a SL is zero when then the netrate from the same SL is zero. Since the scaling is generally small (e.g. less than a few \%) this does not effect the spectrum.
Only one of [SCL\_SL\_OPAC] and [SCL\_LN] can be true.

\item[NEW\_LINE\_BA] Default is T(rue) for SN, and F(alse) for other models. When true, it switches to an alternative method of evaluating d(Rad. Eq.)/dT
which is advantageous at very high optical depths. In the outer region procedure procedeure can cause convergence. Thus, for depths 1 to [INDX\_BA\_METH] we switch to the old method.

\item [INDX\_BA\_METH]. For depths 1 to  [INDX\_BA\_METH]-1 we use the old technique for evaluating d(Rad. Eq.)/dT; outside this range we utilize the new technique. The default value for [INDX\_BA\_METH] is 45. Because of the way we
treat strong lines and the cooling, this can effect convergence.

\end{mylist}

\vskip 20pt
\shd{Physical processes to be included}{phys_proc}

\begin{mylist}{longest label}

\item[INC\_CHG]    Include charge exchange reactions --- with hydrogen and helium. Can be important in models where the neutral hydrogen or helium fraction is greater than 10$^{-4}$. Can be set to TRUE for all models, but in most O and W-R models it is generally set to FALSE.

\item[INC\_TWO]    Include two photon processes (e.g., decay from the 2s\,$^2$S state of H). Set to TRUE.

\item[INC\_PEN] Include Penning ionization. Penning ionization refers to the following process:
\blankline
He\,{\sc i}(1s\,2s\,$^3$S) + H(1s\,$^2$S)  $\rightarrow$ HeI(1s$^2$\,$^1$S) +  H$^+$ + e$^-$

Recall  that the 1s\, 2s\,$^3S$ state of He\,{\sc i} is highly metastable.

\item[INC\_RAY]    Include Rayleigh scattering for H. Accurate longward of Lya, but changes needed at shorter wavelengths. In practice, this should make little difference as very little is flux emitted at these wavelengths when Rayleigh scattering is important. Rayleigh scattering due to neutral He and He$^+$ has not yet been included.    

\pagebreak
\item[INC\_AD]    Include adiabatic cooling. This option should only be included when [INC\_ADV] is T. {\color{red} This option works for both massive stars and time dependent SN calculations (although different routines are called).} 

    For massive star models this option can generally  be set to FALSE. For low-mass loss rate models, adiabatic cooling can be very important for the temperature structure although its effect on the spectrum is generally small (which is why it can generally be neglected). Check GENCOOL file to see if adiabatic cooling should be included -- the relevant terms are output even if switched off. If IN\_ADV is true, and if adiabatic cooling is important, convergence will be slower. This is expected since the temperature structure is strongly coupled to deeper depths even in the optically thin region of the stellar wind. In addition populations of impurity species that are coupled to the temperature may change dramatically. To accelerate code convergence, try changing the [SCALE\_OPT] from MAJOR to NONE for a few iterations.
    
  For SN, adiabatic cooling is related to the material at the previous time step. Thus the treatment is fundamentally different from that used for massive stars.
  
\item[INC\_ADV]    Include advection terms in the rate equations for massive stars with winds.  {\color{red} This option is NOT to be used with SN (for SN set [DO\_DDT]=T)}. These will generally become important when adiabatic cooling is important. Advection terms will be important when there are ionization changes in the wind, and when the flow time (R/V) is comparable to, or shorter than, the recombination time [ $1/(\alpha Ne)$ ]. Essentially the same conditions for including adiabatic cooling. When advection is important, the ionization state becomes ``frozen in''.

\item[LIN\_ADV]    Hidden: compute advection terms using derivatives computed in the linear plane (default is TRUE).

\item[ADV\_RELAX]    Hidden: parameter to allow advection terms to be added slowly. Default is 1.0 (i.e., no scaling).

\end{mylist}

\vskip 20pt
\shd{X-ray options}{vadat_xrays}

\begin{mylist}

\item[INC\_XRAYS]    Include X-rays: Two choices are available: Pure free-free emission (which several underestimates the X-ray emissivity) can be chosen or the X-ray emissivity can be read from a data file which is output from the X-ray code of Raymond and Smith. At present only tables assuming solar, LMC, or SMC abundances available. When X-rays are present, set [MAX\_CF]=1000 (i.e., the maximum continuum frequency is $1000 \times 10^{15}$\,Hz). This ensures that the frequency range is large enough to cover most of the auger ionization edges.
\label{vadat_xrays}

\item[FF\_XRAYS]    Assume X-rays arise from free-free processes only. If false, emissivities are read from table.

\item[X\_SM\_WIND]    When clumping is switched on, there is an ambiguity in how to interpret the filling factors. When [X\_SM\_WIND] is FALSE, the clumped densities are used to evaluate the X-ray emissivity. When [X\_SM\_WIND] is TRUE, the emissivity is multiplied by the clumping factor. In this case the X-ray emissivity will be preserved when $\Mdot/\sqrt{f}$ is held fixed.

\item[VS\_XRAYS]    Sets frequency sampling (in \kms) in the X-ray region. For the SOB/CMF modes, it sets the band over which the X-ray flux is smoothed.

\item[FIL\_FAC\_1]    Filling factor for first X-ray component (e.g., $1.0\times10^{-3}$). The X-ray emission is proportional to the square of the (local density) $\times$ (the filling factor). For very low mass-loss rates it may be necessary for the filling factor to approach, even exceed unity, in order to match the observed X-ray luminosities.      NB: A second X-ray component can be added with \_2 used to denote the second component. 

\item[T\_SHOCK\_1]    Shock temperature in program units of 10$^4$\,K.

\item[V\_SHOCK\_1]    Scale height in km\,s$^{-1}$ indicating where shocks become important: Typically 400 km\,s$^{-1}$.

\item
[XSLOW]    When true, X-rays are included gradually in the MODEL. Its use is recommended when including X-rays in a previously converged model that did NOT contain X-rays. When TRUE, the X-ray filling factor ([FIL\_FAC\_1]) is first set to [XFI1\_BEG]. After a suitable convergence of the model has been obtained, the filling factor is increased by a factor [XSL\_FAC]. This process is continued until convergence, and should be done using LAMBDA iterations. This procedure is recommended since the populations of high ionization species can change by more than 20 orders of magnitude when X-rays are included.

NB:  CMFGEN automatically revises [XFI1\_BEG] and [XFI2\_BEG] in VADAT to the new revised values as currently being used in CMFGEN. If [DO\_LAMBDA\_AUTO]=T in IN\_ITS, or if not set, CMFGEN will automatically switch to non-$\Lambda$ iterations when large changes have ceased, and [FILL\_FAC\_1] and [FILL\_FAC\_2] have reached their desired values.

 \item[XFI1\_BEG]    Starting value for X-ray filling factor [FILL\_FAC\_1] when [XSLOW] is TRUE. Try using a value like  $1.0 \times 10^{-12}$, although the optimal value is model dependent.

\item[XFI2\_BEG]    As for [XFI1\_BEG] but for the second X-ray component.

\item[XSCL\_FAC]    When [XLSOW]=TRUE, this factor is used to ``slowly' increase the X-ray filling factor as the model converges. I typically adopt 100. If the scaled value is greater than [FIL\_FAC\_1] (or FIL\_FAC\_2), [FILL\_FAC\_1] (FIL\_FAC\_2) is used.

\item[V\_XRAY]    Maximum velocity separation, in \kms, between evaluations of the X-ray photoionization cross-sections. Default if 0.5 $\times$ VS\_XRAYS. 

\end{mylist}

\vskip 20pt
\shd{Options for beginning a NEW model}{new_mod}

\begin{mylist}{longest label}

\item[RD\_IN\_R\_GRID]    Logical variable which indicates whether a predetermined R grid should be read in. The file should have the same format as the departure coefficient files. Generally set to FALSE -- it cannot be used with RVSIG\_COL files. The R-grid in RVSIG\_COL files can be revised using \$cmfdist/exe/rev\_rvsig.exe.

\item[LIN\_INT]    Interpolate populations from an ``old model.''  Should be FALSE when fundamental
stellar parameters [L, \Rstar, \Mdot, V(r)] have changed. Set the [LIN\_INT]=TRUE option when changing the atomic models, computation options, and the number of depth points. NB: If [LIN\_INT] =TRUE, no temperature iteration is performed, and the temperature structure is read in.

\item[POP\_SCALE]    Scale pops to satisfy abundance equations. Usually TRUE, but setting it to F(alse) might be useful if fudging start populations because model is having convergence difficulties.

\item[IT\_ON\_T]    When true, CMFGEN iterates on the initial temperature distribution before starting the first iteration. MUST be set to TRUE when initiating new model with NEW stellar parameters. If merely changing the atomic models, or the number of depth points, set to FALSE, and set [LIN\_INT] to TRUE. It should also be set to FALSE for SN models in which the ejecta is ``optically'' thin.

\item[T\_INIT\_TAU]      When [IT\_ON\_T] is true, the temperature is initially set to an interactively determined temperature based on the electron density, and the temperature read in. This is only done for temperatures where the electron scattering optical depth is greater than [T\_INIT\_TAU]. Set to 5 when  corrections are being read in from GREY\_SCL\_FAC\_IN, otherwise 1 is a more reasonable value. Ideally the value should be larger than [GREY\_TAU], since the grey correction provides better accuracy.  The temperature computed with this method is subsequently revised using a grey model atmosphere.

\item[GREY\_TAU]    Set the temperature to its GREY value for $\tau > $[GREY\_TAU], but leave the temperature unchanged in the outer wind. Typically adopt $\tau = 1$. For O stars with weak winds, 0.5 (or even a value of 0.1) is better. Lower values (i.e., around 0.1 to 0.3) work very well when T/TGREY is read in from GREY\_SCL\_FAC\_IN (see below). This option is used to adjust the temperature distribution when initiating a new model.
    
Generally new models start by reading in a table of $T$/\Psub{T}{grey}\ for another similar model from a file called GREY\_SCL\_FAC\_IN (formerly GREY\_SCL\_FAC). This file should contain ND, and \tauross, $T$/\Psub{T}{grey}\ in two column format. For current models, a file GREY\_SCL\_FACOUT is created which contains the relevant data in the appropriate format, and this is copied to GREY\_SCL\_FAC\_IN when the {\bf cpmod} command is used. For very old converged models, the file, GREY\_SCL\_FAC\_IN, can be created in DISPGEN by issuing the following sequence of commands:  {\color{OliveGreen} GREY, GR[e], XROSS, T, TGREY, GR[VAR\{/,1,2,3\};  WXY\{GREY\_SCL\_FAC, 3\}], e}. You will need to edit out the first line of GREY\_SCL\_FAC (which contains the number of plots).

For SN model, higher values of [T\_INIT\_TAU] (=20?) and [GREY\_TAU]  (=10) are often used. 

\end{mylist}

\vskip 20pt
\shd{Options specifying method of handling lines.}{line_handling}

\begin{mylist}{longest label}

\item[TRANS\_XzV]    Method for treating lines in ionization stage XzV. Available options are BLANK, SOB, and CMF. See description of [GLOBAL\_LINE] option discussed above. These specifications need not be present, and are ignored EXCEPT if [GLOBAL\_LINE]\linebreak=NONE. 

\item[DIE\_AS\_LINE]    Treat dielectronic as non-overlapping lines. Preferred option is FALSE, particularly with blanketing calculations. [DIE\_AS\_LINE]=TRUE has not been tested recently, so BEWARE.

\item[VSM\_DIE]    Velocity (km\,s$^{-1}$) used to smooth dielectronic lines when added to photoionization cross-sections. Use a value similar to that used to smooth the photoionization cross-sections. The adopted value should NOT be less than [V\_CROSS]. Used to insure that aliasing doesn't occur when we sample the continuum photoionization cross-sections on a coarse frequency grid.

\item[DIE\_XzV]    Include Low Temperature Dielectronic recombination for species XzV. Two logical values must be specified. The first indicates whether normal dielectronic calculations are included. The second indicates whether dielectronic recombinations from levels forbidden to autoionize in LS coupling are included. The first must be FALSE when OPACITY photoionization cross-sections, which explicitly include the resonances, are utilized. BE CAREFUL.
\end{mylist}

\vskip 20pt 
\shd{Options for assisting convergence}{assist_conv}


\blankline
Of these options, [FIX\_T] and [FIX\_T\_AUTO], are the most useful One useful approach, provided the temperature structure is reasonable, is to set [FIX\_T]=T initially. The model can the use Lambda-iterations, and full iterations, to get the populations "consistent" with the radiation field. If [USE\_T\_AUTO]=T in IN\_ITS, [FIX\_T] will automatically be adjusted to allow the temperature to vary.

\begin{mylist}{longest label}
\item[FIX\_ALL\_SPEC]    When true all species are held fix. This option can be used with [UNFIX\_XzV] to allow a single, or multiple ionization stages, to be updated by the CMFGEN calculation. XzV. Mainly used when adding new species and/or ionization stages. Default is FALSE.

\item[UNFIX\_XzV]    Allows a single ionization stage to be removed from the influence of the [FIX\_ALL\_SPEC] option. Set to 0 to UNFIX the ionization stage

\item[FIX\_XzV]    Fix the first [FIX\_XzV] levels for species XzV. Option is mainly for debugging purposes. If a particular level is causing convergence difficulties, it can be fixed while the other populations are solved for. Usually set to 0, and need not be
present in VADAT.

\item[FIX\_DUM]    To fix the highest available ionization state of species DUM, set to unity. For example, when DUM=HE, He$^{++}$ is held fixed when all ionization stages of helium are included. Usually set to 0.

\item[FIX\_NE]    Fix the electron density. Option is mainly for debugging purposes. If the electron density is causing convergence difficulties, it can be held fixed when the other populations are solved for. Usually set to FALSE.

\item[FIX\_IMP]   When TRUE, impurity species are automatically held fixed. Option is mainly for debugging purposes. If a particular set of levels is causing convergence difficulties, and they have very low populations, they can automatically be held fixed when the other populations are solved for. Usually set to FALSE. A low population level population is
currently defined as $10^{-15}$ of the species population.

\item[FIX\_T]    Tells CMFGEN to fix the electron temperature. This option was mainly used for debugging purposes but setting it to TRUE initially can help with convergence stability. As rule, I now generally set it to TRUE in VADAT (especially for SN models), and set [DO\_AUTO\_T]=TRUE in IN\_ITS. This will automatically cause CMFGEN to allow the temperature to vary, once the populations have stabilized (in which case CMFGEN will set [FIX\_T]=FALSE in VADAT). Setting this option to TRUE is also useful when the temperature is causing convergence difficulties, as often the convergence difficulties are coupled to poor population estimates. [FIX\_T[ can also be used to compute models where the temperature structure is input, and is not to be changed (but make sure [LIN\_INT] is also TRUE).

\item[FIX\_T\_AUTO]   To improve convergence reliability the temperature is typically held fixed until the changes are less than a factor of 5. The temperature is held fixed at all depths with {\color{OliveGreen} $\tauross  < \tauross$(Maximum population change on last iteration $< 5$)}. [FIX\_T\_AUTO] is normally set to TRUE, although occasionally it can cause difficulties, and may need to be  turned off. Turn it off if you see a discontinuity in temperature at the location where the switch is occurring. Setting to FALSE can speed up some models, but may lead to a more erratic convergence. If you set [FIX\_T\_AUTO]  to FALSE, I strongly recommend beginning a new model with [FIX\_T]=TRUE. For SN models I tends to start the model with [FIX\_T]=TRUE and [FIX\_T\_AUTO]=FALSE.

\item[FIX\_INB\_T]    Fix the inner boundary temperature at depth. Parameter is hidden with a default value of FALSE. It was installed to assist convergence of some time-dependent SN models with very large optical depths --- it should only be fixed when T has been set, at depth, to the grey temperature distribution, and the grey temperature distribution at depth is accurate. It should not be needed for stellar models. Used with [FIX\_X\_DPTH].

\item[FIX\_X\_DPTH]     Indicates the number of depths for which T is held fixed. It is only used when [FIX\_INB\_T]=.TRUE.

\item[TAU\_SCL\_T]    Obsolete. Fix T for this, and lower, optical depths. Normally set to 0. Useful for converging difficult models. [FIX\_T\_AUTO] is an automatic version of this option.

\item[T\_MIN]    Approximate minimum value that T can have in a converged model. Usually set to 0.0D0. Can be needed in some models because of exponential overflows when computing LTE populations (i.e., when T=0.2 [i.e., 2000\,K] and you have FeX (for example) in the model.

\item[ADD\_OPAC]    Hidden: Experimental parameter. Adds additional opacity to model. Use for testing purposes ONLY. Unless you know what you are doing, this should be FALSE.

\item[OP\_SCL\_AFC]    The added opacity has the form: \\
 \hbox{} \hspace{1.5in} OP\_SCL\_FAC $\times$SIGMA(THOMSON)$\times$N(ATOM).

\vskip 20pt 
\shd{Options for controlling the linearization}{linear}

\item[SOL\_METH]    Method for solving the LINEARIZE statistical equilibrium equations. Options are DIAG, TRIDIAG, and PENTADIAG. The NUM\_BNDS variable in the  MOD\-EL\_SPEC file must be compatible with the chosen option (i.e., can't choose TRIDIAG and have NUM\_BNDS=1). NUM\_BANDS=3 and the DIAG option is okay, except that it wastes memory. TRIDIAG is the best compromise between convergence and memory requirements. DIAG saves a factor of roughly 2 in memory, but  has a slower rate of convergence. 

\item[SCALE\_OPT]    Option to indicate how the corrections found from the linearized statistical equilibrium equations are scaled, before they are used to improve the population estimates. Preferred option is MAJOR.

\begin{plt_list}{GLOBAL\~}

\item[MAJOR] At EACH depth the maximum change in any population is limited to a factor of [MAX\_LIM]. All other corrections are scaled so that this is satisfied. Impurity species (i.e., those with very small populations) are not considered in determining the scaling factor. Their scaling factor is determined independently. For $\Lambda$--iterations the corrections are limited to a factor of [MAX\_LAM].
           
\item[LOCAL]    As for MAJOR, but no distinction between impurity and major species. 

\item[GLOBAL]   As for local, but all depths are scaled by the same value. This option really slows convergence, and is never used!

\item[NONE]    Scaling performed on each variable, without consideration of the corrections to other variables. Generally not recommended (since it can cause very large changes in some populations with unstable oscillations), but switching to this option might accelerate convergence when the temperature needs large adjustments.

\end{plt_list}

\item[EPS\_TERM]    Terminate model when maximum fractional change for a FULL linearization is $< $ [EPS\_TERM]\%. Typically choose 0.01 to 0.1. Deciding on the level of convergence may be difficult -- often the maximum changes is driven by some irrelevant population. One approach, when entering a different model regime, is to decrease EPS\_TERM and run a more highly converged model. Then compare the obs\_fin file computed with each model. Other checks include seeing how well $L$ is conserved as a function of depth, and checking if the electron cooling balance is satisfied. In practice, numerical errors mean that these comparisons only provide only a check that the model is not grossly in error. Note: Often running a model to a much higher levels of convergence is relatively fast since the linearization matrix is held fixed, and Ng acceleration is generally effective.

\item[MAX\_LIM]    Maximum fractional correction to allow for a full linearization iteration. Thus an atomic population can be reduced by a factor of [MAX\_LIM], or increased by a factor of [MAX\_LIM]. Typically adopt [MAX\_LIM]=10. We use this limit to provide convergence stability  - a larger limit might, in some case, provide faster convergence.

\item[MAX\_LAM]    Maximum fractional correction to allow for $\Lambda-$iteration. Thus an atomic population can be reduced by a factor of [MAX\_LAM], or increased by a factor of [MAX\_LAM]. Typically adopt [MAX\_LAM]=$10^5$. For a new species, with very low abundance, this can be set as high as $10^{20}$.

\item[MAX\_CHNG]    Terminate model with an error if the \% fractional change is greater than [MAX\_CHNG]. Typically we adopt $1.0 \times 10^{40}$ for routine models. A larger value is adopted when introducing a new species whose populations are very uncertain. Although the models stops, the POINT1, POINT2 and SCRTEMP files are updated. You can rewind to the last iteration by editing POINT1. The option is utilized to indicate that the model is likely to have difficulty converging, or that the model may have been incorrectly started/specified.

\item[COMP\_BA]    Write the BA (the linearization) matrix out after each full iteration. Option normally set to FALSE.  The BA matrix will still be written out when MAXIMUM correction is less than [BA\_CHK\_FAC].

\item[STORE\_BA\_INV]    Store the inverse of the BA matrix. For very large matrices, this can save computational effort. These are the large scratch files of the form DSCRTATCH01 (etc) discussed earlier.

\item[N\_FIX\_BA]        When the population corrections are small, the BA matrix is held fixed (see [BA\_CHK\_FAC] option). To save even more time, we now can hold the BA matrix fixed for N\_FIX\_BA iterations, independent of the current correction sizes. The recommended value is 2.


\item[WRT\_PRT\_INV]     Write out part of the inverse. Designed to save memory on systems where the FULL BA matrix is larger than the available memory. 

\item[STORE\_BA]        Write BA MATRIX out. Should be TRUE.

\item[BA\_CHK\_FAC]    This option indicates when to hold the linearization matrix, BA, fixed. The BA matrix is written to the disk and used for subsequent iterations. Typically we adopt 5\% for the largest fractional change. Option could be improved.

\item[LAM\_VAL]    When the largest fractional change (absolute value) is greater than [LAM\_VAL], a $\Lambda$-iteration is performed. Typically do a $\Lambda$-iteration if \% change is greater than 400.

\item[NUM\_LAM]    Maximum number of $\Lambda$-iterations before a full iteration is again performed. Typically choose 2. Lambda iterations are performed to improve convergence.
\end{mylist}

\shd{Options for an enhanced spatial grid.}{spat_grid}

\begin{mylist}{longest label}
\item[INC\_GRID]    When TRUE,  an improved spatial grid is created on which the radiative transfer equation is solved. At present, the linearization on the improved spatial grid  is not available with all radiative transfer modes.  Useful for improving convergence in the presence of ionization fronts. Convergence is generally inferior to models with [INC\_GRID]=F.  In CMFGEN it is generally set to FALSE. However, it is utilized in \$cmfdist/exe/cmf\_flux\_v5.exe to improve the accuracy of spectral calculations (set in CMF\_FLUX\_PARAM\_INIT).

\item[ALL\_FREQ]    Use the improved spatial grid for all frequencies. Not used if [INC\_GRID] is FALSE, and its recommended value is TRUE.

\item[ACC\_END]    Only use the improved spatial grid if the frequency (in units of 10$^{15}$\,Hz) is greater than [ACC\_END].

\item[N\_INS]    Number of points to be inserted into each interval of the depth grid. Extra points are only inserted between [ST\_INT] and [END\_INT]. No effect if [INC\_GRID]=FALSE.

\item[ST\_INT]    Interpolate from [ST\_INT] to [END\_INT]. No effect if [INC\_GRID]=FALSE.

\item[END\_INT]    Interpolate from [ST\_INT] to [END\_INT]. No effect if [INC\_GRID]=FALSE.

\item[ND\_QUAD]    Use quadratic interpolation for ND-ND\_QUAD to ND. Helps preserve diffusion approximation at depth. No effect if [INC\_GRID]=FALSE. If $> $ ND, it is set to 5.

\item[INTERP\_TYPE]    Type of interpolation for [INC\_GRID] option. Use LOG or LIN. No effect if [INC\_GRID]=FALSE.
\end{mylist}

\shd{Eddington and BA matrix computation}{edd_comp}

\begin{mylist}{longest label}
\item[N\_PAR]    Rate at which BA matrix is updated by BA\_PAR matrix. The BA\_PAR matrix was included to help improve stability, as we UPDATE the BA matrix for every line, and for every frequency (although no problems were actually observed). BA is updated by BA\_PAR every [N\_PAR] frequencies in BLANKETING mode. Typically I adopt 2000, and even 5000 for very large modes. NB: For small models, 200 used to be the recommended value. However for large models, and $\Lambda$--iterations, this can significantly increase the required computation time. The file,
TIMING, can be checked to see that the update of the BA matrix is not taking too much time (section ADD\_PAR).

\item[COMP\_F]    Tells CMFGEN whether to compute new Eddington factors. Generally set to FALSE. Simply deleting the EDDINGTON factor file can enforce this option.

\item[ACC\_F]    Accuracy with which Eddington factors are computed on each iteration. Typically we adopt $10^{-4}$ for a new model. In practice, the Eddington factors will improve as the model converges. With the completion of a new model the precise value of [ACC\_F] will be irrelevant unless you had to compute the EDFACTOR file from scratch and only performed a few iterations.  
\end{mylist}

\pagebreak
\shd{Parameters to control Ng accelerations}{ng_accel_params}

\begin{mylist}{longest label}

\item[DO\_NG]    Do an Ng acceleration when the maximum corrections are less than [BEG\_NG]. An Ng acceleration can be forced, after a model has completed, by running the FORTRAN program DO\_NG, which reads the SCRTEMP file.

\item[BEG\_NG]    Typically choose 5\% to 10\%. Typically a model must run for over 10 to 15 iterations before an Ng correction is applied, even if the corrections are small. When an Ng acceleration is applied too early, convergence may be worse. A more sophisticated decision process for performing Ng accelerations would help convergence.

\item[IBEG\_NG]         Even if the [BEG\_NG] criterion is met, an Ng acceleration is not performed until [IBEG\_NG] iterations have been completed.

\item[BW\_NG]           Bandwidth (i.e., range of depths over which Ng acceleration is applied simultaneously). Used to use 1, but a value of ND may be better. This is especially true when using a diagonal operator. NB: A value larger than ND is set to ND.

\item[ITS/NG]    Number of iterations between  Ng accelerations. We typically adopt 8.

\item[DO\_AV]  Perform averaging of oscillating variables in CMFGEN -- simple averaging is DONE. Default is FALSE.

\item[NOSC\_AV] Only do averaging of population oscillations for [NOSC\_AV] iterations. Default is 4. No effect if [DO\_AV]=FALSE.

\item[ITS/AV] Number of oscillations between averaging: Default is 8. No effect if [DO\_AV]=FALSE.

\vfill\eject
\shd{SN model options}{sn_options}

\vspace{-3pt}

\item[TS\_NO]    Sequence number for supernovae model, beginning with model 1. Model 1 must be computed ignoring time derivatives, and hence is usually computed with a fixed temperature structure computed by a hydrodynamical model.

\item[SN\_AGE]    Age of SN in days.

\item[DO\_DDT]    When true, advection terms, via the comoving derivative, are included in the statistical equilibrium equations. This option is for time-dependent SN models.

\item[INC\_RAD\_DEACYS]    Include radioactive decays. For use with time-dependent radiative transfer models.

\item[REL\_OBS]    Include all relativistic terms in the observer's frame solution.

\item[REL\_CMF]    Include all relativistic terms in the CMF solution for the observed intensity.

\item[SN\_T\_OPT] When set to USE\_T\_IN, temperature structure is read in from T\_IN (same format as XzV\_IN file).
                               When set to USE\_HYDRO, it uses the temperature structure in the hydrodynamical model.

\item[JG\_W\_V]    Include velocity terms when evaluating initial grey temperature structure.

\item[COMP\_GREY\_LST\_IT]  -- Indicates whether grey solution should  be computed on the last iteration. The computation of the grey solution was an issue for some SN models, but a failure
no-longer cause the code to crash.

\item[GAMRAY\_TRANS] When set to LOCAL, gamma-rays deposit their energy locally --- their is no radiative transport. When set to ABS\_TRANS a simple approximation (which assumes pure absorption) is used. When set to NONLOCAL, data describing the energy deposition is read in from a file.

\item[GAMMA\_SLOW] Add radioactivity decay energy slowly? This option is mainly used for developmental work.

\item[DECNRG\_SCLFAC\_BEG] Initial Scale factor for adding decay energy.

\item
[N\_IB\_INS]  Number of points for fine grid at inner boundary (default is 2)

\item
[N\_OB\_INS]  Number of points for fine grid at outer boundary (default is 3)

\item
[RMAX\_ON\_RCORE]  Use when you want to shrink the outer radius of a SN model.

\item[REV\_RGRID]    Indicates whether R grid should be automatically revised, after each iteration. Allows better treatment of H ionization fronts.

\vfill
\eject
\shd{Additional options for beginning a NEW SN model}{new_sn}

\item[PURE\_HUB]   Forces flow to be a pure Hubble flow (i.e., homologous expansion) using the age and radius of the SN.


\item[OLD\_MFS]  Use old mass scaling when reading SN\_HYDRO\_DATA.

\item[OLD\_MFO]  Use old mass scaling when writing out  SN\_HYDRO\_DATA.

\item[RMAX\_ON\_RCORE] Used when want to shrink radius of SN model.

\item[DC\_METH] Set the x-ordinate (i.e. the independent variable)  for interpolation of the departure
coefficients. Options are , `ED' (default), `R', `LTE' and `SPH\_TAU'.
 
\item[DC\_SPH\_TAU] When true, we interpolate departure coeficients on the spherical TAU scale.

\item[LTE\_EST] Use LTE for the initial estimates.

\item[T\_EXC] This option is used when [LTE\_EST]=.TRUE. It sets a minimum excitation temperature when computing level populations. When T(estimate) $>$ T\_EXC, LTE populations defined by T are used. When T(estimate) $<$ T\_EXC, LTE populations (and hence the ionization structure) are computed using T\_EXC. Setting T\_EXC to
approximately the photospheric temperature (or a little lower) can prevent overflows and other issues when first beginning a SN model for which there is no initial population estimates. In practice, the populations in the outer optically thin regions are more close related to the radiation field (which we characterize by T\_EXC) than the local temperature.

\item[STOP\_IF\_BP] When true, CMFGEN checks that parameters are consistent. When FALSE only a warning message is given. Mainly used  for SN models, and was implemented for testing purposes.


\vfill\eject
\shd{Non-thermal model options}{non-thermal}

\item[TRT\_NON\_TE] When true, CMFGEN computes the non-thermal electron spectrum, and takes into account its influence on the ionization and excitation state of the gas.

\item[NT\_NKT] Number of non-thermal energy bins. Default is 1000.

\item[NT\_EMIN] Minimum energy of non-thermal electrons in eV. Defaults is 1\,eV.

\item[NT\_EMAX] Maximum energy of non-thermal electrons in eV. Defaults is 1000\,eV.

\item[SCL\_NT\_CROSEC]  Logical variable to indicate whether excitation cross-sections will be scaled.
Scaling factors are read in from NT\_CROSEC\_SCLFAC. Default is FALSE.

\item[SCL\_NT\_ION\_CROSEC]  Logical variable to indicate whether ionization cross-sections will be scaled.
Scaling factors are read in from NT\_ION\_CROSEC\_SCLFAC. Default is FALSE.

\item[NT\_IT\_CNTRL] Controls how often we update the non-thermal electron distribution.

\item[NT\_OMIT\_LEV\_SCALE] Fractional populations below this level are excluded when computing the non-thermal electron spectrum. Default is $1.0 \times 10^{-4}$.

\item[NT\_OMIT\_ION\_SCALE] Excludes ions with population NT\_OMIT\_SCALE\_FRAC $\times$ (species of ion pop). Default is $1.0 \times 10^{-3}$.

\item[NT\_SOURCE] Non-thermal source type - INJECT\_DIRAC, CONSTANT or BELL\_SHAPE (default).
          

\end{mylist}

\vfill
\eject
\mhd{Auxiliary Programs}{aux_progs}

All executable names are lower case, and have a .exe extension (i.e., cmf\_flux.exe), and are located in \$cmfdist/exe.


\vskip 20pt
\mhd{CMF\_FLUX}{cmf_flux}    
       Program to compute the spectrum in the observer's frame. This is the preferred method, and gives the most accurate observer's spectrum. Results from this program are more accurate than those given in OBSFLUX (the output from CMFGEN) particularly for lines formed in the photosphere (e.g., most lines in O stars).

       The full continuum spectrum, defined to treat all bound-free edges, is always computed. Using parameters defined in CMF\_FLUX\_PARAM\_INIT, it is possible to limit the section in which lines are included, resulting in a considerable saving of computation time. A separate calculation must be done to compute the continuum flux at all wavelengths. This is handled automatically by batobs.sh. 

           Note that before the observer's frame calculation is performed, 1 or 2 comoving frame calculations are carried out. In this section the emissivity and opacity are also calculated. These will be transformed from the comoving-frame to the observer's frame in the observer's frame routine. In addition, this calculation allows the computation of the electron-scattering emissivity allowing for the frequency redistribution of line photons due to the thermal and bulk velocities of the electrons.          
       
\blankline
\indent
 A variety of intrinsic absorption profiles can be adopted:

\begin{plt_list}{LIST\_VGT}
\item[DOP\_FIX:]    Doppler but with a fixed width. It is the same for all species and at all depths.
\item[Doppler:]    Variable --- uses correct atomic mass, and allows for a depth dependent turbulent velocity.
\item[Stark:]    For H\,{\sc i} and He\,{\sc ii} only. The approximate Stark profiles are convolved with a Doppler profile. They are generally adequate, except possibly for transitions between adjacent levels (e.g., H$\alpha$). Improved Stark profiles could easily be implemented.
\item[Voigt:]    At present, only radiative damping is included. 
\item[LIST:]    Profile options for individual lines are specified in the file FULL\_STRK\_LIST.
\item[LIST\_VGT:]    This is the preferred option.
Use the options in FULL\_STRK\_LIST when available.
    Uses a Doppler profile for �weak� lines.
    Uses VOIGT profile with radiative damping for strong lines (those with a line to continuum ratio $> 10^4$).
    NB: There is  a problem with some of the Lemke Stark profiles for infrared lines.
\end{plt_list}


\vfill
\eject
\shd{Running CMF\_FLUX}{run_cmf_flux}

If the model was run in directory r1/, it is recommended that flux calculations be done in the directory r1/obs/,
or a sub-directory of similar nomenclature. 

\blankline\blankline
\shd{Required data files}{in_cmf_flux}

\vspace{-10pt}
\begin{file_list}{CMF\_FLUX\_PARAMS\_INIT}

\item[batobs.sh] Primary control file. It assigns atomic data files, can be used to run multiple
spectral calculations with different turbulent velocities, renames output files, and undertakes
a continuum only calculation.

\item[CMF\_FLUX\_PARAMS\_INIT]
    Basic control data. Keywords that have a distinct meaning from those in VADAT are described later in this document. The command file {\bf batobs.sh} copies this file to  CMF\_FLUX\_PARAM\  which is read by cmf\_flux. {\bf batobs.sh} does some simple editing of CMF\_FLUX\_PARAM to change turbulent velocities, and to perform a continuum calculation. On some operating systems, you may need to make sure CMF\_FLUX\_PARAM has been deleted before starting the job. NB: The parameters in CMF\_FLUX\_PARAM\_INIT are generally independent of those in VADAT.

\item[FULL\_STRK\_LIST]    Contains type of intrinsic line absorption profiles to be used for individual lines if [GLOBAL\_PROF] is set to LIST or LIST\_VGT. Use the latest file in \$cmfdist/misc/.


\item[FORB\_LINE\_CONTOL] This simple file is not needed but provides a quick means of omitting forbidden lines (lines between levels of the same parity) when computing the spectrum: File should contain lines with one species (i.e., C2 or CIII ) per line, or ALL. For the later case all forbidden lines  are deleted.

\item[Atomic data files]    As used by CMFGEN. Assigned using \\
 \hbox{} \hspace{20pt}   {\bf batch.sh ass}. \\
Taken care of by batobs.sh.

\item[Output files from CMFGEN]    MODEL, RVTJ,  POPDUM (for all species). The {\bf batobs.sh}
points to the RVTJ file (usually ../RVTJ to indicate that RVTJ is in directory below the spectral calculation). CMF\_FLUX assumes that the other required CMFGEN files are in the same directory as RVTJ.

\end{file_list}

\shd{Output from CMF\_FLUX}{out_cmf_flux}

The following files are generated when the script batobs.sh is used:
      
\begin{file_list}{CMF\_FLUX\_PARAMS\_INIT}

\item[obs\_fin] Main spectrum. The file contains a list of frequencies (in $10^{15}$\,Hz) and then lists the corresponding fluxes in Janskies (assuming $d=1$\,kpc). It is a raw data file --- no smoothing has been done and no effect of rotation is taken into account. The file is initially called OBSFRAME but is renamed by batobs.sh. If spectra with different microturbulent velocities are being computed, the notation obs\_fin\_10  is
used (where 10 is the microturbulent velocity in km\,s$^{-1}$). This is done by the batobs.sh file.

\item[obs\_cont] Continuum spectrum (may contain dielectronic lines if they are treated as part of the photoionization cross-sections).

\item[hydro\_fin] As for CMFGEN. Values are only meaningful if the ENTIRE spectrum has been computed.

\item[MEANOPAC] As for CMFGEN. Values are only meaningful if the ENTIRE spectrum has been computed.

\item[OBSFLUX] If batobs.sh is used, this will contain the continuum spectrum as computed using cmf\_flux, and can be ignored. The full spectrum OBSFLUX file is copied to obs\_cmf by batobs.sh. These files may be deleted. The spectrum in obs\_cmf should be very similar to that computed by CMFGEN and obs\_fin. obs\_cmf will differ from obs\_fin because of numerical inaccuracies --- photospheric lines will be broadened and wind lines may show a bleeding to the red. Interestingly, EW's tend to be preserved. If obs\_fin, obs\_cmf, and the CMFGEN OBSFLUX file show broad disagreement (except for resolution issues) there may be a problem.  The files can also differ in SN models due to time-dependent effects and/or because a wrong transfer option has been set. For example, using the diffusion approximation may lead erroneously to enhanced fluxes when the core has become optically thin.

A new batobs.sh script allows multiple observed spectra to be computed for different turbulent velocities. The output are termed obs\_fin\_N, where N refers to the turbulent velocity in km/s. The procedure only computes the CMF spectrum once (ideally for the intermediate turbulent velocity). Similarly, the EDDFACTOR file is reused. Basically, the script sets USE\_FIXED\_J=T and VTURB before re-running the model.


\end{file_list}


\vfill
\eject
\shd{Explanation of options in CMF\_FLUX\_PARAM}{cmf_flux_param}

\blankhalf
CMF\_FLUX\_PARAM is the main driver file for CMF\_FLUX. Some parameters need to be included even when they are not utilized. A few parameters need only be included when another specific parameter has been set to a specific value. Use an existing model to define this file.

All keywords in the following text are specified in bold between square brackets. Some parameters are checked for validity --- others are not. It is the responsibility of the user to ensure that the parameters are valid. 

Keywords need not be in order, although it is recommended that the ordering of keywords not be changed from that provided. If a keyword can't be found, an error message is output to OUT\_FLUX, and program execution stops. Superfluous keywords (e.g., PROF\_CIII when C\,{\sc iii} is not included) are ignored.

Additional keywords are continually added to improve the accuracy and applicability of the code. This may necessitate the CMF\_FLUX\_PARAM\_INIT file to be revised, if you wish to re-compute an observer's frame spectrum. This generally presents no difficulty, however, since the options are unique, and the code will inform you through the OUT\_FLUX file when they are not present. 

Keywords that have the same meaning as in VADAT are NOT repeated here.

\begin{mylist}

\item[COH\_ES]   When FALSE (the preferred option), incoherent scattering due to the thermal motions of the electrons is taken into account. For cmf\_flux, this is the preferred option (for CMFGEN preferred option is TRUE).

\item[NUM\_ES]    Number of iterations to be performed so that the incoherent electron scattering source function can be accurately computed. Set to 1 if [COH\_ES]=TRUE.  A value of 2 is generally found to be adequate, especially for an O star. Higher values may be necessary when line-photons undergo many scatterings off electrons. On the first pass coherent scattering is assumed.

\item[VTURB\_FIX]    Turbulent velocity in km\,s$^{-1}$ at all depths. Only utilized if \hfill \break
[GLOBAL\_PROF] $ = $ DOP\_FIX.

\item[VTURB\_MIN]    Minimum turbulent velocity in km\,s$^{-1}$. Utilized when \hfill \break [GLOBAL\_PROF] $ \ne $ DOP\_FIX. The turbulent velocity is assumed to have the form 
  $$\hbox{VTURB}= \hbox{[VTURB\_MIN]}+ (\hbox{[VTURB\_MAX]}-\hbox{[VTURB\_MIN]})*v(r)/\Vinf$$

\item[VTURB\_MAX]    Maximum turbulent velocity in km\,s$^{-1}$. Utilized when \hfill \break [GLOBAL\_PROF] $\ne$ DOP\_FIX. 

\item[TAU\_MAX]    Integrations of the source function along a ray are truncated when $\tau$ exceeds [TAU\_MAX]. 20 is a reasonable value.

\item[ES\_TAU]    Maximum step size for integrations on the Thompson electron scattering optical depth scale. 0.1 seems to work well in a variety of models. Higher values can be utilized with some models.


\item[INT\_METH]    Method for determining the intensity arising from a single ray in the observer's frame. Two options are available:

\begin{plt_list}{ETAZ}

\item[ETAZ]    Integral of the emissivity over $z$. Its advantage is that it can handle negative optical depths.

\item[STAU]    Integral of S over $\tau$. Possibly the more accurate, but depends on choice of [ES\_TAU].

\end{plt_list}

\label{cmf_flux_gl}
\item[GLOBAL\_LINE]    Switch to indicate which modes are to be used to compute the net-rates (etc) for individual line transitions. Similar to VADAT option but extra flexibility has been added. 

\begin{plt_list}{BLANK\_SPEC}

\item
[BLANK]   All lines treated in blanketing mode (preferred mode).

\item
[BLANK\_SPEC]   All lines treated in blanketing mode, except for lines belonging to ionization stages(s) whose transition type has been specified by [TRANS\_XzV].``\_SPEC" may also be added to the [SOB] and [CMF] options.

\item
[SOB]    All lines treated using the SOBOLEV approximation. Use this model for calculation of the continuum spectrum, or a Sobolev model.

\item
[CMF]    All lines treated in the CMF but in NON-BLANKETED mode. This option is obsolete. Use SOB option for fast-dirty models, and the BLANK option for sophisticated modeling.

\item
[NONE]    The computation mode for lines from EACH species XzV is specified by [TRANS\_XzV]. Use this mode to examine the effect of lines due to a particular species (on a model or on the observed spectrum). The direct influence of individual lines on the observed spectrum is now calculated better with CMF\_FLUX.

\end{plt_list}

\item[TRANS\_XzV]    Method for treating lines in ionization stage XzV. Available options are BLANK, SOB, and CMF. See description of [GLOBAL\_LINE] option discussed above. These specifications need not be present, and are ignored EXCEPT if [GLOBAL\_\linebreak LINE]=NONE or {\color{red} \_SPEC} has been appended to BLANK, SOB, or CMF. When\linebreak[4] [GLOBAL\_LINE]=NONE, [TRANS\_XzV] must be specified for every ionization stage. When {\color{red} \_SPEC} has been appended, [TRANS\_XzV] take precedence over the option specified by [GLOBAL\_LINE].

\item[WR\_ETA]    When true, CMF\_FLUX writes $\eta$, $J_{es}$, $\chi$, to direct access files, with the same format as EDDFACTOR. Can be plotted using PLT\_JH. Also used by the obsolete OBS\_FRAME.

\item[WR\_FLUX]   When true, CMF\_FLUX writes H as a function of frequency and depth to a direct access file, with the same format as EDDFACTOR. Can be plotted using PLT\_JH.

\item[WR\_CMF\_FORCE]    When true, CMF\_FLUX writes the radiative line force computed in the CMF. The file has the same format as EDDFACTOR and gives the cumulative line force (starting in at short wavelengths) as a function of frequency. Use PLT\_JH (specific options available) to display. This is only computed when [BLANK] is set to T.

\item[WR\_SOB\_FORCE]    When true, CMF\_FLUX writes the radiative line force computed using the Sobolev force and the unblanketed continuum. The file has the same format as EDDFACTOR and gives the cumulative line force (starting in at short wavelengths) as a function of frequency. Use PLT\_JH (specific options available) to display. This is only computed when [BLANK] is set to F (i.e., the continuum calculation) and [DO\_SOB\_LINES]=T. Because we use an unblanketed continuum, the force generally exceeds that computed in the CMF. It would probably be more realistic to smooth the blanketed continuum first.

\item[WR\_ION\_FORCE]    When true, CMF\_FLUX writes the radiative line force computed in the CMF for each ion (bound-bound transitions only). The file (ION\_LINE\_FORCE) is in ascii format, and is self explanatory. Due to non-linear interactions, the contribution by a given ion specified in ION\_LINE force will not be identical to the change in line force that will occur when that species is removed. Only computed when [BLANK] is TRUE.

\item[WR\_IP]    When true, CMF\_FLUX writes the outer boundary specific intensity, I, as a function of impact parameter p and frequency. File is a direct access file and can be plotted using PLT\_IP. This file was created to allow comparison of models with interferometric data. 

\item[WR\_RTAU]    When true, CMF\_FLUX writes out $R(\tau_{Ref})$ as a function of impact parameter and frequency. The file is a direct access file and can be plotted using PLT\_IP. 

\item[TAU\_REF]    Reference optical depth for WR\_TAU option. Because of the Eddington-Barbier relation, a good choice for [TAU\_REF] is 1 (recall the value 2/3 arise after integrating over all rays [angles]) when you want to plot the characteristic radius of formation along each ray.

\item[WR\_dFR]    When true, CMF\_FLUX writes out {\color{red} dFR} as a function of R and frequency. dFR(I) is the flux contribution arising from the spherical volume bracketed by R(I) and R(I+1). Thus $\Sigma_R$ dFR yields the observed flux (assuming $d=1$\,kpc). The output file is a direct access file and can be plotted using PLT\_DFR.  This option can be used to determine over what regions a line originates. Unlike the EW and EP options in DISPGEN (which utilize the Sobolev, and other, approximations that only apply to emission lines) this option can be used for both absorption and emission lines.

\item[FRAC\_DOP\_OBS]    Indicates the spacing, in Doppler widths, across the center of a line in the Observer's frame. Installed to allow more points across the photospheric profile in O stars. Has no effect if (effectively) larger than [dV\_OBS\_PROF].

\item[GLOBAL\_PROF]    Indicates method for determining intrinsic line absorption profiles. Options are:
    
\begin{plt_list}{LIST\_VOIGT}

\item[NONE]    Options for each species determined by [PROF\_XzV]. 
\item[DOP\_FIX] Fixed Doppler width for all species. 
\item[DOPPLER]    Species and depth dependent variable Doppler width.
\item[HZ\_STARK]    Stark profile (convolved with Doppler profile) for H I and He II. 
\item[LIST]    Profile options for individual lines are specified in the file FULL\_STRK\_LIST.
\item[LIST\_VOIGT]  Use the option in FULL\_STRK\_LIST when available. Uses a Doppler profile for �weak� lines.
    Uses VOIGT profile with radiative damping for strong lines.
    
\end{plt_list}

\item[PROF\_XzV]    Profile for individual ionization stage. Only utilized when [GLOBAL\_\linebreak PROF]=NONE.

\item[SCL\_ABUND\_DUM]    Option to scale the abundance of an impurity species (NOT H or He). A default value of 1 (i.e., no scaling) is used when the option is not present. This option was inserted to facilitate spectral analysis by allowing the effect of abundance changes to be quickly gauged. As a simple scaling may not be appropriate, DO NOT use this option for the final analysis --- instead you should use CMFGEN to compute a new model.  Be careful -- I would use a distinct names for batobs.sh and  CMF\_FLUX\_PRAM\_INIT to avoid scaling abundances accidentlly.

\end{mylist}

\vfill
\eject
\mhd{DISPGEN }{dispgen}   
       Display package for examining the model atomic populations, plotting, test calculations etc. Routine requires MODEL, RVTJ, POPDUM files, and atomic data files. The code and options are more fully discussed in \$cmfdist/web/dispgen.htm.
       
       This routine is powerful, and easy to modify. Dispgen.f is the main program -- options are performed in {\bf \$cmfdist/disp/{\-}maingen.f}. Using this package it is possible to:
           
\begin{enumerate}
\item
Plot T, V, departure coefficients, populations, ionization fractions etc. versus a variety of parameters (e.g., log r/\Rstar, Ne, column density, depth etc.) 

\item
Compute approximate equivalents for individual transitions. 

\item
Compute net-radiative brackets, mean intensities etc. 

\item
Compare different models.
\end{enumerate}
       
       A basic philosophy of DISPGEN is that it is OPTION driven. Each option passes data to GRAMON\_PGPLOT, the plotting package, via a call to CURVE. The data is generally not plotted until the GR option (the default) is given. This allows many different curves to be placed on the same plot.

DISPGEN requires the atomic data files used by CMFGEN in computing the model. Soft links to the data files can be obtained by entering 
\blankhalf
\hbox{} \hspace{0.5in}    {\bf batch.sh ass}
\blankhalf
\noindent
where batch.sh was the shell script used to run the model under consideration. The internal help does not work -- use the supplied web documentation.
       
\blankhalf
DISPGEN will create a lot of files of the form
\blankhalf
   \hbox{} \hspace{0.5in} *.sve. 
\blankhalf
 These are used so that an option can be repeated. For example,
\blankhalf
    \hbox{} \hspace{0.5in}    .EW\_CIV
\blankhalf
will use the same parameters last used by the same option (NB: ew\_civ is the same option as EQ\_CIV, but the .sve filename will be different. Hence, .EW\_CIV and .ew\_civ are not equivalent, since they will use parameters from distinct sve files). In DISPGEN options must match EXACTLY. The shell command {\bf dsve} will remove ALL files of the form *.sve .
       
\blankhalf
In addition, 
\blankhalf
    \hbox{} \hspace{0.5in}     *.box 
\blankhalf
files can be created. These can contain a list of options, with one option per line, that are repeated in sequence  (e.g., .EW\_CIV, EW\_NV). This is more useful with the PLT\_SPEC program.
       
\blankhalf
Commands beginning with x (e.g., xtemp) set the default X-axis but do not generate any plots. XTEMP, for example, sets the X-axis to temperature in units of $10^4$\,K. Most other (y options) then use this as the default X-axis. A few options, because of their nature, use an alternate axis. 

DISPGEN operates on only one model at a time, and thus, by itself, cannot compare different models. However, it is fairly trivial to overplot a second model using the WP and RP options in gramon\_pgplot. Simply run DISPGEN in alternative window for the second model (I often set the devise to /null for this
running of DISPGEN). Use this model to create a plot, and send the plot to a scratch, direct access, output file (def is PLT\_SCR) using the WP option. This plot is given a name, and can be read in by referring to this name using the RP option. ``?" returns the labels, of all known plots. Make sure that the two DISPGEN programs are accessing the {\color{red} SAME} PLT\_SCR file.

\vfill
\eject
\mhd{PLT\_SPEC}{plt_spec}    
       
       Display package for plotting model and observed fluxes. The use of {\bf plt\_spec} and its options are more fully discussed in {\bf \$cmfdist{\-}/web/plt\_spec.htm.} %The help files can be made available to PLT\_SPEC by issuing the shell script astxt in the current directory. This command should also be issued to make available the data for the interstellar lines.
PLT\_SPEC contains a single plot buffer -- some commands alter the contents of this buffer while other commands move the buffer contents (sometimes with modification) to the buffers associated with the general plotting package (GRAMON\_PGPLOT).
       
The philosophy behind PLT\_SPEC is similar to that behind DISPGEN except options can have extensions. Thus 
\blankhalf
\hbox{}\indent    RD\_MOD \\
\hbox{}\indent      RD\_MOD1 \\
\hbox{}\indent      RD\_MOD\_T32
\blankhalf
are all valid versions of the RD\_MOD option (which allows model data to be input). Each will generate its own unique save file. Remember that the data in obs\_fin (\& OBSFLUX) assume d=1kpc, which is the default assumed by PLT\_SPEC.
       
With PLT\_SPEC it is possible to redden model data using a variety of extinction curves. It is also possible to make a crude allowance for interstellar H and H$_2$, to apply a velocity shift, and to rectify data using a computed continuum. Further, it is possible to smooth the data to match the instrumental resolution, and to modify the observed spectrum for the effects of stellar rotation for a given $v \sin i$. The philosophy behind PLT\_SPEC is that model data is ``altered" to match the observations --- the observed data is not altered.
       
The most used options are RD\_MOD, RD\_OBS, CNVLV, FLAM, NORM, and ROT which are discussed below.
       
\begin{plt_list}{RD\_MOD}

\item[RD\_MOD]  Used to read in model data (e.g., obs\_fin or OBSFLUX). The X \& Y axes will be determined by the XU (default is wavelength in \AA) and YU (default is Jy) commands. RD\_MOD immediately sends the data to the plot package (the assumed distance will be 1\,kpc) unless a hidden keyword (OVER) is set to T. If the model data is to be operated on, it must be read in with the OVER keyword set to true {i.e, RD\_MOD(OVER=T) }. This will allow reddening corrections etc. to be done BEFORE the data is sent to the plot package. There is only one data buffer in PLT\_SPEC --- the data can be easily read in repeatedly (if necessary) using the .RD\_MOD option. Options which alter the data in the buffer include ROT (allow for rotation of the star), ISABS (apply H$_2$ and H absorption by interstellar medium), CNVLV (smooth data). The NORM option can be used to divide the data by the continuum (obs\_cont) in which case the default is to send the data in the buffer to the plot package. Alternatively the divided data can be saved in the buffer.      
       
\item[CNVLV] Allows smoothing of buffer spectrum -- either at fixed d$\lambda$ or at a fixed dV(\kms).

\item[FLAM] Sends the data in the buffer to the plot package. At the same time reddening corrections, and a change in the assumed distance, are allowed for. The data in the buffer are not affected.  NB: For historical reasons, the FLAM option has nothing to do with the units used for the y-axis -- that is set by the YU option.

\item[NORM] Produces a rectified spectrum by dividing the buffer (containing the model spectrum) by the continuum (as in obs\_cont).
                                  The data can be sent straight to PGPLOT, or the data buffer can be overwritten.

\item[ROT] Allows the buffer spectrum to be corrected for the effects of stellar rotation. The adopted default procedure is good for photospheric absorption features, but may fail for wind  lines, and photospheric emission lines. The data buffer is overwritten.
      
\item[ RD\_OBS] is used to read in observational data. Observational data is read in from text files in X,Y column format. As the default, X is assumed to be the wavelength, and Y the flux. Data can also be read in when in multi-column format. The default is to send the data immediately to the plot package, although it can also be read in to  the data buffer by setting the keyword OVER to true [i.e., RD\_OBS(OVER=T)]. At the top of each file several keywords can be specified. These are used to indicate units, and whether the data is to be scaled.

       The keyword {\color{OliveGreen}\bf FLUX\_UNIT=}{} is compulsory, and refers to the flux data in column 2. Comments can be listed at the top of the file before the keyword FLUX\_UNIT. I suggest using ``!'' in position 1 of a comment line to allow for possible format changes. Possible values for FLUX\_UNIT are as follows:

   \begin{tabular}{l}
       {ergs/cm$^2$/s/Ang} \\
      {egs/cm$^2$/s/Hz}  \\
        {mJy}\\ {Jy} \\ {Jansky} \\
        {norm}. 
\end{tabular}

On input the data is converted to Jy, but the {\bf YU} command in PLT\_SPEC allows the default y-axis to be changed. With the {norm} unit, no change is made. This is useful for reading in rectified data. {\bf XU} allows the x-axis unit to be changed.
       
Non compulsory keywords:

\begin{plt_list}{SCALE\_FACTOR}
 \item[WAVE\_UNIT] Angstroms or Micrometers or UM or Hz
  \item[AIR\_LAM] TRUE or FALSE ---
       Used to indicate whether wavelengths \\ $> 2000$\,\AA\  are in air. 
\item[SCALE\_FACTOR] Number ---
            scale factor applied to the flux data. 
\item[FLUX\_UNIT\_n] Used to refer to the data in column n. 
\end{plt_list}


       Multiple data sets can be included in a single file. These  MUST be separated by at least one row of �*********�. Each data set must also have its own set of keywords which MUST begin with FLUX\_UNIT=.
       
\end{plt_list}

       

\vfill
\eject
\mhd{TLUSTY\_VEL}{tlusty}
       
Designed to allow CMFGEN to use the hydrostatic structure from a TLUSTY run. To do this a file containing R, V and SIGMA (called RVSIG\_COL for convenience) must be generated using TLUSTY\_VEL.EXE (or some other program). The [VEL\_LAW] is set to 7, [VEL\_OPT] to RVSIG\_COL, and [VINF] to the largest value in RVSIG\_COL. The TLUSTY file *.11 (e.g., S40000g400v10.11) is required by TLUSTY\_VEL.EXE. Basically a $\beta$-type velocity law is matched to the hydrostatic structure so that the velocity law, and its first derivative, are continuous. Ideally, this matching should be done around 1/3 to 1/2 times the sound speed.

{Parameters needed for TLUSTY\_VEL}

\begin{green_list}{Beta Scale Height}
\item[\Rstar]          Innermost radius of star (in \Rsun)  
\item[Rmax]         Outer radius of star (in \Rstar)  
\item[Mdot]          Mass-loss rate (in \Msunyr)  
\item[Vinf]             Terminal velocity (in \kms). Depending on Rmax and $\beta$  
               this value may be slightly smaller than the value in VADAT  
\item[Beta\_out]       Value for $\beta$ in outer wind -- $\beta$ is the parameter in the classic velocity law $V_{\infty}(1-r/ R_*)^\beta$.  
\item[Beta\_in]         Value for $\beta$ in the inner wind. This was included  to allow  the wind velocity to be joined to the hydrostatic velocity law at velocities close to the sound speed (i.e., 1/3 to 1/2). Low $\beta$ ($<1$) can yield a matching velocity of $< 1$\,\kms.  
\item[Beta\_h]     Indicates how quickly to switch from the inner $\beta$ to the outer $\beta$ (in \Rstar).  
\end{green_list}


The first 4 parameters must be consistent with those in VADAT. Usually we choose the innermost point to have an optical depth slightly less than 100. Three options can be used to determine the grid (which will be the grid used by CMFGEN). Basically one grid is adopted for the photosphere below 1 km\,s$^{-1}$, and a second grid is utilized above 1 km\,s$^{-1}$. The new default method appears to work well, and is recommended. It is a hard coded merged version of the other 2 options described below.

The two earlier methods are retained for compatibility; one is based on density, the other on optical depth. Neither worked too well in all cases. For O stars, I generally found that the best depth grid was created by combining the bottom 2/3 of one grid (ND=60) with the top 1/3 of the grid constructed with the alternate option (ND=60). This yields good coverage in velocity and optical depth at all locations in the atmosphere, with (typically) ND=67.   


\vfill
\eject
\mhd{GRAMON\_PGPLOT}{gramon_pgplot}

Basic plotting subroutine which is option driven. Basic calling method can be seen in PLT\_SPEC or DISPGEN. Package has a lot of options for making pretty plots. At present it does not allow multiple panels on the same page. This is not a limitation --- postscript files can easily be combined to give a panel format either by editing, or by running simple scripts (programs). N\_COL\_MERGE.EXE and N\_MULTI\_MERGE.EXE are two Fortran programs designed to facilitate plot merging. ``.sve'' files are not utilized by GRAMON\_PGPLOT.

\blankhalf
Data is passed to GRAMON\_PGPLOT by calls of the form

\hbox{}\indent\indent    CALL DP\_CURVE(NPTS,XVEC,YVEC)

\blankline
or similar variants. Up to 50 plots, of �arbitrary� length, can be passed. Error bars can also be passed. The plot package is called by
        
\hbox{}\indent\indent         CALL GRAMON\_PGPLOT(XLABEL,YLABEL,TITLE,OPTIONS)

\blankline
All arguments are CHARACTER, and may be blank.

\blankhalf
The basic philosophy is that the plot package should provide reasonable default plots. These defaults can then be modified to make pretty plots. Once the user is happy with a plot, it can be written to a hard file using the Z option. For pretty plots with strings, it is advisable to set an explicit aspect ratio (the default aspect ratio is device dependent). When this is done, plots on the screen will identical (except for the background color) to those printed.

\blankhalf
It is well worth learning how to use this plot package --- if nothing else it provides a quick and dirty method
of examining CMFGEN output. All of CMFGEN's (plotting) programs utilize this package. For most
purposes you need to know less than half-a-dozen commands.

\blankhalf
{\noindent \color{blue} Plot options}

\begin{plt_list}{WXY}
\item[H]   Help
\item[P]   Plot graphs (default)
\item[NOI]    Leave data intact on exit (actually a switch). Default is to destroy data on exit. Make sure to cancel the VEL option before issuing this command. 
\item[LP]  Allow a long postscript plot to be created (device CPS only).
\item[E]    Exit from PLOT package
\item[CL]    Clear Graphics Screen
\item[Z]   Hardcopy (ZN=Asks for new hard device). Plots are automatically numbered as pgplot.ps, pgplot\_2.ps etc.
\end{plt_list}

\vfill
\eject
\blankhalf
{\noindent \color{blue} Axis and plot format}

\begin{plt_list}{WXY}
\item[A]    Define basic axis Parameters (Xstart, Xend etc)
\item[2A]    Define parameters for axis on right hand side.
\item[F]    Change default axis parameters (to make pretty plots)
\item[L]    Modify axis labels and titles
\item[N]   Change size of labels, tick marks, and plot borders. Set aspect ratio of plot.
\item[LY]    Switch between LINEAR/LOG Y axis labeling.
\item[LXY]    Switch between LINEAR/LOG labeling for X and Y axes
\end{plt_list}

\blankhalf
{\noindent \color{blue} Line styles }
   
\begin{plt_list}{WXY}
\item[B]    Switch error bars on/off
\item[C]   Indicate how curves are to be connected
\begin{plt_list_2nd}{WXY}
            \item[L]    Normal line 
            \item[E]    Non-monotonic
            \item[B]    Broken
            \item[I]     Invisible
            \item[V]    Vertical lines 
            \item[A]    Histogram - assume the X values are the vertices of the box.
            \item[H]    Histogram-  box is drawn from [x(i-1)+x(i)]/2 to [x(i)+x(i+1)]/2.
            \item[LG]  Plots $ \log | {\hbox{\rm Data}} |$ but indicates by distinct marker data that was originally negative. 
\end{plt_list_2nd}
\item[CC]  Change color settings
\item[CP]    Change pens (Color Index)
\item[D]    Switch dashed lines on/off
\item[DE]    Edit dashed lines one by one
\item[W]  Change thickness (line weights) of curves
\item[WE]    Edit line weights one by one
\item[M]   Switch marking of data points on/off
\end{plt_list}

\blankhalf
{\noindent \color{blue} Spectral line measurements and options}

\begin{plt_list}{WXY}
\item[DC]    Define a straight line continuum for EW.
\item[EW]    Measure the EW or AREA of a single line in a plot. If continuum has not been previously defined, the continuum is assumed to be normalized to unity.
\item[GF]  Fit a set of Gaussians (with an exponent not necessarily=2) to a section of normalized spectrum. The parameters of the Gaussian, and the EWs are output. Rerunning GF allows previous fit parameters to be used/edited.
\item[DG]    Draw the Gaussian fits (in black) (allows the plot screen to be cleared and updated.
\item[MGF]    Automatically fit a set of Gaussians to multiple plots. Start parameters are obtained either from a previous fit using GF, or from a file which lists previous fits obtained using GF. Two results files, with different formats, are output.
\item[EGF]  Edit (adjust) the Gaussian fits by hand.
\item[RID]  Read in a line list that has been created by DISPGEN(LNID). This can be used to label plots. It works best in the optical region where the line density is not too high. The default labeling is for O star spectra in which the continuum is normalized, and where the y-range must cover the interval 1 (or $< 1$) to 1.2.  The [SID] option can be used to the default size/location parameters.

\end{plt_list}

\vfill
\eject
\blankhalf
{\noindent \color{blue} Line and string options}

\begin{plt_list}{WXY}
\item[VC]    Define line vectors on the plot using cursor control.
\item[VF]    Read vector definitions from a file.
\item[VE]    Provides an interactive edit of vectors (colors, size, location etc.).
\item[SC]    Define strings on the plot using cursor control. String location is done by the numeric keypad (1 to 9).
\item[SF]    Read string definitions from a file.
\item[SE]    Provides and interactive edit of strings (colors, size etc.).
\end{plt_list}

\blankhalf
{\noindent \color{blue} Data IO}

\begin{plt_list}{WXY}
\item[WP]    Write plots to a direct access file. Plots are labeled, and these labels are subsequently used by RP to access the plot. This option and RP can be used to compare plots from different models. RUN DISPGEN in two windows, and use these options to transfer data between the programs. One program can open a null window for plotting.
\item[WPF]    As for WP but file name can be changed.  
\item[RP]       Read plots from a (WP) direct access file. WP and RP are useful to transfer plots between different programs or models.
\item[RPF]    As for RP but file name can be changed.
\item[WXY]   Write a simple ASCII data file in column format.
\item[RXY]    Read a simple data file that is in ASCII format.
\item[SXY]    Writes ASCII format in column format with id, I, and x(i), y(i) given sequentially for all plots.
\end{plt_list}


\blankhalf
{\noindent \color{blue} Simple plot manipulation.}

\begin{plt_list}{WXY}
\item[NM]   Scale plot level to unity, or to match another plot.
\item[VEL]    Convert X axis to use km\,s$^{-1}$. Entering 0 will return you to the original input axis.
\item[XAR]    Simple X axis arithmetic with a constant (+, -, / ,*, LG, ALG, R [=const./X]). The default constant allows you to automatically switch between wavelength (angstroms, and assuming vacuum) and units of $10^{15}$\,Hz (or vise-versa).
\item[YAR]    Simple Y axis arithmetic with a constant (+, -, / ,*, LG, ALG, R [=const./Y])
\item[VAR]    Simple arithmetic on two plots (+, -, / ,*). If the X-axes are distinct, the data is interpolated onto a common axis.
\end{plt_list}

\blankhalf
{\noindent \color{blue} History mechanism}

\begin{plt_list}{WXY}
\item[OLF]    Open file to log commands.
\item[CLF]     Close log file.
\item[OIF]    Open a previously generated log file for input. Somewhat cumbersome, as a few commands can�t be used in this mode (e.g., doing the first hardcopy plot), and previously issued commands can be important.
Very useful for generating plots for a sequence of models.
\item[CIF]    Return to terminal IO.
\end{plt_list}

\vfill
\eject

\shd{MAIN\_LTE}{main_lte}

Used to generate a table of Rosseland mean opacities. See README file, and examples, in \$cmdist/lte\_hydro. The opacity table is required by WIND\_HYDRO, and if using DO\_HYDRO option in CMFGEN. One minor bug is that the code can spend a long time iterating on Ne if Tmin is set too low, and if the corresponding low ionization species are not included. The code can also crash if you have a low Tmin and very high ionization stages. The opacity table is used when adjusting the density and temperature structure to satisfy hydrostatic equilibrium --- using the wrong table may affect convergence but does not affect the final solution if convergence is obtained.

To run {\color{blue} main\_lte.exe} create a sub-directory {\bf lte} in the model directory. In this directory you need to include VADAT and MODEL\_SPEC since these indicate which species and ionizations stages will be considered, and the species abundances. These files should be identical to those to be used in the model --- it is no longer necessary to edit MODL\_SPEC. You also need GRID\_PARAMS and ltebat.sh. These later files usually do not need to be altered. GRID\_PARAMS may need to be altered when switching from O stars to WDs (for example) because the density and temperature range specified in GRID\_PARAMS does not cover the desired range. An example GRID\_PARAMS file is shown below.\

\blankline
\centerline{GRID\_PARAMS}
\blankhalf

\begin{mynarrow}{0.2in}
\begin{tabular}{lll}
 25 &  37              &  !\# of T values; \# of Electron density values \\
1.0 & 15.0             & !Tmin, Tmax (10$^4$\,K) \\
1.0E+06 \hbox{  } & 1.0E+18 \hbox{  } &     !Ne(min), Ne(max)
\end{tabular}
\end{mynarrow}

\blankline
\blankline
At the end of the calculation a lot of extraneous files exist. For most purposes, the only useful file
is ROSSELAND\_LTE\_TAB -- all other output files can be deleted. The execution of this code is greatly facilitated by using multiple processors.

\vfill
\eject
\blankline
\shd{DO\_NG\_V2}{do_ng_v2}
    Reads in SCRTEMP file and does an Ng acceleration. Useful when manually assisting a model to converge. SCRTEMP is automatically updated with the new population estimates, however you must stop and restart the model. Various options are available.

\begin{plt_list}{UNDO}
\item[NG] Performs an Ng acceleration. Step size, bandwidth, and depths can be specified.
\item[AV] Averages the last 2 iterations. This is very useful if some populations are oscillating.
\item[SOR] Applies the last set of corrections times SCALE\_FAC. The maximum correction is limited to BIG\_FAC.
\item[NSR] Applies the last set of corrections K times as a geometric series. That is, \\
\hbox{} \hspace{30pt} X0=X1*(1+T)**K where T=(X1-X2)/X2.
\item[REP] Repeats the correction as defined by the last iteration, and a previous iteration (input by the user).
\item[UNDO] Replaces a set of depths by the previous iteration.
\end{plt_list}

Recently an automatic plot was added to illustrate the T corrections arising from the NG acceleration, and the T correction arising from the last normal iteration. If the two plots are similar, apart from a simple scaling (you can use the YAR option in PGPLOT to scale the second plot), the NG acceleration is likely to be effective in accelerating the convergence. if the two plots are very different (i.e., they contain significant regimes of different sign) the acceleration should be cancelled (e.g., by \^C). The comparison has been found to be very useful with supernovae models. So far, I have limited experience with non SN models.

For stability reasons, I strongly recommend that you perform a LAMBDA iteration when restarting the model. This can be done by setting both                  [DO\_LAM\_IT] and [DO\_LAM\_AUTO]   to TRUE in IN\_ITS

Note: You can (Generally) run do\_ng\_v2 while the model is running -- the model only needs to be stopped when you accept the results of the NG acceleration (i.e., POINT1 and SCRTEMP are updated).

\shd{GUESS\_DC}{guess_dc}

Allows new input files to be created with estimates of the departure coefficients. Requires an EDDFACTOR file with a previously computed radiation field, and all the atomic files need to be assigned (via batch.sh). For adding an entire new species, start with the lowest ionization stage first. This is now the preferred method for adding additional atoms and species. After generating, it is recommended that [USE\_FIXED\_J] be set to TRUE, and several $\Lambda$--iterations be performed. [USE\_FIXED\_J] should then be set to FALSE, and EDDFACTOR/EDDFACTOR\_INFO deleted. This procedure is now handled automatically --- CMFGEN will revert to normal iterations once sufficient convergence has been achieved using [USE\_FIXED\_J] and normal $\Lambda$--iterations.
 
\blankline
\shd{LAND\_COL\_MERGE \& LAND\_MULTI\_MERGE}{land_col_merge}  Merge several LANDSCAPE pgplots together to make a single postscript file. The plots must have the correct size/aspect ratio if merging is to look correct. Use CPS mode in GRAMON to create the figures. LAND\_MULTI\_MERGE allows the creation of multiple column plots.

\blankline
\shd{MOD\_COOL}{mod_cool}
Program to rewrite the GENCOOL file into a more convenient format. Two files are written:
\begin{plt_list}{GENCOOL\_SUMmm}
\item
[GENCOOL\_SUM]   Same format as GENCOOL except that we have summed up over all bound-free rates. \\
\item
[GENCOOL\_SORT] Similar format to GENCOOL except only the top N rates are printed. The rates are sorted using depths 1, 11, 21 etc. Thus, at depth 20, for example, the rates will not be in order. This file is very useful.
\end{plt_list}

\blankline
\shd{MOD\_PRRR}{mod_prrr}
 Program to rewrite the XzVPRRR file, for a given species, in a more convenient summary format (output file is
 XzVPRRR\_SUM). Routine also allows the rates, as a function of depth, to be plotted. This allow an
 easy way to check which processes are directly important for controlling the ionization structure.
 
\blankline
\shd{N\_COL\_MERGE \& N\_MULTI\_MERGE }{n_col_merge}  \hfill
As for LAND\_COL\_MERGE but creates figures in PORTRAIT mode. Landscape mode is still used to generate the raw postscript files. N\_MULTI\_MERGE allows figures in multiple rows/columns to be created.

\blankline
\shd{PLT\_CMF\_LUM}{plt_cmf}
 Plot comoving-frame luminosity and auxiliary vectors from OBSFLUX. Program is designed to check how well the
 ``conserved luminosity" is conserved. Program plots $Dr^3/Dt$, mechanical term, radioactive energy deposition, 
 and the gas term.  Note: For a homologous flow, $Dr^4/Dt$=$Dr^3/Dt$ + mechanical term.
 
\blankline
\shd{PLT\_IP}{plt_ip} Plot data in IP\_DATA file. The IP\_DATA file contains the intensity data as a function of impact parameter and frequency. It can be created using CMF\_FLUX. 

\blankline
\shd{PLT\_JH}{plot_jh}  Plot data from a variety of scratch files created with CMFGEN and CMF\_FLUX. Multiple files can be read in, using the RD\_MOD option. Data files that can be read include \linebreak EDDFACTOR, ES\_J\_CONV, CHI\_DATA, ETA\_DATA, FLUX\_DATA, CMF\_FORCE\_DATA, and SOB\_FORCE\_DATA. Be careful with the Y-axis label, which is not always correct.

\blankline
\shd{PLT\_JH\_CUR}{plot_jh_cur}  Plot J and H moments contained in the supernovae data file JH\_AT\_CURRENT\_TIME (or JH\_AT\_OLD\_TIME).  Multiple files can be read in, using the RD\_MOD option. Options are similar to PLT\_JH.

\blankline
\shd{PLT\_RJ}{plot_rj}
 Plot data (ie., J) in EDDFACTOR file and/or ES\_J\_CONV file. Superceded by PLT\_JH.

\blankline
\shd{PLT\_SCR}{plt_scr}
 Plot data in SCRTEMP file, which contains the populations as a function of iteration. A variety of options are available to illustrate convergence.
 These can, for example,plot both populations and corrections as a function of depth and iteration. Options are listed when you run PLT\_SCR.

\blankline
\shd{REWRITE\_DC}{rewrite_dc}
 Modify departure coefficient file to allow for level splitting. Primarily used to split individual LS levels into J states. Requires 2 oscillator files for use.
 
 \shd{WIND\_HYD}{wind_hyd}
WIND\_HYD can be used to generate a RVSIG\_COL for use with a new CMFGEN model. Alternatively you
can use TLUSTY\_VEL, or an existing RVSIG\_COL file from a model (especially when generating
a model sequence of grid). See README file, and examples, in \$cmdist/lte\_hydro. A table of Rosseland mean opacities must be supplied. The density structure is assumed to be hydrostatic below the sonic point, and specified by the velocity law above the sonic point.

\blankline
\shd{WR\_F\_TO\_S}{wr_f_to_s}

Generates file containing links between FULL and SUPER levels. The oscillator file is required as input. Can also read old F\_TO\_S link files.  Old F\_TO\_S files can be edited by hand to create new super levels. WR\_F\_TO\_S can be used to clean these files so that super levels are consecutively numbered. Unfortunately WR\_F\_TO\_S does not currently operate on the interpolation column which is in a few F\_TO\_S files (e.g., He2\_F\_TO\_S).

\end{document}


